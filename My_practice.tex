%%%% Шаблон Отчета по практике <<SPbPU-student-thesis-template>>  %%%%
%%
%%   Создан на основе глубокой переработки шаблона российских кандидатских и докторских диссертаций [1]. 
%%   
%%   Полный список различий может быть получен командами git.
%%   Лист авторов-составителей расположен в README.md файле.
%%   Подробные инструкции по использованию в [1,2].
%%   
%%   Рекомендуем установить TeX Live + TeXstudio
%%   <<Стандартная>> компиляция 2-3 РАЗА с помощью pdflatex + biber (для библиографии)     
%%  
%%%% Student thesis template <<SPbPU-student-thesis-template>> %%%%
%%
%%   Created on the basis of deep modifification of the Russian candidate and doctorate thesis template [1]. 
%%   
%%   Full list of differences can be achieved by git commands.
%%   List of template authors can be seen in the README.md file.
%%   Detailed instructions of usage, see, please in [1,2].
%%     
%%   [1] github.com/AndreyAkinshin/Russian-Phd-LaTeX-Dissertation-Template 
%%   [2] Author_guide_SPBPU-student-thesis-template.pdf
%%   
%%   It is recommended to install TeX Live + TeXstudio   
%%   Default compilation 2-3 TIMES with pdflatex + biber (for the bibliography)
%%  
%%%% Preamble start %%%%  
%%
%%   Please, do not modify files in the preamble
%%
\newcommand*{\anyptfilebase}{template_settings/bpfont} 
\newcommand*{\anyptsize}{14} 		 
\RequirePackage[l2tabu,orthodox]{nag} 
\documentclass[extrafontsizes,a4paper,*pt,oneside,openany]{memoir}
%%%%%%%%%%%%%%%%%%%%%%%%%%%%%%%%%%%%%%%%%%%%%%%%%%%%%%
%%%% Файл упрощённых настроек шаблона диссертации %%%%
%%%%%%%%%%%%%%%%%%%%%%%%%%%%%%%%%%%%%%%%%%%%%%%%%%%%%%

%%% Инициализирование переменных, не трогать!  %%%
\newcounter{intvl}
\newcounter{otstup}
\newcounter{contnumeq}
\newcounter{contnumfig}
\newcounter{contnumtab}
\newcounter{pgnum}
\newcounter{chapstyle}
\newcounter{headingdelim}
\newcounter{headingalign}
\newcounter{headingsize}
\newcounter{tabcap}
\newcounter{tablaba}
\newcounter{tabtita}
\newcounter{docType} 		% тип документа
\newcounter{tskPrint} 		% печать Задания на ВКР двух(одно)сторонняя
\newcounter{tskPages}       % для учёта количества страниц в Задании
\newcounter{tskPageFirst}   % для учёта количества страниц в Задании
\newcounter{tskPageLast}    % для учёта количества страниц в Задании 
\newcounter{sumPrint} 		% печать Реферата на ВКР двух(одно)сторонняя
\newcounter{sumPages}       % для учёта количества страниц в Реферате
\newcounter{sumPageFirst}   % для учёта количества страниц в Реферате
\newcounter{sumPageLast}    % для учёта количества страниц в Реферате 
\newcommand{\Single}{0.78}  % пропорция для одинароного отступа в \Spacing
%%%%%%%%%%%%%%%%%%%%%%%%%%%%%%%%%%%%%%%%%%%%%%%%%%

%%% Область упрощённого управления оформлением %%%

% Управление перенесено в главые файлы компиляции ВКР, Задания, Реферата
\setcounter{tskPrint}{0} %по умолчанию односторонняя печать              
%\setcounter{sumPrint}{0} %по умолчанию односторонняя печать 

%% Интервал между заголовками и между заголовком и текстом
% Заголовки отделяют от текста сверху и снизу тремя интервалами (ГОСТ Р 7.0.11-2011, 5.3.5)
\setcounter{intvl}{3}               % Коэффициент кратности к размеру шрифта

% Заголовки отделяют от текста сверху и снизу тремя интервалами 
\newcommand{\intvlS}{1.5}               % Коэффициент кратности к размеру шрифта SPbPU-student-templates

\newcommand{\intervalS}{\vspace{\intvlS\curtextsize}}

% печать списка источников в Задании
\newcommand{\printbibliographyTask}{\vspace{-0.28\curtextsize}
	\printbibliography[env=tsk] % печать списка литературы в исходных данных
	\vspace{-0.28\curtextsize}}


%% Отступы у заголовков в тексте
\setcounter{otstup}{0}              % 0 --- без отступа; 1 --- абзацный отступ

%% Нумерация формул, таблиц и рисунков
\setcounter{contnumeq}{0}           % Нумерация формул: 0 --- пораздельно (во введении подряд, без номера раздела); 1 --- сквозная нумерация по всей диссертации
\setcounter{contnumfig}{0}          % Нумерация рисунков: 0 --- пораздельно (во введении подряд, без номера раздела); 1 --- сквозная нумерация по всей диссертации
\setcounter{contnumtab}{0}          % Нумерация таблиц: 0 --- пораздельно (во введении подряд, без номера раздела); 1 --- сквозная нумерация по всей диссертации


%% Нумерация подстраничных сносок (ссылок)
%сквозная
\counterwithout{footnote}{chapter} %сквозная нумерация подразделов (во всех главах)


%% Нумерация подразделов
%убрать номер главы в секции
%\counterwithout{section}{chapter} %сквозная нумерация подразделов (во всех главах)
%\renewcommand\thesection{\arabic{section}} %в каждой главе нумерация заново

%\renewcommand\thesection{\arabic{section}}
%\renewcommand\thefigure{\fbox{\arabic{figure}}}
%\renewcommand\thetable{\arabic{table}}
%\renewcommand\theequation{\arabic{equation}}



%\counterwithout{section}{chapter}
%\counterwithout{figure}{chapter}
%\counterwithout{table}{chapter}
%\counterwithout{equation}{chapter}

%\counterwithin{section}{chapter}
%\counterwithin{figure}{chapter}
%\counterwithin{table}{chapter}

%% Оглавление

\setcounter{pgnum}{1}               %NB УДАЛЕНО ФИЗИЧЕСКИ 0 --- номера страниц никак не обозначены; 1 --- Стр. над номерами страниц (дважды компилировать после изменения)  
\settocdepth{subsection} %             до какого уровня подразделов выносить в оглавление
\setsecnumdepth{subsubsection}         % до какого уровня нумеровать подразделы


%% Текст и форматирование заголовков
\setcounter{chapstyle}{1}           % 0 --- разделы только под номером; 1 --- разделы с названием "Глава" перед номером
\setcounter{headingdelim}{2}        % 0 --- номер отделен пропуском в 1em или \quad; 1 --- номера разделов и ений отделены точкой с пробелом, подразделы пропуском без точки; 2 --- номера разделов, подразделов и приложений отделены точкой с пробелом.

%% Выравнивание заголовков в тексте
\setcounter{headingalign}{0}        % 0 --- по центру; 1 --- по левому краю

%% Размеры заголовков в тексте
\setcounter{headingsize}{0}         % 0 --- SPbPU style, все всегда 14 пт; 1 --- пропорционально изменяющийся размер в зависимости от базового шрифта;

%% Подпись таблиц
\setcounter{tabcap}{1}              % 0 --- по ГОСТ, номер таблицы и название разделены тире, выровнены по левому краю, при необходимости на нескольких строках; 1 --- подпись таблицы не по ГОСТ, на двух и более строках, дальнейшие настройки: 
%Выравнивание первой строки, с подписью и номером
\setcounter{tablaba}{2}             % 0 --- по левому краю; 1 --- по центру; 2 --- по правому краю
%Выравнивание строк с самим названием таблицы
\setcounter{tabtita}{1}             % 0 --- по левому краю; 1 --- по центру; 2 --- по правому краю
%Разделитель записи «Таблица #» и названия таблицы
\newcommand{\tablabelsep}{space}   % space = пробел, period =  (определены в подключенных пакетах)

%% Подпись рисунков
%Разделитель записи «Рисунок #» и названия рисунка
\newcommand{\figlabelsep}{period}   % emdash = тире, определён в common/styles; period = точка определён в подключенных пакетах; space
%\newcommand{\figlabelsep}{emdash}   % emdash = тире, определён в common/styles; period = точка определён в подключенных пакетах


%%% Цвета гиперссылок %%%
% Latex color definitions: http://latexcolor.com/

%\definecolor{linkcolor}{rgb}{0.9,0,0}
%\definecolor{citecolor}{rgb}{0,0.6,0}
%\definecolor{urlcolor}{rgb}{0,0,1}


%\definecolor{linkbordercolor}{rgb}{0,0,1}

\definecolor{linkcolor}{HTML}{FF0000} %very light red from the SPbPU brandbook (2nd level)
\definecolor{citecolor}{HTML}{6CF479} %very light green from the SPbPU brandbook (2nd level)
\definecolor{urlcolor}{HTML}{4481BA} %very light blue from the SPbPU brandbook (2nd level)

%\definecolor{linkcolor}{rgb}{0,0,0} %black
%\definecolor{citecolor}{rgb}{0,0,0} %black
%\definecolor{urlcolor}{rgb}{0,0,0} %black               
%%% Проверка используемого TeX-движка %%%
\usepackage{iftex}[2013/04/04]
\newif\ifxetexorluatex   % определяем новый условный оператор (http://tex.stackexchange.com/a/47579/79756)
\ifXeTeX
    \xetexorluatextrue
\else
    \ifLuaTeX
        \xetexorluatextrue
    \else
        \xetexorluatexfalse
    \fi
\fi

\RequirePackage{etoolbox}[2015/08/02]               % Для продвинутой проверки разных условий

%%% Поля и разметка страницы %%%

\usepackage{pdflscape}                              % Для включения альбомных страниц
\usepackage{geometry}                               % Для последующего задания полей

%%% Математические пакеты %%%
\usepackage{amsfonts,amsmath,amssymb,amscd,amsthm}  % Математические дополнения от AMS
% %amsthm should be loaded after amsmath!!

\usepackage{mathtools}                              % Добавляет окружение multlined

%%%% Установки для размера шрифта 14 pt %%%%
%% Формирование переменных и констант для сравнения (один раз для всех подключаемых файлов)%%
%% должно располагаться до вызова пакета fontspec или polyglossia, потому что они сбивают его работу
\newlength{\curtextsize}
\newlength{\bigtextsize}
\setlength{\bigtextsize}{13.9pt}

\makeatletter
%\show\f@size                                       % неплохо для отслеживания, но вызывает стопорение процесса, если документ компилируется без команды  -interaction=nonstopmode 
\setlength{\curtextsize}{\f@size pt}
\makeatother

%%% Кодировки и шрифты %%%
\ifxetexorluatex
    \usepackage{polyglossia}[2014/05/21]            % Поддержка многоязычности (fontspec подгружается автоматически)
\else
    \RequirePDFTeX                                  % tests for PDFTEX use and throws an error if a different engine is being used
   %%% Решение проблемы копирования текста в буфер кракозябрами
%    \input glyphtounicode.tex
%    \input glyphtounicode-cmr.tex %from pdfx package
%    \pdfgentounicode=1
    \usepackage{cmap}                               % Улучшенный поиск русских слов в полученном pdf-файле
    \defaulthyphenchar=127                          % Если стоит до fontenc, то переносы не впишутся в выделяемый текст при копировании его в буфер обмена
    
%    \usepackage[T2A]{fontenc}                       % Поддержка русских букв
    \usepackage[T2A,T1]{fontenc}
    \usepackage[utf8]{inputenc}[2014/04/30]         % Кодировка utf8
    \usepackage[english, russian]{babel}[2014/03/24]% Языки: русский, английский
\fi
\usepackage{tempora} %TemporaLGCUni of Times type
\usepackage{newtxmath} %math font of Times type
% need to set the monospace=typewritter font
%https://tex.stackexchange.com/questions/213835/using-many-typewriter-fonts-in-a-single-document

\makeatletter %load fonts for cmtt
\providecommand{\EC@ttfamily}[5]{%
	\DeclareFontShape{#1}{#2}{#3}{#4}{
		<-8.5>#50800
		<8.5-9.5>#50900
		<9.5-10.5>#51000
		<10.5-11.5>#51095
		<11.5-13>#51200
		<13-15.5>#51440
		<15.5-18.5>#51728
		<18.5-22>#52074
		<22-27>#52488
		<27-32>#52986
		<32->#53583}{}}
\DeclareFontFamily{T1}{cmtt}{}
\DeclareFontFamily{T2A}{cmtt}{}
\EC@ttfamily{T1}{cmtt}{m}{n}{ectt}
\EC@ttfamily{T1}{cmtt}{m}{sl}{ecst}
\EC@ttfamily{T1}{cmtt}{m}{it}{ecit}
\EC@ttfamily{T1}{cmtt}{m}{sc}{ectc}
\DeclareFontShape{T1}{cmtt}{bx}{n}%
{<->ssub*cmtt/m/n}{}
\DeclareFontShape{T1}{cmtt}{bx}{it}%
{<->ssub*cmtt/m/it}{}
\EC@ttfamily{T2A}{cmtt}{m}{n}{latt}
\EC@ttfamily{T2A}{cmtt}{m}{sl}{last}
\EC@ttfamily{T2A}{cmtt}{m}{it}{lait}
\EC@ttfamily{T2A}{cmtt}{m}{sc}{latc}
\DeclareFontShape{T2A}{cmtt}{bx}{n}%
{<->ssub*cmtt/m/n}{}
\DeclareFontShape{T2A}{cmtt}{bx}{it}%
{<->ssub*cmtt/m/it}{}
\makeatletter

%\makeatletter %load fonts for cmtt
%\providecommand{\EC@ttfamily}[5]{%
%	\DeclareFontShape{#1}{#2}{#3}{#4}{
%		<-8.5>#50800
%		<8.5-9.5>#50900
%		<9.5-10.5>#51000
%		<10.5-11.5>#51095
%		<11.5-13>#51200
%		<13-15.5>#51440
%		<15.5-18.5>#51728
%		<18.5-22>#52074
%		<22-27>#52488
%		<27-32>#52986
%		<32->#53583}{}}
%\DeclareFontFamily{T2A}{cmtt}{\hyphenchar\font\m@ne}
%\EC@ttfamily{T2A}{cmtt}{m}{n}{latt}
%\EC@ttfamily{T2A}{cmtt}{m}{sl}{last}
%\EC@ttfamily{T2A}{cmtt}{m}{it}{lait}
%\EC@ttfamily{T2A}{cmtt}{m}{sc}{latc}
%\DeclareFontShape{T2A}{cmtt}{bx}{n}%
%{<->ssub*cmtt/m/n}{}
%\DeclareFontShape{T2A}{cmtt}{bx}{it}%
%{<->ssub*cmtt/m/it}{}
%\makeatletter

%\makeatletter
%\input{t1lmtt.fd}
%\@namedef{T1+lmtt}{}
%\makeatother


\renewcommand{\ttdefault}{cmtt}
%\renewcommand{\ttdefault}{lcmtt} %покрупнее
%\usepackage[scaled=.85]{DejaVuSansMono} %слишком похож на рубленый
%\newfont{\wasyten}{wasy10} %название команды для вызова / название шрифта



%Другие шрифты:
% математика
%\usepackage[lite]{mtpro2}
%https://pctex.com/mtpro2.html
% текст        
% https://www.ctan.org/pkg/paratype
%       \usepackage[scaled=0.925]{XCharter}[2017/06/25] % Подключение русифицированных шрифтов XCharter
%\usepackage{pscyr}
%    \IfFileExists{pscyr.sty}{}{}  % Красивые русские шрифты
%\fi

%https://tex.stackexchange.com/questions/8260/what-are-the-various-units-ex-em-in-pt-bp-dd-pc-expressed-in-mm
\usepackage{printlen} %для измерения и вывода параменторов шрифтов, отступов, интервалов

\usepackage{bm} %для жирных начертаний символов

\usepackage{csquotes} %to check quotes

%%% Оформление абзацев %%%
\usepackage{indentfirst}                            % Красная строка

%%% Цвета %%%
%\usepackage[dvipsnames,usenames]{color}
\usepackage{colortbl}
\usepackage[dvipsnames, table, hyperref, cmyk]{xcolor} % Вероятно, более новый вариант, вместо предыдущих двух строк. Конвертация всех цветов в cmyk заложена как удовлетворение возможного требования типографий. Возможно конвертирование и в rgb.

%%% Таблицы %%%
\usepackage{longtable}                              % Длинные таблицы
\usepackage{multirow,makecell}                      % Улучшенное форматирование таблиц:
													% multirow - строки на несколько ячеек, 
												
													% makecell - сесколько строк в ячейке.
													% не работает, если внутри, например, \verb|text| -> \texttt{text}
													% аналоги
%https://tex.stackexchange.com/questions/2441/how-to-add-a-forced-line-break-inside-a-table-cell								
						
													

%%% Общее форматирование
%\usepackage{soul} % используется ulem
\usepackage{soulutf8}                               % Поддержка переносоустойчивых подчёркиваний и зачёркиваний
\usepackage{icomma}                                 % Запятая в десятичных дробях



%%% Предметный указатель  ГОСТ 7.78-99 Index %%%
%c обобщенными рубриками или развернутый
%или указатель терминов (в общем случае - произвольное число указателей)
%подключать до hyperref

%\usepackage{makeidx} %возможно, необходимо подключить И/ИЛИ пройти Tools-> Commands -> MakeIndex

\usepackage{imakeidx} 
%\indexsetup{level=\section*,toclevel=section,noclearpage}
\makeindex[program=makeindex,
options=-s template_settings/common/myindex.ist, %подключаем стилевой файл для форматирования вывода
name=ru, % префикс для русских указателей 
% если убрать <<ru>>, то для работы дефолтового придется вручную включать Tools-> Commands -> MakeIndex
title={\chapterLight{} 
%   \hrule{}
	Предметный указатель
%	\hrule{}
} 
%,columns=1 %по умолчанию 2
]
\makeindex[program=makeindex,
options=-s template_settings/common/myindex.ist, %подключаем стилевой файл для форматирования вывода
name=en, % префикс для английских указателей
title={\chapterLight{}
%	\hrule{}
	Index
%	\hrule{}
} 
%,columns=1 %по умолчанию 2
] 
%убрать добавление <<title>> в содержание:
%\noindexintoc %not to add index title in PURE makeidx %intoc is false by default with imakeidx


%       https://tex.stackexchange.com/a/132415/44348
%\makeatletter
%% we want hyphenation also in the first word
\renewcommand{\@idxitem}{\par\hangindent40\p@\hspace{0pt}\ignorespaces}
%% we don't want a page break before a subitem %implemented in the previous one
%%\renewcommand\subitem{\@idxitem\nobreak\hspace*{20\p@}}
%\makeatother


%%% Фиксация плавающих объектов





%%% Гиперссылки %%%
\usepackage{hyperref}[2012/11/06]

%%% Изображения %%%
\usepackage{graphicx}[2014/04/25]                   % Подключаем пакет работы с графикой

%%% Списки %%%
\usepackage[shortlabels]{enumitem} % shortlabels для того, чтобы изменять токены в списках с дефолтных (иерархическая структура) на произвольныею

%%% Подписи %%%
\usepackage{caption}[2013/05/02]                    % Для управления подписями (рисунков и таблиц) % Может управлять номерами рисунков и таблиц с caption %Иногда может управлять заголовками в списках рисунков и таблиц


\usepackage{subcaption}[2013/02/03]                 % Работа с подрисунками и подобным

%%% Счётчики %%%
%\usepackage[figure,table]{totalcount}               % Счётчик рисунков и таблиц. Взамен используется xassoccnt 
\usepackage{totcount}                               % Пакет создания счётчиков на основе последнего номера подсчитываемого элемента (может требовать дважды компилировать документ)
\usepackage{totpages}                               % Счётчик страниц, совместимый с hyperref (ссылается на номер последней страницы). Желательно ставить последним пакетом в преамбуле

\usepackage{xassoccnt} % для подсчета сумм приложений, рисунков, таблиц 


%%% Продвинутое управление групповыми ссылками (пока только формулами) %%%
\ifxetexorluatex
    \usepackage{cleveref}                           % cleveref корректно считывает язык из настроек polyglossia
\else
    \usepackage[russian]{cleveref}                  % cleveref имеет сложности со считыванием языка из babel. Такое решение русификации вывода выбрано вместо определения в documentclass из опасности что-то лишнее передать во все остальные пакеты, включая библиографию.
\fi
\creflabelformat{equation}{#2#1#3}                  % Формат по умолчанию ставил круглые скобки вокруг каждого номера ссылки, теперь просто номера ссылок без какого-либо дополнительного оформления



\ifnumequal{\value{draft}}{1}{% Черновик
    \usepackage[firstpage]{draftwatermark}
    \SetWatermarkText{DRAFT}
    \SetWatermarkFontSize{14pt}
    \SetWatermarkScale{15}
    \SetWatermarkAngle{45}
}{}

  
%%% Прикладные пакеты %%% 
%\usepackage{calc}               % Пакет для расчётов параметров, например длины

%%% Для добавления Стр. над номерами страниц в оглавлении
%%% http://tex.stackexchange.com/a/306950
\usepackage{afterpage}

\urlstyle{rm} % links in Times


%\makeatletter
%%расстояние после ToC title до 1ой строчки 
%%для достижения одинаковых отсупов переопределено формирование базового ToC
%\renewcommand{\aftertoctitle}{\par\nobreak\vskip1\curtextsize}
%\makeatother

%https://tex.stackexchange.com/questions/170912/contents-page-in-two-different-languages
%\makeatletter
\newcommand\russiantableofcontents{%
%	\if@twocolumn
%	\@restonecoltrue\onecolumn
%	\else
%	\@restonecolfalse
%	\fi
	%  \begin{otherlanguage}{russian}
	\chapter*{%
	\normalfont\MakeUppercase{Содержание} %слово <<Содержание>> в стилю chaperLight, по факту убираем \bfseries
%		    \contentsname
%		    \@mkboth{\MakeUppercaseСодержание}
%		            {\MakeUppercaseСодержание}%
	}%
%\hrule
\vspace*{-1\curtextsize} %убрать лишний отступ в таблице
	\@starttoc{tuc}%
	%  \end{otherlanguage}
%	\if@restonecol\twocolumn\fi
}
\newcommand{\addtocru}[2]{%
	\addcontentsline{tuc}{#1}{\protect\numberline{\csname the#1\endcsname}#2}%
%	\addcontentsline{tuc}{#1}{#2}%
}
\newcommand{\addtocruNoProtect}[2]{%
%	\addcontentsline{tuc}{#1}{\protect\numberline{\csname the#1\endcsname}#2}%
		\addcontentsline{tuc}{#1}{#2}%
}

%обеспечение красивого порядка вывода содержаний и названий разделов, подразделов и т.п.
\newcommand\englishtableofcontents{%
	%	\if@twocolumn
	%	\@restonecoltrue\onecolumn
	%	\else
	%	\@restonecolfalse
	%	\fi
	%  \begin{otherlanguage}{russian}
	\chapter*{%
		\normalfont\MakeUppercase{Content} %слово <<Содержание>> в стилю chaperLight, по факту убираем \bfseries
		%		    \contentsname
		%		    \@mkboth{\MakeUppercaseСодержание}
		%		            {\MakeUppercaseСодержание}%
	}%
	%\hrule
	\vspace*{-1\curtextsize} %убрать лишний отступ в таблице
	\@starttoc{tec}%
	%  \end{otherlanguage}
	%	\if@restonecol\twocolumn\fi
}
\newcommand{\addtocen}[2]{%
		\addcontentsline{tec}{#1}{\protect\numberline{\csname the#1\endcsname}#2}%
%	\addcontentsline{tec}{#1}{#2}%
}
\newcommand{\addtocenNoProtect}[2]{%for preface, introduction etc
%	\addcontentsline{tec}{#1}{\protect\numberline{\csname the#1\endcsname}#2}%
		\addcontentsline{tec}{#1}{#2}%
}


%стандартный вывод в toc можно использовать, если издание только на английском или русском.
%переопределена, чтобы обеспечить одинаковые отсупы от названия ToC (toc, tec, tuc) до первой строки
\renewcommand\tableofcontents{%
	%	\if@twocolumn
	%	\@restonecoltrue\onecolumn
	%	\else
	%	\@restonecolfalse
	%	\fi
	%  \begin{otherlanguage}{russian}
	\chapter*{%
		\MakeUppercase{Содержание} %слово <<Содержание>> 
		%		    \contentsname
		%		    \@mkboth{\MakeUppercaseСодержание}
		%		            {\MakeUppercaseСодержание}%
	}%
	%\hrule
%	\vspace*{-0.58\curtextsize} %убрать/добавить отступ от таблицы
	\@starttoc{toc}%
	%  \end{otherlanguage}
	%	\if@restonecol\twocolumn\fi
}
\newcommand{\addetoc}[2]{%
		\addcontentsline{toc}{#1}{\protect\numberline{\csname the#1\endcsname}#2}%
}
%\newcommand{\addtocru}[2]{%
%	\addcontentsline{tuc}{#1}{\protect\numberline{\csname the#1\endcsname}#2}%
%	%	\addcontentsline{tuc}{#1}{#2}%
%}

%\makeatother

%http://latex.org/forum/viewtopic.php?t=5438         
\usepackage{tabularx}

%%https://tex.stackexchange.com/a/362229
\usepackage{datatool-base}
\usepackage{mfirstuc} %первая буква прописная

\usepackage{layouts}

\newenvironment{abstr}{\smallA\itshape}{\normalfont\normalsize}


\usepackage[normalem]{ulem} % для перечеркнутых сроков команда \sout{text}
\newcommand{\soutthick}[1]{%
	\renewcommand{\ULthickness}{2.4pt}%
	\sout{#1}%
	\renewcommand{\ULthickness}{.4pt}% Resetting to ulem default
}

%для подчёркнутых команд
%https://tex.stackexchange.com/questions/270286/uline-not-work-for-command-arguments
\useunder{\uline}{\ulined}{}

\usepackage{environ} % for Uppercase in Keywords
%https://tex.stackexchange.com/questions/249628/uppercase-whole-newenvironment
% недостаток - новые окружения не подхватываются TexStudio

\usepackage{textcase} % for \MakeTextUppercase

%for svg pictures
%\usepackage{svg}


%%% Mailto %%% 
%%%https://tex.stackexchange.com/questions/128424/how-to-create-email-hyperlink-with-predefined-subject-in-latex
%% unfortunatelly Adobe does not handle Recipient name + email, e.g.
%% Vladimir Parkhomenko<parhomenko.v@gmail.com>


%mailto with subject (impossible with href)
%mailto anybody without email body
\makeatletter
\newcommand\mailtoab[3]{%                %\newcommand\tpj@compose@mailto[3]{%
	\edef\@tempa{mailto:#1?subject=#2 }%
	\edef\@tempb{\expandafter\html@spaces\@tempa\@empty}%
	\href{\@tempb}{#3}}
\catcode\%=11
\def\html@spaces#1 #2{#1%20\ifx#2\@empty\else\expandafter\html@spaces\fi#2}
	\catcode\%=14
	\makeatother
	
	
	%${email}{Subject}{email start body}{text in pdf}
	\makeatletter
	\newcommand\mailto[4]{%                %\newcommand\tpj@compose@mailto[3]{%
		\edef\@tempa{mailto:#1?subject=#2\&body=#3 }%
		\edef\@tempb{\expandafter\html@spaces\@tempa\@empty}%
		\href{\@tempb}{#4}}
	%% with %20 instead of spaces
	%\catcode\%=11
	%\def\html@spaces#1 #2{#1%20\ifx#2\@empty\else\expandafter\html@spaces\fi#2}
	%\catcode\%=14
	\makeatother
	
	%% MLABSED 2017 author
	%%${email}{Subject}{email start body}{text in pdf}
	\makeatletter
	\newcommand\mailtoMLABSEDauthor[3]{%                
		\edef\@tempa{mailto:#1?subject=MLABSED 2017\&body=#2 }%
		\edef\@tempb{\expandafter\html@spaces\@tempa\@empty}%
		\href{\@tempb}{#3}}
	%% with %20 instead of spaces
	%\catcode\%=11
	%\def\html@spaces#1 #2{#1%20\ifx#2\@empty\else\expandafter\html@spaces\fi#2}
	%\catcode\%=14
	\makeatother
	
	
	%%Vladimir Parkhomenko
	\makeatletter
	\newcommand\mailtopa[1]{%                %\newcommand\tpj@compose@mailto[3]{%
		\edef\@tempa{mailto:parhomenko.v@gmail.com?subject=#1\&body=Dear Vladimir, }%
		\edef\@tempb{\expandafter\html@spaces\@tempa\@empty}%
		\href{\@tempb}{Vladimir.Parkhomenko@spbstu.ru}}
	\catcode\%=11
	\def\html@spaces#1 #2{#1%20\ifx#2\@empty\else\expandafter\html@spaces\fi#2}
		\catcode\%=14
		\makeatother
		
		%%Alexey Buzmakov
		\makeatletter
		\newcommand\mailtobu[1]{%                %\newcommand\tpj@compose@mailto[3]{%
			\edef\@tempa{mailto:abuzmakov@gmail.com?subject=#1\&body=Dear Alexey, }%
			\edef\@tempb{\expandafter\html@spaces\@tempa\@empty}%
			\href{\@tempb}{abuzmakov@gmail.com}}
		\catcode\%=11
		\def\html@spaces#1 #2{#1%20\ifx#2\@empty\else\expandafter\html@spaces\fi#2}
			\catcode\%=14
			\makeatother
			
			%%Xenia Naidenova
			\makeatletter
			\newcommand\mailtona[1]{%                %\newcommand\tpj@compose@mailto[3]{%
				\edef\@tempa{mailto:ksennaidd@gmail.com?subject=#1\&body=Dear Xenia, }%
				\edef\@tempb{\expandafter\html@spaces\@tempa\@empty}%
				\href{\@tempb}{ksennaidd@gmail.com}}
			\catcode\%=11
			\def\html@spaces#1 #2{#1%20\ifx#2\@empty\else\expandafter\html@spaces\fi#2}
				\catcode\%=14
				\makeatother
				
				
				%%Konstantin Shvetsov
				\makeatletter
				\newcommand\mailtosh[1]{%                %\newcommand\tpj@compose@mailto[3]{%
					\edef\@tempa{mailto:shvetsov@inbox.ru?subject=#1\&body=Dear Konstantin, }%
					\edef\@tempb{\expandafter\html@spaces\@tempa\@empty}%
					\href{\@tempb}{Konstantin.Shvetsov@spbstu.ru}}
				\catcode\%=11
				\def\html@spaces#1 #2{#1%20\ifx#2\@empty\else\expandafter\html@spaces\fi#2}
					\catcode\%=14
					\makeatother


\usepackage{tabu, tabulary}  %таблицы с автоматически подбирающейся шириной столбцов
\usepackage{fr-longtable}    %ради \endlasthead

% Листинги с исходным кодом программ
\usepackage{fancyvrb}
\usepackage{listings}
\lccode`\~=0\relax %Без этого хака из-за особенностей пакета listings перестают работать конструкции с \MakeLowercase и т. п. в (xe|lua)latex

% Русская традиция начертания греческих букв
\usepackage{upgreek} % прямые греческие ради русской традиции

%https://tex.stackexchange.com/a/62351/44348
% Микротипографика
\ifnumequal{\value{draft}}{0}{% Только если у нас режим чистовика
    \usepackage[final,letterspace=150]{microtype}[2016/05/14] % улучшает представление букв и слов в строках, может помочь при наличии отдельно висящих слов
%    \lsstyle for letterspace style of letters
}{}

% Отметка о версии черновика на каждой странице
% Чтобы работало надо в своей локальной копии по инструкции
% https://www.ctan.org/pkg/gitinfo2 создать небходимые файлы в папке
% ./git/hooks
% If you’re familiar with tweaking git, you can probably work it out for
% yourself. If not, I suggest you follow these steps:
% 1. First, you need a git repository and working tree. For this example,
% let’s suppose that the root of the working tree is in ~/compsci
% 2. Copy the file post-xxx-sample.txt (which is in the same folder of
% your TEX distribution as this pdf) into the git hooks directory in your
% working copy. In our example case, you should end up with a file called
% ~/compsci/.git/hooks/post-checkout
% 3. If you’re using a unix-like system, don’t forget to make the file executable.
% Just how you do this is outside the scope of this manual, but one
% possible way is with commands such as this:
% chmod g+x post-checkout.
% 4. Test your setup with “git checkout master” (or another suitable branch
% name). This should generate copies of gitHeadInfo.gin in the directories
% you intended.
% 5. Now make two more copies of this file in the same directory (hooks),
% calling them post-commit and post-merge, and you’re done. As before,
% users of unix-like systems should ensure these files are marked as
% executable.
\ifnumequal{\value{draft}}{1}{% Черновик
   \IfFileExists{.git/gitHeadInfo.gin}{                                        
      \usepackage[mark,pcount]{gitinfo2}
      \renewcommand{\gitMark}{rev.\gitAbbrevHash\quad\gitCommitterEmail\quad\gitAuthorIsoDate}
      \renewcommand{\gitMarkFormat}{\color{Gray}\small\bfseries}
   }{}
}{}         
%%%%%%%%%%%%%%%%%%%%%%%%%%%%%%%%%%%%%%%%%%%%%%%%%%%%%%
%%%% Файл упрощённых настроек шаблона диссертации %%%%
%%%%%%%%%%%%%%%%%%%%%%%%%%%%%%%%%%%%%%%%%%%%%%%%%%%%%%

%%% Инициализирование переменных, не трогать!  %%%
\newcounter{intvl}
\newcounter{otstup}
\newcounter{contnumeq}
\newcounter{contnumfig}
\newcounter{contnumtab}
\newcounter{pgnum}
\newcounter{chapstyle}
\newcounter{headingdelim}
\newcounter{headingalign}
\newcounter{headingsize}
\newcounter{tabcap}
\newcounter{tablaba}
\newcounter{tabtita}
\newcounter{docType} 		% тип документа
\newcounter{tskPrint} 		% печать Задания на ВКР двух(одно)сторонняя
\newcounter{tskPages}       % для учёта количества страниц в Задании
\newcounter{tskPageFirst}   % для учёта количества страниц в Задании
\newcounter{tskPageLast}    % для учёта количества страниц в Задании 
\newcounter{sumPrint} 		% печать Реферата на ВКР двух(одно)сторонняя
\newcounter{sumPages}       % для учёта количества страниц в Реферате
\newcounter{sumPageFirst}   % для учёта количества страниц в Реферате
\newcounter{sumPageLast}    % для учёта количества страниц в Реферате 
\newcommand{\Single}{0.78}  % пропорция для одинароного отступа в \Spacing
%%%%%%%%%%%%%%%%%%%%%%%%%%%%%%%%%%%%%%%%%%%%%%%%%%

%%% Область упрощённого управления оформлением %%%

% Управление перенесено в главые файлы компиляции ВКР, Задания, Реферата
\setcounter{tskPrint}{0} %по умолчанию односторонняя печать              
%\setcounter{sumPrint}{0} %по умолчанию односторонняя печать 

%% Интервал между заголовками и между заголовком и текстом
% Заголовки отделяют от текста сверху и снизу тремя интервалами (ГОСТ Р 7.0.11-2011, 5.3.5)
\setcounter{intvl}{3}               % Коэффициент кратности к размеру шрифта

% Заголовки отделяют от текста сверху и снизу тремя интервалами 
\newcommand{\intvlS}{1.5}               % Коэффициент кратности к размеру шрифта SPbPU-student-templates

\newcommand{\intervalS}{\vspace{\intvlS\curtextsize}}

% печать списка источников в Задании
\newcommand{\printbibliographyTask}{\vspace{-0.28\curtextsize}
	\printbibliography[env=tsk] % печать списка литературы в исходных данных
	\vspace{-0.28\curtextsize}}


%% Отступы у заголовков в тексте
\setcounter{otstup}{0}              % 0 --- без отступа; 1 --- абзацный отступ

%% Нумерация формул, таблиц и рисунков
\setcounter{contnumeq}{0}           % Нумерация формул: 0 --- пораздельно (во введении подряд, без номера раздела); 1 --- сквозная нумерация по всей диссертации
\setcounter{contnumfig}{0}          % Нумерация рисунков: 0 --- пораздельно (во введении подряд, без номера раздела); 1 --- сквозная нумерация по всей диссертации
\setcounter{contnumtab}{0}          % Нумерация таблиц: 0 --- пораздельно (во введении подряд, без номера раздела); 1 --- сквозная нумерация по всей диссертации


%% Нумерация подстраничных сносок (ссылок)
%сквозная
\counterwithout{footnote}{chapter} %сквозная нумерация подразделов (во всех главах)


%% Нумерация подразделов
%убрать номер главы в секции
%\counterwithout{section}{chapter} %сквозная нумерация подразделов (во всех главах)
%\renewcommand\thesection{\arabic{section}} %в каждой главе нумерация заново

%\renewcommand\thesection{\arabic{section}}
%\renewcommand\thefigure{\fbox{\arabic{figure}}}
%\renewcommand\thetable{\arabic{table}}
%\renewcommand\theequation{\arabic{equation}}



%\counterwithout{section}{chapter}
%\counterwithout{figure}{chapter}
%\counterwithout{table}{chapter}
%\counterwithout{equation}{chapter}

%\counterwithin{section}{chapter}
%\counterwithin{figure}{chapter}
%\counterwithin{table}{chapter}

%% Оглавление

\setcounter{pgnum}{1}               %NB УДАЛЕНО ФИЗИЧЕСКИ 0 --- номера страниц никак не обозначены; 1 --- Стр. над номерами страниц (дважды компилировать после изменения)  
\settocdepth{subsection} %             до какого уровня подразделов выносить в оглавление
\setsecnumdepth{subsubsection}         % до какого уровня нумеровать подразделы


%% Текст и форматирование заголовков
\setcounter{chapstyle}{1}           % 0 --- разделы только под номером; 1 --- разделы с названием "Глава" перед номером
\setcounter{headingdelim}{2}        % 0 --- номер отделен пропуском в 1em или \quad; 1 --- номера разделов и ений отделены точкой с пробелом, подразделы пропуском без точки; 2 --- номера разделов, подразделов и приложений отделены точкой с пробелом.

%% Выравнивание заголовков в тексте
\setcounter{headingalign}{0}        % 0 --- по центру; 1 --- по левому краю

%% Размеры заголовков в тексте
\setcounter{headingsize}{0}         % 0 --- SPbPU style, все всегда 14 пт; 1 --- пропорционально изменяющийся размер в зависимости от базового шрифта;

%% Подпись таблиц
\setcounter{tabcap}{1}              % 0 --- по ГОСТ, номер таблицы и название разделены тире, выровнены по левому краю, при необходимости на нескольких строках; 1 --- подпись таблицы не по ГОСТ, на двух и более строках, дальнейшие настройки: 
%Выравнивание первой строки, с подписью и номером
\setcounter{tablaba}{2}             % 0 --- по левому краю; 1 --- по центру; 2 --- по правому краю
%Выравнивание строк с самим названием таблицы
\setcounter{tabtita}{1}             % 0 --- по левому краю; 1 --- по центру; 2 --- по правому краю
%Разделитель записи «Таблица #» и названия таблицы
\newcommand{\tablabelsep}{space}   % space = пробел, period =  (определены в подключенных пакетах)

%% Подпись рисунков
%Разделитель записи «Рисунок #» и названия рисунка
\newcommand{\figlabelsep}{period}   % emdash = тире, определён в common/styles; period = точка определён в подключенных пакетах; space
%\newcommand{\figlabelsep}{emdash}   % emdash = тире, определён в common/styles; period = точка определён в подключенных пакетах


%%% Цвета гиперссылок %%%
% Latex color definitions: http://latexcolor.com/

%\definecolor{linkcolor}{rgb}{0.9,0,0}
%\definecolor{citecolor}{rgb}{0,0.6,0}
%\definecolor{urlcolor}{rgb}{0,0,1}


%\definecolor{linkbordercolor}{rgb}{0,0,1}

\definecolor{linkcolor}{HTML}{FF0000} %very light red from the SPbPU brandbook (2nd level)
\definecolor{citecolor}{HTML}{6CF479} %very light green from the SPbPU brandbook (2nd level)
\definecolor{urlcolor}{HTML}{4481BA} %very light blue from the SPbPU brandbook (2nd level)

%\definecolor{linkcolor}{rgb}{0,0,0} %black
%\definecolor{citecolor}{rgb}{0,0,0} %black
%\definecolor{urlcolor}{rgb}{0,0,0} %black               
\input{template_settings/Dissertation/preamblenames}       
%%% Кодировки и шрифты %%%
\ifxetexorluatex
    \setmainlanguage[babelshorthands=true]{russian}  % Язык по-умолчанию русский с поддержкой приятных команд пакета babel
    \setotherlanguage{english}                       % Дополнительный язык = английский (в американской вариации по-умолчанию)
    \setmonofont{Courier New}
    \newfontfamily\cyrillicfonttt{Courier New}
    \ifXeTeX
        \defaultfontfeatures{Ligatures=TeX,Mapping=tex-text}
    \else
        \defaultfontfeatures{Ligatures=TeX}
    \fi
    \setmainfont{Times New Roman}
    \newfontfamily\cyrillicfont{Times New Roman}
    \setsansfont{Arial}
    \newfontfamily\cyrillicfontsf{Arial}
\else
    \IfFileExists{pscyr.sty}{\renewcommand{\rmdefault}{ftm}}{}
\fi

%%% Подписи %%%
\captionsetup{%
singlelinecheck=off,                % Многострочные подписи, например у таблиц
skip=5pt,                           % Вертикальная отбивка между подписью и содержимым рисунка или таблицы определяется ключом
justification=centering            % Центрирование подписей, заданных командой \caption
}

%\setlength{\abovecaptionskip}{0pt} %альтернатива для skip, но не распространяется на longtable!
%\setlength{\belowcaptionskip}{0pt}
%\captionwidth{\linewidth}
%\normalcaptionwidth

% для изменения отступов от floats (e.g. table,figure) & minipage
\newlength{\mfloatsep}
\setlength{\mfloatsep}{4mm plus 0.7mm minus 0.7mm} %3mm для A5

% фиксируем расстояния с помощью пакета layouts
\setlength{\textfloatsep}{\mfloatsep} % расстояние от текста до float, если float прижат к верхнему или нижнему краю
\setlength{\floatsep}{\mfloatsep} % расстояние от float до float (если оба сверху/снизу)
\setlength{\intextsep}{\mfloatsep} % расстояние от текста до float, если float снизу и сверху ограничен текстом 
%
%% фактически из-за бокса, внутрь которого помещается \captionof{figure} происходит увеличаение на 1мм отступа в соответствующем элементе!
%
%% по требованиям СПбПУ как раз необходим отступ 4мм от рисунка


%Возможно более гибко задавать отступы, например:
%\setlength{\floatsep}{12pt plus 2pt minus 2pt}
%\setlength{\textfloatsep}{20pt plus 2pt minus 4pt}
%\setlength{\intextsep}{\floatsep}

%https://tex.stackexchange.com/questions/60477/remove-space-after-figure-and-before-text
%https://tex.stackexchange.com/questions/26521/how-to-change-the-spacing-between-figures-tables-and-text




%%% Парный к \smallA шрифт 13bp в подписи%%%
%TO-DO как напрямую связать со \smallA
%\DeclareCaptionFont{font13bp}{\smallA\selectfont} %к сожалению, приводит к отсупу после номера рисунка
\DeclareCaptionFont{font13bp}{\fontsize{13bp}{16.77bp}\selectfont} %аналог задания вручную
\DeclareCaptionFont{font12bp}{\small\selectfont} %аналог задания вручную



%%% Рисунки %%%
\DeclareCaptionLabelSeparator*{emdash}{~--- }             % (ГОСТ 2.105, 4.3.1)

\DeclareCaptionLabelFormat{figwithoutspace}{#1#2}
%\captionsetup[figure]{labelformat=figwithoutspace,labelsep=none,name=Fig.}

\captionsetup[figure]{labelformat=figwithoutspace,labelsep=\figlabelsep,position=bottom,labelfont={font12bp},textfont={font12bp}}

%\setlength{\belowcaptionskip}{0pt} %расстояние между 
%\caption* -- подрисуночной подписи и
%\caption  -- подписи к рисунку с номером
%необходимо менять вслед за добавлением \vskip в \captionsetup

%\setfloatadjustment{figure}{%
%	\setlength{\belowcaptionskip}{-3pt}   % чтобы обивка после рисунков была 3mm, так как caption добавляет около 1мм к 3мм. 
%}




%%% Таблицы %%%
\ifnumequal{\value{tabcap}}{0}{%
    \newcommand{\tabcapalign}{\raggedright}  % по левому краю страницы или аналога parbox

    \DeclareCaptionFormat{tablecaption}{\tabcapalign #1#2#3}
    \captionsetup[table]{labelsep=emdash}        % тире как разделитель идентификатора с номером от наименования
}{%
    \ifnumequal{\value{tablaba}}{0}{%
        \newcommand{\tabcapalign}{\raggedright}  % по левому краю страницы или аналога parbox
    }{}

    \ifnumequal{\value{tablaba}}{1}{%
        \newcommand{\tabcapalign}{\centering}    % по центру страницы или аналога parbox
    }{}

    \ifnumequal{\value{tablaba}}{2}{%
        \newcommand{\tabcapalign}{\raggedleft}   % по правому краю страницы или аналога parbox
    }{}

    \ifnumequal{\value{tabtita}}{0}{%
        \newcommand{\tabtitalign}{\raggedright}  % по левому краю страницы или аналога parbox
    }{}

    \ifnumequal{\value{tabtita}}{1}{%
        \newcommand{\tabtitalign}{\centering}    % по центру страницы или аналога parbox
    }{}

    \ifnumequal{\value{tabtita}}{2}{%
        \newcommand{\tabtitalign}{\raggedleft}   % по правому краю страницы или аналога parbox
    }{}

    \DeclareCaptionFormat{tablecaption}{\tabcapalign #1#2\par %\hline  % Идентификатор таблицы на отдельной строке
        \tabtitalign{#3}}                                       % Наименование таблицы строкой ниже
    \captionsetup[table]{labelsep=\tablabelsep}                 % разделитель идентификатора с номером от наименования
}
\DeclareCaptionFormat{tablenocaption}{\tabcapalign #1\strut}    % Наименование таблицы отсутствует

\newlength{\ltskip}
\setlength{\ltskip}{2pt}
\captionsetup[table]{format=tablecaption,singlelinecheck=off,position=top,labelfont={font12bp},textfont={font12bp}}  % многострочные наименования и прочее
\DeclareCaptionLabelFormat{continued}{\\[-14pt]Продолжение табл.~\!#2}



%%% Подписи подрисунков %%%
\renewcommand{\thesubfigure}{\alph{subfigure}}           % Буквенные номера подрисунков
\captionsetup[subfigure]{font={font12bp},               % Шрифт подписи названий подрисунков (отличается от основного)
	labelfont={font12bp},textfont={font12bp},
    labelformat=brace,                                    % Формат обозначения подрисунка
    singlelinecheck=off,
%    position=left,
    justification=raggedright 							 %выравнивание влево
%    justification=centering,                              % Выключка подписей (форматирование), один из вариантов            
}

%%% Подписи подрисунков SPbPU%%%
% реализован подход по первой ссылке, он позволяет масштабировать количество подрисунков
%https://tex.stackexchange.com/a/273169/44348
%https://tex.stackexchange.com/a/225914/44348
\usepackage[export]{adjustbox}



%%% Настройки гиперссылок %%%
\ifLuaTeX
    \hypersetup{
        unicode,                % Unicode encoded PDF strings
    }
\fi


\newcommand{\thesisTitle}{Название выпускной квалификационной работы}


\hypersetup{
    linktocpage=true,           % ссылки с номера страницы в оглавлении, списке таблиц и списке рисунков
%    linktoc=all,                % both the section and page part are links
%    pdfpagelabels=false,        % set PDF page labels (true|false)
    plainpages=false,           % Forces page anchors to be named by the Arabic form  of the page number, rather than the formatted form
    %colorlinks,                 % ссылки отображаются раскрашенным текстом, а не раскрашенным прямоугольником, вокруг текста
    citebordercolor={0.287 0.89 0.349}, %(RGB colour) with default {0 1 0}: The colour of the box around citations
    filebordercolor={0 .5 .5}, % (RGB colour) with default {0 .5 .5}: The colour of the box around links to files
    linkbordercolor={0.93 0 0}, % (RGB colour) with default {1 0 0}: The colour of the box around normal links
    menubordercolor={1 0 0}, % (RGB colour) with default {1 0 0}: The colour of the box around Acrobat menu links
    urlbordercolor={0.313 0.776 0.878}, % (RGB colour) with default {0 1 1}: The colour of the box around links to URLs
    pdfborder={0 0 1}, % The style of box around links; defaults to a box with lines of 1pt thickness, but the colorlinks option resets it to produce no border.
%    linkcolor={linkcolor},
%    citecolor={citecolor},      % цвет ссылок-цитат
%    urlcolor={urlcolor},        % цвет гиперссылок
    %hidelinks,                  % Hide links (removing color and border)
%    pdftitle={\thesisTitle},    % Заголовок pdf-файла
%    pdfauthor={\AuthorFull},  % Автор
%    pdfsubject={\thesisSpecialtyNumber\ \thesisSpecialtyTitle},      % Тема
%    pdfcreator={Создатель},     % Создатель, Приложение
%    pdfproducer={Производитель},% Производитель, Производитель PDF
%    pdfkeywords={\keywords},    % Ключевые слова
    pdflang={eng}, %eng %ru
    % % bookmarks settings
    bookmarks=true,
    bookmarksnumbered=true, % put section numbers
    bookmarksopen=true, %open when the pdf is opened
    bookmarksopenlevel=0, %chapter's level is enough to see
    bookmarksdepth=0 %set the depth of the levels in the pdf navigation bar
}

% %improve the bookmarksnumbered representation:
\makeatletter
\renewcommand{\Hy@numberline}[1]{#1. } %add the dot after a number
\makeatother


\ifnumequal{\value{draft}}{1}{% Черновик
    \hypersetup{
        draft,
    }
}{}

%%% Шаблон %%%
\DeclareRobustCommand{\todo}{\textcolor{red}}       % решаем проблему превращения названия цвета в результате \MakeUppercase, http://tex.stackexchange.com/a/187930/79756 , \DeclareRobustCommand protects \todo from expanding inside \MakeUppercase
\AtBeginDocument{%
    \setlength{\parindent}{2.5em}                   % Абзацный отступ. Должен быть одинаковым по всему тексту и равен пяти знакам (ГОСТ Р 7.0.11-2011, 5.3.7).
}

%%% Списки %%%
% Используем короткое тире (endash) для ненумерованных списков (ГОСТ 2.105-95, пункт 4.1.7, требует дефиса, но так лучше смотрится)
\renewcommand{\labelitemi}{\normalfont\bfseries{--}}

%% Перечисление строчными буквами латинского алфавита (ГОСТ 2.105-95, 4.1.7)
\renewcommand{\theenumi}{\Alph{enumi}} % первый уровень иерархии %латинскийалфавит заглавные
\renewcommand{\labelenumi}{\theenumi.} 
%\renewcommand{\theenumii}{\alph{enumii}} % второй уровень иерархии %латинскийалфавит
%\renewcommand{\labelenumii}{\theenumii)} 
%
%
%% Перечисление строчными буквами русского алфавита (ГОСТ 2.105-95, 4.1.7)
\makeatletter
\AddEnumerateCounter{\asbuk}{\russian@alph}{щ}      % Управляем списками/перечислениями через пакет enumitem, а он 'не знает' про asbuk, потому 'учим' его
\makeatother
%%\renewcommand{\theenumi}{\asbuk{enumi}} %первый уровень нумерации
%%\renewcommand{\labelenumi}{\theenumi)} %первый уровень нумерации 
%%\renewcommand{\theenumii}{\asbuk{enumii}} %второй уровень нумерации
%%\renewcommand{\labelenumii}{\theenumii)} %второй уровень нумерации 
\renewcommand{\theenumii}{\arabic{enumii}} %второй уровень нумерации %арабские цифры
\renewcommand{\labelenumii}{\theenumii.} %второй уровень нумерации 
%\renewcommand{\theenumi}{\arabic{enumi}} %первый уровень нумерации %арабские цифры
%\renewcommand{\labelenumi}{\theenumi)} %первый уровень нумерации 
%
%\renewcommand{\theenumiii}{\asbuk{enumiii}} %третий уровень нумерации %русский алфавит
\renewcommand{\theenumiii}{\alph{enumiii}} %третий уровень нумерации %английский алфавит
\renewcommand{\labelenumiii}{\theenumiii)} %третий уровень нумерации 
%\renewcommand{\theenumiii}{\arabic{enumiii}} %третий уровень нумерации %арабские цифры
%\renewcommand{\labelenumiii}{\theenumiii)} %третий уровень нумерации 



\setlist{nosep,%                                    % Единый стиль для всех списков (пакет enumitem), без дополнительных интервалов.
    labelindent=\parindent,leftmargin=*%            % Каждый пункт, подпункт и перечисление записывают с абзацного отступа (ГОСТ 2.105-95, 4.1.8)
}



% asm packages required! In particular amsthm
%http://tex.stackexchange.com/questions/37472/spacing-before-and-after-with-newtheoremstyle

%theoremstyle{}
%plain : italic text, extra space above and below;
%definition : upright text, extra space above and below;
%remark : upright text, no extra space above or below.

\newtheoremstyle{myplain} %
{0} %space above
{0} %space below
{\itshape}
{\parindent}
{\bfseries}
{.}
{.5em}
{}

\newtheoremstyle{mydefinition} %
{0} %space above
{0} %space below
{}
{\parindent}
{\bfseries}
{.}
{.5em}
{}

\theoremstyle{myplain} %improved plain style
\newtheorem{m-theorem}{Теорема}[chapter] % reset theorem numbering for each chapter
\newtheorem{m-corollary}{Следствие}[chapter] % definition numbers are 
\newtheorem{m-proposition}{Утверждение}[chapter] % definition numbers are dependent on theorem numbers
\newtheorem{m-lemma}{Лемма}[chapter]
\newtheorem{m-axiom}{Аксиома}[chapter]

\theoremstyle{mydefinition} %improved definition style
\newtheorem{m-example}{Пример}[chapter] % same for example numbers
\newtheorem{m-definition}{Определение}[chapter]  % definition numbers
\newtheorem{m-condition}{Условие}[chapter]
\newtheorem{m-problem}{Проблема}[chapter]
\newtheorem{m-exercise}{Упражнение}[chapter]
\newtheorem{m-question}{Вопрос}[chapter]
\newtheorem{m-hypothesis}{Гипотеза}[chapter]
\newtheorem{m-task}{Задание}[chapter]



%%control skip of thm, plain style - ANOTHER VARIANT
%%http://tex.stackexchange.com/questions/85400/how-to-change-space-around-theorem-environments
%\makeatletter
%\def\thm@space@setup{%
%	\thm@preskip=0cm %
%	%	\thm@preskip=0cm plus 0.2cm minus 0.2cm
%	\thm@postskip=0cm % or whatever, if you don't want them to be equal
%	%	\thm@postskip=\thm@preskip % or whatever, if you don't want them to be equal
%}
%\makeatother
    
%%% Изображения %%%
\graphicspath{{images/}{Dissertation/images/}}         % Пути к изображениям

%%% Макет страницы %%%
% Выставляем значения полей (ГОСТ 7.0.11-2011, 5.3.7)
\makeatletter
\geometry{a4paper,top=2cm,bottom=2cm,left=3cm,right=1cm,
	headsep=0.5cm, %отступ от колонтитула от живописного поля
	head=1cm, %верхняя граница колонтитула
	headheight=1cm,
	nofoot,
%includefoot,
	nomarginpar
%	,showframe
} 
%\setlength{\topskip}{0pt}   %размер дополнительного верхнего поля
\makeatother

%%% Интервалы %%%
%% По ГОСТ Р 7.0.11-2011, пункту 5.3.6 требуется полуторный интервал
%% Реализация средствами класса (на основе setspace) ближе к типографской классике.
%% И правит сразу и в таблицах (если со звёздочкой) 
%\DoubleSpacing*     % Двойной интервал
\OnehalfSpacing*    % Полуторный интервал % * to force it in the floats
%\setSpacing{1.42}   % Полуторный интервал, подобный Ворду (возможно, стоит включать вместе с предыдущей строкой)

%https://tex.stackexchange.com/questions/65849/confusion-onehalfspacing-vs-spacing-vs-word-vs-the-world/276516#276516
%https://tex.stackexchange.com/questions/13742/what-does-double-spacing-mean
%https://tex.stackexchange.com/questions/30073/why-is-the-linespread-factor-as-it-is/30114#30114

%A possible definition of \onehalfspacing and \doublespacing is that the ratio between font size and \baselineskip should be 1.5 resp. 2.....
%\baselineskip (vertical skip between the base lines of two successive lines of type) of XXpt. 


%%% Выравнивание и переносы %%%
%% http://tex.stackexchange.com/questions/241343/what-is-the-meaning-of-fussy-sloppy-emergencystretch-tolerance-hbadness
%% http://www.latex-community.org/forum/viewtopic.php?p=70342#p70342
\tolerance 1414
\hbadness 1414
\emergencystretch 1.5em % В случае проблем регулировать в первую очередь
\hfuzz 0.3pt
\vfuzz \hfuzz
%\raggedbottom
%\sloppy                 % Избавляемся от переполнений
\clubpenalty=10000      % Запрещаем разрыв страницы после первой строки абзаца
\widowpenalty=10000     % Запрещаем разрыв страницы после последней строки абзаца

%%% Блок управления параметрами для выравнивания заголовков в тексте %%%
\newlength{\otstuplen}
\setlength{\otstuplen}{\theotstup\parindent}
\ifnumequal{\value{headingalign}}{0}{% выравнивание заголовков в тексте
    \newcommand{\hdngalign}{\centering}                % по центру
    \newcommand{\hdngaligni}{}% по центру
    \setlength{\otstuplen}{0pt}
}{%
    \newcommand{\hdngalign}{}                 % по левому краю
    \newcommand{\hdngaligni}{\hspace{\otstuplen}}      % по левому краю
} % В обоих случаях вроде бы без переноса, как и надо (ГОСТ Р 7.0.11-2011, 5.3.5)







%%% Оглавление %%%

\renewcommand{\cftchapterdotsep}{\cftdotsep}                % отбивка точками до номера страницы начала главы/раздела



%% снятие жирности %%
%\cftKleader defines the leader between the title and the page number; it can be
%changed by \renewcommand. The spacing between any dots in the leader is controlled
%by \cftKdotsep
\makeatletter
\renewcommand{\cftchapterpagefont}{\normalfont}        % нежирные номера страниц у глав в оглавлении
\renewcommand{\cftchapterleader}{\cftdotfill{\cftchapterdotsep}}% нежирные точки до номеров страниц у глав в оглавлении
\renewcommand{\cftchapterfont}{}                       % нежирные названия глав в оглавлении
\renewcommand{\cftchapterpagefont}{}                       % нежирные названия номеров глав в оглавлении
\makeatother


%% Форматирование SPbPU %%
% Варианты форматирования
%https://tex.stackexchange.com/questions/394227/memoir-toc-indent-the-second-line-by-numberspace-width-in-the-previous-line-or



%% работа с расстояниями между точками, переносами слов
\makeatletter
\renewcommand{\cftdotsep}{0.1}
%\renewcommand{\@dotset}{0.1} %old macro DOES NOT WORK
\setpnumwidth{2.84538em} %2.84538em = 1cm  
%\renewcommand{\@pnumwidth}{0em} %old macro
%%\setrmarg > \setpnumwidth !!!
\setrmarg{2.84539em}
%set large number to restrict hyphenation
%plus1fil makes the distance between the words smaller!
%it helps to make the equal indent
\makeatother

%\usepackage{tocloft}    % tocloft for table of contents style

%% отступы %%
\makeatletter
\renewcommand{\cftchapterbreak}{}        % set a page break before rather than after the entry
%\renewcommand{\cftparskip}{10em} % эта настройка не работает, вместо неё изменен \parskip непостредственно перед \tableofcontents
\setlength{\cftbeforechapterskip}{0pt plus 0pt} %delete skip after chapter block (last section) %%%<-SPbPU pure
\setlength{\cftbeforepartskip}{0pt plus 0pt} %delete skip after chapter block (last section) %%%<-SPbPU pure
\makeatother



%% Продолжение редактирования оглавления настройками CandDoctDiss %%		


\ifnumgreater{\value{headingdelim}}{0}{%
	%<- SPbPU точка после номера страницы
	\renewcommand\cftchapteraftersnum{.\space}       % добавляет точку с пробелом после номера раздела в оглавлении
	%\renewcommand\cftchapteraftersnumb{\enspace\textperiodcentered\enspace} %\enspace - настоящий пробел, \space не работает
	%\renewcommand\chapternumberlinebox[2]{#2}
}{}
\ifnumgreater{\value{headingdelim}}{1}{%%%<-SPbPU 
	%	   	\renewcommand\cftsectionpresnum{..}       % добавляет smth перед number %выравнивает в box
	% точка после номера страницы
	\renewcommand\cftsectionaftersnum{.\space}       % добавляет точку с пробелом после номера подраздела в оглавлении
	% last is \hfil !
	%   	\renewcommand\cftsectionaftersnumb{...}       % добавляет точки перед названием %можно удалить пробел
	\renewcommand\cftsubsectionaftersnum{.\space}    % добавляет точку с пробелом после номера подподраздела в оглавлении
	\renewcommand\cftsubsubsectionaftersnum{.\space} % добавляет точку с пробелом после номера подподподраздела в оглавлении
	\AtBeginDocument{% без этого polyglossia сама всё переопределяет
		\setsecnumformat{\csname the#1\endcsname.\space}
		%\setsecnumformat{\csname the#1\endcsname:\quad}
	}
}{%
	\AtBeginDocument{% без этого polyglossia сама всё переопределяет
		\setsecnumformat{\csname the#1\endcsname\quad} %
	}
}

\renewcommand*{\cftappendixname}{\appendixname\space} % Слово Приложение в оглавлении


%%% Различные варианты форматирования SPbPU %%%

%% форматирование отсупов до номеров страниц стр. 151 мемуара !!!
%\renewcommand*{\cftsectionnumwidth}{} %выставление абсолютного значения
%\addtolength{\cftsectionnumwidth}{5em} %не работает

%убираем фиксированные размеры of the box %%%<-SPbPU pure
\AtBeginDocument{%
\renewcommand\numberlinebox[2]{#2} % for sections %%%<-SPbPU pure
\renewcommand\chapternumberlinebox[2]{#2} % for chapters 
%\newcommand\chapternumberlinebox[2]{%
%	\hb@xt@#1{#2\hfil}}
%
%\newcommand\chapternumberlinebox[2]{%
%	#1{\hfil#2}}

%\numberlinebox{hlengthi}{hcodei} %выставление абсолютного значения
%\chapternumberlinebox{hlengthi}{hcodei} %выставление абсолютного значения
}

%убираем растояния до \cftsectionpresnum в размере одного абзацного отступа %%%<-SPbPU pure
%\cftsetindents{hkindi}{hindenti}{hnumwidthi}


%https://tex.stackexchange.com/questions/306851/multiline-unnumbered-chapter-in-table-of-contents
%https://tex.stackexchange.com/questions/40022/customized-table-of-contents-same-indentation-for-every-line-of-multi-line-titl
%\parindent % standart padding
% это здорово экономит место, но нужно тогда синхронизировать стиль обычных отступов в перечислениях
% недостаток - не видно выравнивания по первому слову в названии предыдущего раздела
\AtBeginDocument{
	\cftsetindents{chapter}{0pt}{% первая строка
		-0.05em} % последующие строки от первой
	\cftsetindents{section}{%
		0pt
%3.5ex plus 1ex minus .2ex
	}{%
		\parindent
%2.3ex plus .2ex
}
	\cftsetindents{subsection}{%
	0pt}{% 
		\parindent}
	\cftsetindents{subsubsection}{%
		0pt}{% 
		\parindent}
}



%%% Колонтитулы %%%
% Порядковый номер страницы печатают на середине верхнего поля страницы (ГОСТ Р 7.0.11-2011, 5.3.8)
%сделаем справа внизу
%\makeatletter
\makeevenhead{plain}{}{}{\thepage}
\makeoddhead{plain}{}{}{\thepage}
\makeevenfoot{plain}{}{}{}
\makeoddfoot{plain}{}{}{}
\pagestyle{plain}

%%% добавить Стр. над номерами страниц в оглавлении
%%% http://tex.stackexchange.com/a/306950
\newif\ifendTOC
%
\newcommand*{\tocheader}{
%\ifnumequal{\value{pgnum}}{1}{%
%    \ifendTOC\else\hbox to \linewidth%
%      {\noindent{}~\hfill{Pages}}\par%
%      \ifnumless{\value{page}}{3}{}{%
%        \vspace{0.5\onelineskip}
%      }
%      \afterpage{\tocheader}
%    \fi%
%}{}%
}%


%epigraph with DOI
%\usepackage{quotchap}




%%% SPbPU %%% Оформление заголовков глав, разделов, подразделов %%%

\newcommand{\printTheAbstract}{%распечатать the Abstract
	\begingroup
	\par
	\renewcommand{\cleardoublepage}{}
	\renewcommand{\clearpage}{}
	\vspace{\theintvl\curtextsize}
	\chapter*{Abstract}
	\endgroup %after chapter in case of inline using
	\thispagestyle{empty}%
}


\makechapterstyle{SPbPUstyle}{%
	\chapterstyle{default}
	\setlength{\beforechapskip}{0pt}
	\setlength{\midchapskip}{0pt} 
	\setlength{\afterchapskip}{\intvlS\curtextsize}
	\renewcommand*{\chapnamefont}{\normalfont\bfseries\MakeTextUppercase} %не используется слово <<Глава>>
	\renewcommand*{\chapnumfont}{\normalfont\bfseries\MakeTextUppercase}
%	\renewcommand*{\chaptitlefont}{\normalfont\bfseries\MakeTextUppercase} %не работает \MakeTextUppercase
	\renewcommand\printchaptertitle{\normalfont\bfseries\MakeTextUppercase}% аналог \chaptitlefont
	\renewcommand*{\chapterheadstart}{}
	\ifnumgreater{\value{headingdelim}}{0}{%
		\renewcommand*{\afterchapternum}{.\space}   % добавляет точку с пробелом после номера раздела
	}{%
		\renewcommand*{\afterchapternum}{.\quad}     % добавляет точку и \space (\quad) после номера раздела
	} % настройки добавление в СОДЕРЖАНИЕ (по умолчанию название раздела переходит само)
	\renewcommand*{\printchapternum}{\hdngaligni\hdngalign\chapnumfont \thechapter}
	\renewcommand*{\printchaptername}{}
	\renewcommand*{\printchapternonum}{\hdngaligni\hdngalign}
	}
\newcommand{\chapterLight}{\normalfont\MakeTextUppercase\normalsize} %не менять последовательность команд!
\renewcommand*{\printtoctitle}[1]{\normalfont\MakeTextUppercase #1} %слово <<Content>> в стилю chaperLight, по факту убираем \bfseries
%\chapterLight не действует в этой команде
\makeatletter


\makechapterstyle{SPbPUstylechapname}{% для <<будет вписано слово Глава перед каждым номером раздела в оглавлении>>
	\chapterstyle{SPbPUstyle}
	\renewcommand*{\printchapternum}{\chapnumfont \thechapter}
	\renewcommand*{\printchaptername}{\hdngaligni\hdngalign\chapnamefont \@chapapp} %

}
\makeatother

\chapterstyle{SPbPUstyle}

%% удалить перенос на новую строку перед командой \chapter
\newcommand{\delnewpagebeforech}{
	%\begingroup
	\renewcommand{\cleardoublepage}{}
	\renewcommand{\clearpage}{}
	\vspace{\theintvl\curtextsize}
	%\endgroup %after chapter in case of inline using
}

%% Оформление шрифтов и отсупов подразделов, подподразделов и подподподразделов

\makeatletter
\setsecheadstyle{\normalfont\bfseries\hdngalign}
\setsecindent{\otstuplen} %отступ от левого края живописного поля
\setbeforesecskip{\intvlS\curtextsize} %базовые настройки с плюс/минус точностью, что позволяет более гибко располагать рисунки и изображения на странице
\setaftersecskip{\intvlS\curtextsize}


\setsubsecheadstyle{\normalfont\bfseries\itshape\hdngalign}
\setsubsecindent{\otstuplen}
\setbeforesubsecskip{1\curtextsize}
\setaftersubsecskip{1\curtextsize}

\setsubsubsecheadstyle{\normalfont\itshape\hdngalign}
\setsubsubsecindent{\otstuplen}
\setbeforesubsubsecskip{1\curtextsize}
\setaftersubsubsecskip{1\curtextsize}

%ОLD  ГИА
%\setsubsecheadstyle{\normalfont\hdngalign}
%\setsubsecindent{\otstuplen}
%\setbeforesubsecskip{\intvlS\curtextsize}
%\setaftersubsecskip{\intvlS\curtextsize}
%
%\setsubsubsecheadstyle{\normalfont\hdngalign}
%\setsubsubsecindent{\otstuplen}
%\setbeforesubsubsecskip{\intvlS\curtextsize}
%\setaftersubsubsecskip{\intvlS\curtextsize}
\makeatother

%попытки форматирования \part можно продолжить
%сейчас реализован более простой вариант
\renewcommand{\partnamefont}{\LARGE\MakeTextUppercase}
\renewcommand{\partnumfont}{\LARGE\MakeTextUppercase}
\renewcommand*{\parttitlefont}{\LARGE\MakeTextUppercase}

%[section], чтобы заставить все floats быть до расположиться до окончания подраздела
%\FloatBarrier локальное ограничение, чтобы 
% расставить повсеместно по разделам, то всего лишь подключить [section];
% разрешить до \FloatBarrier размещать foats, то добавить окцию  [above].
\usepackage[above]{placeins} 

\sethangfrom{\noindent #1} %все заголовки подразделов центрируются с учетом номера, как block 

\ifnumequal{\value{chapstyle}}{1}{%
    \chapterstyle{SPbPUstylechapname}
    \renewcommand*{\cftchaptername}{Глава\space} % будет вписано слово Глава перед каждым номером раздела в оглавлении
}{}% вместо Chapter \chaptername

%%% Интервалы между заголовками
%\setbeforesecskip{\theintvl\curtextsize}% Заголовки отделяют от текста сверху и снизу тремя интервалами (ГОСТ Р 7.0.11-2011, 5.3.5).
%\setaftersecskip{\theintvl\curtextsize}
%\setbeforesubsecskip{\theintvl\curtextsize}
%\setaftersubsecskip{\theintvl\curtextsize}
%\setbeforesubsubsecskip{\theintvl\curtextsize}
%\setaftersubsubsecskip{\theintvl\curtextsize}


%%% Блок дополнительного управления размерами заголовков
\ifnumequal{\value{headingsize}}{1}{% Пропорциональные заголовки и базовый шрифт 14 пт
	\renewcommand{\normalfont}{\large\bfseries}
	\renewcommand*{\chapnamefont}{\Large\bfseries}
	\renewcommand*{\chapnumfont}{\Large\bfseries}
	\renewcommand*{\chaptitlefont}{\Large\bfseries}
}{}




% ОФОРМЛЕНИЕ Приложений Appendix - Вариант 2 - действующий
%https://stackoverflow.com/questions/717316/how-to-make-appendix-appear-in-toc-in-latex
\makeatletter
\newcommand\appendix@chapter[1]{%
	\renewcommand*{\chapnamefont}{\normalfont\normalsize\bfseries} %не используется слово <<Глава>>
	\renewcommand*{\chapnumfont}{\normalfont\normalsize\bfseries}
	\renewcommand\printchaptertitle{\normalfont\normalsize\bfseries}
	\renewcommand*{\printchapternum}{\chapnumfont \thechapter}
	\renewcommand*{\printchaptername}{\hdngaligni\hdngalign\chapnamefont \@chapapp} %
	\renewcommand*\thechapter{\arabic{chapter}} % работает
	\settocdepth{chapter} % выводить только названия Приложений
	\refstepcounter{chapter}%
	\def\app@ct{\hfill{}\appendixname{} {}\@arabic\c@chapter %
%	\vspace{\intvlS\curtextsize}
	\newline #1
	\vspace{\curtextsize}
}
	\orig@chapter*{\app@ct}%
	\addcontentsline{toc}{chapter}{\appendixname{} \@arabic\c@chapter. #1}%\app@ct % input to TOC-table
}
\let\orig@chapter\chapter
\g@addto@macro\appendix{\newpage\let\chapter\appendix@chapter\renewcommand*{\afterchapternum}{\par\nobreak\vskip \midchapskip}}
\makeatother


%https://tex.stackexchange.com/questions/250834/dont-break-page-for-new-chapter-unless-chapter-heading-wont-fully-fit-on-curre
\newcommand{\ContinueChapterBegin}{%
\let\clearpage\relax
\renewcommand*{\chapterheadstart}{%
	\FloatBarrier % make floats stop
\par
\ifartopt % если сверху сраницы, то
% ничего не делать
\else % в противном случае
\vspace{\theintvl\curtextsize} % добавить интервал
\fi
}
}%

\newcommand{\ContinueChapterEnd}{%
	\let\clearpage\newpage
\renewcommand*{\chapterheadstart}{% ничего не делаем
\FloatBarrier % make floats stop
}
}%

\newcommand{\NewPage}{% в случае, если на последней странице приложения есть <<висячая>> таблица или рисунок
\newpage\leavevmode\thispagestyle{plain}\newpage %начать новое приложение с новой страницы 
}%




\makeatletter %настройка отображения floates
\setlength{\@fptop}{0pt}%отключить вертикальное центрирование рисунка/таблицы на странице
%\setlength{\@fpsep}{8pt}%отключить вертикальное центрирование рисунка/таблицы на странице
%\setlength{\@fpbot}{0pt plus 1fil}%отключить вертикальное центрирование рисунка на странице
\makeatother



%%% Счётчики %%%

%% DOI
\newcounter{mychapternumber} 
\newcounter{chapterDOI}

%% Упрощённые настройки шаблона диссертации: нумерация формул, таблиц, рисунков
\ifnumequal{\value{contnumeq}}{1}{%
    \counterwithout{equation}{chapter} % Убираем связанность номера формулы с номером главы/раздела
}{}
\ifnumequal{\value{contnumfig}}{1}{%
    \counterwithout{figure}{chapter}   % Убираем связанность номера рисунка с номером главы/раздела
}{}
\ifnumequal{\value{contnumtab}}{1}{%
    \counterwithout{table}{chapter}    % Убираем связанность номера таблицы с номером главы/раздела
}{}


%%%http://www.linux.org.ru/forum/general/6993203#comment-6994589 (используется totcount)
\makeatletter
\def\formbytotal#1#2#3#4#5{%
    \newcount\@c
    \@c\totvalue{#1}\relax
    \newcount\@last
    \newcount\@pnul
    \@last\@c\relax
    \divide\@last 10
    \@pnul\@last\relax
    \divide\@pnul 10
    \multiply\@pnul-10
    \advance\@pnul\@last
    \multiply\@last-10
    \advance\@last\@c
    \total{#1}~#2%
    \ifnum\@pnul=1#5\else%
    \ifcase\@last#5\or#3\or#4\or#4\or#4\else#5\fi
    \fi
}
\makeatother

% xassoccnt to make total values: tables, figures, chapters 
%https://tex.stackexchange.com/questions/295857/calculate-amount-of-figures?noredirect=1
\NewTotalDocumentCounter{mytotalfigures}
\NewTotalDocumentCounter{mytotalfiguresInApp}
\NewTotalDocumentCounter{mytotaltables}
\NewTotalDocumentCounter{mytotaltablesInApp}
\NewTotalDocumentCounter{myappendices}
\DeclareAssociatedCounters{figure}{mytotalfigures}
\DeclareAssociatedCounters{table}{mytotaltables}

%https://tex.stackexchange.com/questions/317434/mytotal-pages-number-warning-and-miscalculated
%\NewTotalDocumentCounter{mytotalpages} % not supported yet in xassoccnt, use totpages package
%\DeclareAssociatedCounters{page}{mytotalpages}

%счетчики для вывода на печать
\newcounter{mypages} % счетчик 
\setcounter{mypages}{0} % 
\newcounter{mytotalpagesInApp} % cчетчик 
\setcounter{mytotalpagesInApp}{0} %
\newcounter{myfigures} % счетчик 
\setcounter{myfigures}{0} % 
\newcounter{mytables} % счетчик 
\setcounter{mytables}{0} %  




\AtBeginDocument{
	%% регистрируем счётчики в системе totcounter
	%% позволяет использовать: 
	%% 1) команду \total{counter} для печати значения
	%% 2) спрягать значения слов с помощью \formbytotal
	\regtotcounter{mypages}      % simple counter
	\regtotcounter{TotPages}     % totpages package
	\regtotcounter{myfigures}      % simple counter
	\regtotcounter{mytotalfigures} % xassoccnt package
	\regtotcounter{mytables}      % simple counter
	\regtotcounter{mytotaltables} % xassoccnt package
	\regtotcounter{myappendices}  % xassoccnt package
}
\newtotcounter{citenum} %счетчик для библиографии из totcount package


\preto\appendix{% когда будет команда \appendix 
	% см. также выше переопределение chapter для appendix
	%% Сохранение сумм: рисунки, таблицы, страницы.
	\setcounter{mytotalpagesInApp}{\value{TotPages}}% 
	% count total values
	\AddAssociatedCounters{figure}{mytotalfiguresInApp}
	\AddAssociatedCounters{table}{mytotaltablesInApp}
	\AddAssociatedCounters{chapter}{myappendices}
	%% Форматирование
	%\renewcommand\thechapter{\arabic{chapter}} % см. переопределение chapter для appendix
	\renewcommand{\appendixname}{Приложение} %
	\renewcommand{\thetable}{П\thechapter.\arabic{table}}
	\renewcommand{\thefigure}{П\thechapter.\arabic{figure}}
	\renewcommand{\theequation}{П\thechapter.\arabic{equation}}
	\renewcommand{\thesection}{П\thechapter.\arabic{section}}
	\renewcommand{\thesubsection}{\thesection.\arabic{subsection}}
	\renewcommand{\thesubsubsection}{\thesubsection.\arabic{subsubsection}}
	\counterwithin{footnote}{chapter} %связанная нумерация глав-сносок
	\renewcommand{\thefootnote}{П\thechapter.\arabic{footnote}}
}


%%% Подсчет сумм: рисунки, таблицы, страницы
%% Вариант 1 (рабочий)
\AtEndDocument{
	\setcounter{myfigures}{\value{mytotalfigures}}%
	\addtocounter{myfigures}{-\value{mytotalfiguresInApp}}%
	\setcounter{mytables}{\value{mytotaltables}}%
	\addtocounter{mytables}{-\value{mytotaltablesInApp}}%
	\setcounter{mypages}{\value{mytotalpagesInApp}}%
	\addtocounter{mypages}{1}%
%	\addtocounter{mypages}{-\value{mytotalpagesInApp}}%
}
%% Вариант 2 (для отладки)
%% работает только в месте вывода на экран суммы, т.е. в реферате
%\setcounter{myfigures}{\numexpr\TotalValue{mytotalfigures}-\TotalValue{mytotalfiguresInApp}\relax}



%%% Правильная нумерация приложений %%%
%% По ГОСТ 2.105, п. 4.3.8 Приложения обозначают заглавными буквами русского алфавита,
%% начиная с А, за исключением букв Ё, З, Й, О, Ч, Ь, Ы, Ъ.
%% Здесь также переделаны все нумерации русскими буквами.
%\ifxetexorluatex
%    \makeatletter
%    \def\russian@Alph#1{\ifcase#1\or
%       А\or Б\or В\or Г\or Д\or Е\or Ж\or
%       И\or К\or Л\or М\or Н\or
%       П\or Р\or С\or Т\or У\or Ф\or Х\or
%       Ц\or Ш\or Щ\or Э\or Ю\or Я\else\xpg@ill@value{#1}{russian@Alph}\fi}
%    \def\russian@alph#1{\ifcase#1\or
%       а\or б\or в\or г\or д\or е\or ж\or
%       и\or к\or л\or м\or н\or
%       п\or р\or с\or т\or у\or ф\or х\or
%       ц\or ш\or щ\or э\or ю\or я\else\xpg@ill@value{#1}{russian@alph}\fi}
%    \makeatother
%\else
%    \makeatletter
%    \if@uni@ode
%      \def\russian@Alph#1{\ifcase#1\or
%        А\or Б\or В\or Г\or Д\or Е\or Ж\or
%        И\or К\or Л\or М\or Н\or
%        П\or Р\or С\or Т\or У\or Ф\or Х\or
%        Ц\or Ш\or Щ\or Э\or Ю\or Я\else\@ctrerr\fi}
%    \else
%      \def\russian@Alph#1{\ifcase#1\or
%        \CYRA\or\CYRB\or\CYRV\or\CYRG\or\CYRD\or\CYRE\or\CYRZH\or
%        \CYRI\or\CYRK\or\CYRL\or\CYRM\or\CYRN\or
%        \CYRP\or\CYRR\or\CYRS\or\CYRT\or\CYRU\or\CYRF\or\CYRH\or
%        \CYRC\or\CYRSH\or\CYRSHCH\or\CYREREV\or\CYRYU\or
%        \CYRYA\else\@ctrerr\fi}
%    \fi
%    \if@uni@ode
%      \def\russian@alph#1{\ifcase#1\or
%        а\or б\or в\or г\or д\or е\or ж\or
%        и\or к\or л\or м\or н\or
%        п\or р\or с\or т\or у\or ф\or х\or
%        ц\or ш\or щ\or э\or ю\or я\else\@ctrerr\fi}
%    \else
%      \def\russian@alph#1{\ifcase#1\or
%        \cyra\or\cyrb\or\cyrv\or\cyrg\or\cyrd\or\cyre\or\cyrzh\or
%        \cyri\or\cyrk\or\cyrl\or\cyrm\or\cyrn\or
%        \cyrp\or\cyrr\or\cyrs\or\cyrt\or\cyru\or\cyrf\or\cyrh\or
%        \cyrc\or\cyrsh\or\cyrshch\or\cyrerev\or\cyryu\or
%        \cyrya\else\@ctrerr\fi}
%    \fi
%    \makeatother
%\fi


%%% Алгоритмы %%%

%\usepackage[linesnumbered]{algorithm2e}
\usepackage[linesnumbered,vlined,figure,scleft]{algorithm2e}

%% Glogal params %%
%ruled, tworuled, algoruled --- put lines to wrap the caption plus a line at the bottom (top) - one should not use this together with inline captions!
%vlined 										--- instead of begin...end will be lines
%boxed 											--- make a box
% figure 										--- count as Fig. ...


% Settings of caption       --- if one will use \caption{} option 	instead of inline + environment caption.
%\SetAlgoCaptionSeparator{.}
%\SetAlgorithmName{Algorithm}{} %last arg is the title of listing table


% Settings for lines numbers
\SetNlSkip{0em}							% sets the value of the space between the line numbers and the text, by default 1em.
\SetNlSty{normalsize}{\hphantom{0}}{.}%defines how to print line numbers
%\hspace*{5mm} does not work 
\SetAlgoNlRelativeSize{-1}	% sets the relative size of line numbers. By default, line numbers are two size smaller than algorithm text

% How to ignore line nuber and to wrap
%http://tex.stackexchange.com/questions/153646/algorithm2e-disabling-line-numbers-for-specific-lines
%http://tex.stackexchange.com/questions/86580/how-to-adjust-line-numbers-of-algorithm2e-package
\makeatletter
%\newcommand{\nosemic}{\renewcommand{\@endalgocfline}{\relax}}% Drop semi-colon ;
%\newcommand{\dosemic}{\renewcommand{\@endalgocfline}{\algocf@endline}}% Reinstate semi-colon ;
%\newcommand{\pushline}{\Indp}% Indent
%\newcommand{\popline}{\Indm\dosemic}% Undent
\let\oldnl\nl% Store \nl in \oldnl
\newcommand{\nonl}{\renewcommand{\nl}{\let\nl\oldnl}}% Remove line number for one line
\makeatother


% Settings for vlines 			
%\SetInd{0.3em}{0.5em}			%default and spaces before and after are 0.5em and 1em
%\SetVlineSkip{5em}					% Sets the value of the vertical space after the little horizontal line which closes a block in vlined mode

%% User abbreviations for ASTRA %%
\SetKwInput{KwInput}{Input}
\SetKwInput{KwOutput}{Output}
%% See also %%
%http://tex.stackexchange.com/questions/145114/caption-below-boxed-algorithm2e-when-used-as-a-figure
%http://tex.stackexchange.com/questions/83536/align-comments-in-algorithm-with-package-algorithm2e



           
\input{template_settings/Dissertation/userstyles}          
%%% Библиография. Общие настройки для двух способов её подключения %%%


%%% Выбор реализации %%%
\ifnumequal{\value{bibliosel}}{0}{%
    \input{template_settings/biblio/predefined}  % Встроенная реализация с загрузкой файла через движок bibtex8
}{
    %%% Реализация библиографии пакетами biblatex и biblatex-gost с использованием движка biber %%%

%\usepackage{csquotes} % biblatex рекомендует его подключать. Пакет для оформления сложных блоков цитирования.
%%% Загрузка пакета с основными настройками %%%
\ifnumequal{\value{draft}}{0}{% Чистовик
\usepackage[%
backend=biber,% движок
bibencoding=utf8,% кодировка bib файла
%sorting=none,% настройка сортировки списка литературы
style=gost-numeric,% стиль цитирования и библиографии (по ГОСТ)
%%style=gost-authoryear,% стиль цитирования и библиографии (по ГОСТ)
%%%% выставить следующую опцию <<babel>> и закомментировать <<language=english>> для достижения многоязычных ссылок
babel=other, %выставим для отображения разных языков
%%language=english,%только английский = \setlanguage{}; autobib получение языка из babel/polyglossia, default: autobib % если ставить autocite или auto, то цитаты в тексте с указанием страницы, получат указание страницы на языке оригинала
%%autolang=other,%other многоязычная библиография
%%clearlang=true,% внутренний сброс поля language, если он совпадает с языком из babel/polyglossia
defernumbers=true,% нумерация проставляется после двух компиляций, зато позволяет выцеплять библиографию по ключевым словам и нумеровать не из большего списка
sortcites=true,% сортировать номера затекстовых ссылок при цитировании (если в квадратных скобках несколько ссылок, то отображаться будут отсортированно, а не абы как)
doi=true,% Показывать или нет ссылки на DOI
isbn=false% Показывать или нет ISBN, ISSN, ISRN
]{biblatex}[2016/09/17]%
}{%Черновик
\usepackage[%
backend=biber,% движок
bibencoding=utf8,% кодировка bib файла
sorting=none,% настройка сортировки списка литературы
]{biblatex}[2016/09/17]%
}
%%TO-DO: продумать автозамену всех полей hyphenation на language



%%% Подключение файлов bib %%%
%\addbibresource[label=other]{biblio/othercites.bib}
%\addbibresource[label=vak]{biblio/authorpapersVAK.bib}
%\addbibresource[label=papers]{biblio/authorpapers.bib}
%\addbibresource[label=conf]{biblio/authorconferences.bib}



%http://tex.stackexchange.com/a/141831/79756
%There is a way to automatically map the language field to the langid field. The following lines in the preamble should be enough to do that.
%This command will copy the language field into the  field and will then delete the contents of the language field. The language field will only be deleted if it was successfully copied into the langid field.
\DeclareSourcemap{ %модификация bib файла перед тем, как им займётся biblatex 
    \maps{
    	%% SPbPU
    	%% https://tex.stackexchange.com/a/229505/44348
    	\map{% delete month
    		\step[fieldset=month, null]
    		\step[fieldsource=date,
    		match=\regexp{([0-9]{4})-[0-9]{2}(-[0-9]{2})*},
    		replace=\regexp{$1}$5] % <<$5>> only for syntax highlihgting in IDE
    	}
%    	\map{% set current urldate
%    	\step[fieldset=urldate, null]	
%		\step[fieldset=urldate,fieldvalue={\the\year-\the\month-\the\day}]
%    	} 
%		\map{% не отображаем поле ``Глава''
%			\step[fieldset=chapter, null]
%			\step[fieldset=editor, null]
%		}
%    	} 
		\map{% перекидываем значения полей hyphenation в поля langid, которыми пользуется biblatex
			\step[fieldsource=hyphenation, fieldset=langid, origfieldval, final]
		}
        \map{% перекидываем значения полей language в поля langid, которыми пользуется biblatex
            \step[fieldsource=language, fieldset=langid, origfieldval, final]
            \step[fieldset=language, null]
        }
%        \map[overwrite, refsection=0]{% стираем значения всех полей addendum
%            \perdatasource{biblio/authorpapersVAK.bib}
%            \perdatasource{biblio/authorpapers.bib}
%            \perdatasource{biblio/authorconferences.bib}
%            \step[fieldsource=addendum, final]
%            \step[fieldset=addendum, null] %чтобы избавиться от информации об объёме авторских статей, в отличие от автореферата
%        }
        \map{% перекидываем значения полей numpages в поля pagetotal, которыми пользуется biblatex
            \step[fieldsource=numpages, fieldset=pagetotal, origfieldval, final]
            \step[fieldset=pagestotal, null]
        }
        \map{% если в поле medium написано "Электронный ресурс", то устанавливаем поле media, которым пользуется biblatex, в значение eresource.
            \step[fieldsource=medium,
            match=\regexp{Электронный\s+ресурс},
            final]
            \step[fieldset=media, fieldvalue=eresource]
        }
        \map[overwrite]{% стираем значения всех полей issn
            \step[fieldset=issn, null]
        }
        \map[overwrite]{% стираем значения всех полей abstract, поскольку ими не пользуемся, а там бывают "неприятные" латеху символы
            \step[fieldsource=abstract]
            \step[fieldset=abstract,null]
        }
        \map[overwrite]{ % переделка формата записи даты
            \step[fieldsource=urldate,
            match=\regexp{([0-9]{2})\.([0-9]{2})\.([0-9]{4})},
            replace={$3-$2-$1$4}, %, % $4 вставлен исключительно ради нормальной работы программ подсветки синтаксиса, которые некорректно обрабатывают $ в таких конструкциях
            final]
        }
%        \map[overwrite]{ % добавляем ключевые слова, чтобы различать источники
%            \perdatasource{biblio/othercites.bib}
%            \step[fieldset=keywords, fieldvalue={biblioother,bibliofull}]
%        }
%        \map[overwrite]{ % добавляем ключевые слова, чтобы различать источники
%            \perdatasource{biblio/authorpapersVAK.bib}
%            \step[fieldset=keywords, fieldvalue={biblioauthorvak,biblioauthor,bibliofull}]
%        }
%        \map[overwrite]{ % добавляем ключевые слова, чтобы различать источники
%            \perdatasource{biblio/authorpapers.bib}
%            \step[fieldset=keywords, fieldvalue={biblioauthornotvak,biblioauthor,bibliofull}]
%        }
%        \map[overwrite]{ % добавляем ключевые слова, чтобы различать источники
%            \perdatasource{biblio/authorconferences.bib}
%            \step[fieldset=keywords, fieldvalue={biblioauthorconf,biblioauthor,bibliofull}]
%        }
%        \map[overwrite]{% стираем значения всех полей series
%            \step[fieldset=series, null]
%        }
        \map[overwrite]{% перекидываем значения полей howpublished в поля organization для типа online
            \step[typesource=online, typetarget=online, final]
            \step[fieldsource=howpublished, fieldset=organization, origfieldval]
            \step[fieldset=howpublished, null]
        }
        % Так отключаем [Электронный ресурс]
%        \map[overwrite]{% стираем значения всех полей media=eresource
%            \step[fieldsource=media,
%            match={eresource},
%            final]
%            \step[fieldset=media, null]
%        }
		\map{
			\step[fieldsource=langid, match=russian, final]
			\step[fieldset=presort, fieldvalue={a}]
		}
		\map{
			\step[fieldsource=langid, notmatch=russian, final]
			\step[fieldset=presort, fieldvalue={z}]
		}%    
	}
}


%\DeclareSourcemap{
%	\maps[datatype=bibtex]{
%		\map{
%			\step[fieldsource=langid, match=russian, final]
%			\step[fieldset=presort, fieldvalue={a}]
%		}
%		\map{
%			\step[fieldsource=langid, notmatch=russian, final]
%			\step[fieldset=presort, fieldvalue={z}]
%		}
%	}
%}


%%% Убираем неразрывные пробелы перед двоеточием и точкой с запятой %%%
%\makeatletter
%\ifnumequal{\value{draft}}{0}{% Чистовик
%    \renewcommand*{\addcolondelim}{%
%      \begingroup%
%      \def\abx@colon{%
%        \ifdim\lastkern>\z@\unkern\fi%
%        \abx@puncthook{:}\space}%
%      \addcolon%
%      \endgroup}
%
%    \renewcommand*{\addsemicolondelim}{%
%      \begingroup%
%      \def\abx@semicolon{%
%        \ifdim\lastkern>\z@\unkern\fi%
%        \abx@puncthook{;}\space}%
%      \addsemicolon%
%      \endgroup}
%}{}
%\makeatother

%%% Правка записей типа thesis, чтобы дважды не писался автор
%\ifnumequal{\value{draft}}{0}{% Чистовик
%\DeclareBibliographyDriver{thesis}{%
%  \usebibmacro{bibindex}%
%  \usebibmacro{begentry}%
%  \usebibmacro{heading}%
%  \newunit
%  \usebibmacro{author}%
%  \setunit*{\labelnamepunct}%
%  \usebibmacro{thesistitle}%
%  \setunit{\respdelim}%
%  %\printnames[last-first:full]{author}%Вот эту строчку нужно убрать, чтобы автор диссертации не дублировался
%  \newunit\newblock
%  \printlist[semicolondelim]{specdata}%
%  \newunit
%  \usebibmacro{institution+location+date}%
%  \newunit\newblock
%  \usebibmacro{chapter+pages}%
%  \newunit
%  \printfield{pagetotal}%
%  \newunit\newblock
%  \usebibmacro{doi+eprint+url+note}%
%  \newunit\newblock
%  \usebibmacro{addendum+pubstate}%
%  \setunit{\bibpagerefpunct}\newblock
%  \usebibmacro{pageref}%
%  \newunit\newblock
%  \usebibmacro{related:init}%
%  \usebibmacro{related}%
%  \usebibmacro{finentry}}
%}{}

%\newbibmacro{string+doi}[1]{% новая макрокоманда на простановку ссылки на doi
%    \iffieldundef{doi}{#1}{\href{http://dx.doi.org/\thefield{doi}}{#1}}}

%\ifnumequal{\value{draft}}{0}{% Чистовик
%\renewcommand*{\mkgostheading}[1]{\usebibmacro{string+doi}{#1}} % ссылка на doi с авторов. стоящих впереди записи
%\renewcommand*{\mkgostheading}[1]{#1} % только лишь убираем курсив с авторов
%}{}
%\DeclareFieldFormat{title}{\usebibmacro{string+doi}{#1}} % ссылка на doi с названия работы
%\DeclareFieldFormat{journaltitle}{\usebibmacro{string+doi}{#1}} % ссылка на doi с названия журнала
%%% Убрать тире из разделителей элементов в библиографии:
%\renewcommand*{\newblockpunct}{%
%    \addperiod\space\bibsentence}%block punct.,\bibsentence is for vol,etc.

%%% Возвращаем запись «Режим доступа» %%%
%\DefineBibliographyStrings{english}{%
%    urlfrom = {Mode of access}
%}
%\DeclareFieldFormat{url}{\bibstring{urlfrom}\addcolon\space\url{#1}}

%%%% В списке литературы обозначение одной буквой диапазона страниц англоязычного источника %%%
\DefineBibliographyStrings{english}{%
    pages = {P\adddot} %заглавность буквы затем по месту определяется работой самого biblatex
}

%%% В ссылке на источник в основном тексте с указанием конкретной страницы обозначение одной большой буквой %%%
%\DefineBibliographyStrings{russian}{%
%    page = {C\adddot}
%}

%%% Исправление длины тире в диапазонах %%%
%\DefineBibliographyExtras{russian}{%
%  \protected\def\bibrangedash{%
%    \textendash\penalty\value{abbrvpenalty}}% almost unbreakable dash
%  \protected\def\bibdaterangesep{\bibrangedash}%тире для дат
%}

%Set higher penalty for breaking in number, dates and pages ranges
\setcounter{abbrvpenalty}{10000} % default is \hyphenpenalty which is 12

%Set higher penalty for breaking in names
\setcounter{highnamepenalty}{10000} % If you prefer the traditional BibTeX behavior (no linebreaks at highnamepenalty breakpoints), set it to ‘infinite’ (10 000 or higher).
\setcounter{lownamepenalty}{10000}

%%% Set low penalties for breaks at uppercase letters and lowercase letters
%\setcounter{biburllcpenalty}{500} %управляет разрывами ссылок после маленьких букв RTFM biburllcpenalty
%\setcounter{biburlucpenalty}{3000} %управляет разрывами ссылок после больших букв, RTFM biburlucpenalty

%% Список литературы с красной строки (без висячего отступа) %%%
%\printfield  ----  This command prints a hfieldi using the formatting directive hformati, as defined
%with \DeclareFieldFormat. 
%\printtext   ----  This command prints htexti, which may be printable text or arbitrary code generating
%printable text. It clears the punctuation buffer before inserting htexti and
%informs biblatex that printable text has been inserted.
%https://github.com/odomanov/biblatex-gost/blob/master/tex/latex/biblatex-gost/bbx/gost-numeric.bbx
\defbibenvironment{SSTfirst} % Примерно соответствует 1 варианту оформления списка использованных источников
  {\list
     {
    \printtext[labelnumberwidth]{%
	\printfield{labelprefix}%not numberprefix !
	\printfield{labelnumber}}
	}%
     {%
      \setlength{\labelwidth}{\labelnumberwidth}% This length register indicates the width of the widest labelnumber
      \addtolength{\labelwidth}{6pt}% \labelwidth Сдвигаем label, чтобы визуально сравнять с enumitem
      \setlength{\leftmargin}{\labelwidth}% default is \labelwidth используют также \parindent в enumerations
      \setlength{\labelsep}{\widthof{\  }}% Управляет длиной отступа после точки % default is \biblabelsep
      \setlength{\leftskip}{-2.2em}
      \addtolength{\leftmargin}{\leftskip}%<----- here
      \setlength{\itemsep}{0pt}% Управление дополнительным вертикальным разрывом между записями. \bibitemsep по умолчанию соответствует \itemsep списков в документе.
      \setlength{\itemindent}{\bibhang}% Пользуемся тем, что \bibhang по умолчанию принимает значение \parindent (абзацного отступа), который переназначен в styles.tex
      \addtolength{\itemindent}{\labelwidth}% \labelwidth Сдвигаем правее на величину номера с точкой
      \addtolength{\itemindent}{\labelsep}% \labelsep Сдвигаем ещё правее на отступ после точки
      \setlength{\parsep}{\bibparsep}% расстояние между параграфами 
     }%
      \renewcommand*{\makelabel}[1]{\hss##1} % выравнивание по labelnumberwidth >= 2 строки item
  }
  {\endlist}
  {\item}
% % % %
\defbibenvironment{tsk} % переопределяем окружение библиографии из gost-numeric.bbx пакета biblatex-gost
  {\list
	{
		\printtext[labelnumberwidth]{%
			\printfield{labelprefix}%not numberprefix !
			\printfield{labelnumber}}
	}%
	{%
		\setlength{\labelwidth}{\labelnumberwidth}% This length register indicates the width of the widest labelnumber
		%      \addtolength{\labelwidth}{-3pt}% \labelwidth Сдвигаем label, чтобы визуально сравнять с enumitem
		\setlength{\leftmargin}{\labelwidth}% default is \labelwidth используют также \parindent в enumerations
		\setlength{\labelsep}{\widthof{\  }}% Управляет длиной отступа после точки % default is \biblabelsep
		\setlength{\itemsep}{0pt}% Управление дополнительным вертикальным разрывом между записями. \bibitemsep по умолчанию соответствует \itemsep списков в документе.
		\setlength{\itemindent}{0pt}% Пользуемся тем, что \bibhang по умолчанию принимает значение \parindent (абзацного отступа), который переназначен в styles.tex
		\addtolength{\itemindent}{0pt}% \labelwidth Сдвигаем правее на величину номера с точкой
		\addtolength{\itemindent}{0pt}% \labelsep Сдвигаем ещё правее на отступ после точки
		\setlength{\parsep}{\bibparsep}% расстояние между параграфами 
	}%
	\renewcommand*{\makelabel}[1]{\hss##1} % выравнивание по labelnumberwidth >= 2 строки item
}
{\endlist}
{\item}

%%% Счётчик использованных ссылок на литературу, обрабатывающий с учётом неоднократных ссылок
%%http://tex.stackexchange.com/a/66851/79756
%\newcounter{citenum}
%\newtotcounter{citenum}
%\makeatletter
%\defbibenvironment{counter} %Env of bibliography
%  {\setcounter{citenum}{0}%
%  \renewcommand{\blx@driver}[1]{}%
%  } %what is doing at the beginining of bibliography. In your case it's : a. Reset counter b. Say to print nothing when a entry is tested.
%  {} %Здесь то, что будет выводиться командой \printbibliography. \thecitenum сюда писать не надо
%  {\stepcounter{citenum}} %What is printing / executed at each entry.
%\makeatother
%\defbibheading{counter}{}

%
%
%\newtotcounter{citeauthorvak}
%\makeatletter
%\defbibenvironment{countauthorvak} %Env of bibliography
%{\setcounter{citeauthorvak}{0}%
%    \renewcommand{\blx@driver}[1]{}%
%} %what is doing at the beginining of bibliography. In your case it's : a. Reset counter b. Say to print nothing when a entry is tested.
%{} %Здесь то, что будет выводиться командой \printbibliography. Обойдёмся без \theciteauthorvak в нашей реализации
%{\stepcounter{citeauthorvak}} %What is printing / executed at each entry.
%\makeatother
%\defbibheading{countauthorvak}{}
%
%\newtotcounter{citeauthornotvak}
%\makeatletter
%\defbibenvironment{countauthornotvak} %Env of bibliography
%{\setcounter{citeauthornotvak}{0}%
%    \renewcommand{\blx@driver}[1]{}%
%} %what is doing at the beginining of bibliography. In your case it's : a. Reset counter b. Say to print nothing when a entry is tested.
%{} %Здесь то, что будет выводиться командой \printbibliography. Обойдёмся без \theciteauthornotvak в нашей реализации
%{\stepcounter{citeauthornotvak}} %What is printing / executed at each entry.
%\makeatother
%\defbibheading{countauthornotvak}{}
%
%\newtotcounter{citeauthorconf}
%\makeatletter
%\defbibenvironment{countauthorconf} %Env of bibliography
%{\setcounter{citeauthorconf}{0}%
%    \renewcommand{\blx@driver}[1]{}%
%} %what is doing at the beginining of bibliography. In your case it's : a. Reset counter b. Say to print nothing when a entry is tested.
%{} %Здесь то, что будет выводиться командой \printbibliography. Обойдёмся без \theciteauthorconf в нашей реализации
%{\stepcounter{citeauthorconf}} %What is printing / executed at each entry.
%\makeatother
%\defbibheading{countauthorconf}{}
%
%\newtotcounter{citeauthor}
%\makeatletter
%\defbibenvironment{countauthor} %Env of bibliography
%{\setcounter{citeauthor}{0}%
%    \renewcommand{\blx@driver}[1]{}%
%} %what is doing at the beginining of bibliography. In your case it's : a. Reset counter b. Say to print nothing when a entry is tested.
%{} %Здесь то, что будет выводиться командой \printbibliography. Обойдёмся без \theciteauthor в нашей реализации
%{\stepcounter{citeauthor}} %What is printing / executed at each entry.
%\makeatother
%\defbibheading{countauthor}{}
%
%\defbibheading{authorpublications}[\authorbibtitle]{\section*{#1}}
%\defbibheading{pubsubgroup}{\noindent\textbf{#1}}
%\defbibheading{otherpublications}{\section*{#1}}


%%%% Создание команд для вывода списка литературы %%%
%\newcommand*{\insertbibliofull}{
%\printbibliography[keyword=bibliofull,section=0]
%\printbibliography[heading=counter,env=counter,keyword=bibliofull,section=0]
%}
%
%\newcommand*{\insertbiblioauthor}{
%\printbibliography[heading=authorpublications,keyword=biblioauthor,section=1,title=\authorbibtitle]
%}
%\newcommand*{\insertbiblioauthorimportant}{
%\printbibliography[heading=authorpublications,keyword=biblioauthor,section=2,title={Наиболее значимые \MakeLowercase{\protect\authorbibtitle{}}}]
%}
%\newcommand*{\insertbiblioauthorgrouped}{% Заготовка для вывода сгруппированных печатных работ автора. Порядок нумерации определяется в соответствующих счетчиках внутри окружения refsection в файле common/characteristic.tex
%\section*{\authorbibtitle}
%\printbibliography[heading=pubsubgroup, keyword=biblioauthorvak, section=1,title=\vakbibtitle]%
%\printbibliography[heading=pubsubgroup, keyword=biblioauthorconf, section=1,title=\confbibtitle]%
%\printbibliography[heading=pubsubgroup, keyword=biblioauthornotvak, section=1,title=\notvakbibtitle]%
%}
%
%\newcommand*{\insertbiblioother}{
%\printbibliography[heading=otherpublications,keyword=biblioother]
%}




%Bibliography update
%TO delete predefined description of References
\defbibheading{bibliography}[\bibname]{%
	\markboth{#1}{#1}}
%https://tex.stackexchange.com/a/307757/44348
\newcommand{\fullbibtitle}{Библиографический список} % (ГОСТ Р 7.0.11-2011, 4)


%%%delete / in urldate (works together with \map)
\NewBibliographyString{urldateprefix}
%
\DefineBibliographyStrings{russian}{%
	urldateprefix = {дата обращения\addcolon}}
\DefineBibliographyStrings{english}{%
	urldateprefix = {visited on}}
\DeclareFieldFormat{urldate}{(\bibstring{urldateprefix}\addspace\mkdayzeros{\thefield{urlday}}\adddot\mkmonthzeros{\thefield{urlmonth}}\adddot\mkyearzeros{\thefield{urlyear}})}




%% small ``p'' before total page number of books can be made automatically only by 
%%


%renew (chapter+pages) to series format
%\newbibmacro*{chapter+pages}{%
%	\iffieldundef{chapter}
%	{}%
%	{\printfield{chapter}\space---\space}% 
%%	\setunit*{\bibpagespunct}%
%	\printfield{pages}%
%	\newunit}
\newbibmacro*{chapter+pages}{%
	\iffieldundef{chapter}
	{}%
	{\printfield{chapter}%
		\setunit*{\addspace---\space}}%
	\printfield{pages}%
	\newunit}



%delete / from date format - REPLACED BY \map to delete totally
%\DeclareFieldFormat{date}{\thefield{year}}
%TO-DO: add month with condition if it does not exist than do not write the dot
%\thefield{month}\nobreakdash\adddot

%delete : from DOI format 
\DeclareFieldFormat{doi}{%
	\mkbibacro{DOI}\space
	\ifhyperref
	{\href{https://doi.org/#1}{\nolinkurl{#1}}}
	{\nolinkurl{#1}}\isdot} 


%%%add Ser.: to series format
%% First approach
\NewBibliographyString{seriesprefix}
%
\DefineBibliographyStrings{english}{%
	seriesprefix = {Ser}}
\DefineBibliographyStrings{russian}{%
	seriesprefix = {Сер}}
\DeclareFieldFormat{series}{\bibstring{seriesprefix}\adddot\addcolon\space{#1}\isdot}

%% Second approach % nested conditions
%\DeclareFieldFormat{series}{\iffieldequalstr{langid}{russian}{Сер}{\iffieldequalstr{langid}{english}{Bingo}{NotSpecified}}\adddot\addcolon\space{#1}\isdot} %


%add brackets to note format
\DeclareFieldFormat{note}{\mkbibparens{{#1}}\isdot} 

%delete spaces before ; and :
\renewcommand*{\addcolondelim}{\addcolon\space}
\renewcommand*{\addsemicolondelim}{\addsemicolon\space}

%modify PhD theis, for candidate one can use words CANDthesis
\DefineBibliographyStrings{english}{phdthesis = {diss\adddot\space\ldots}}
\renewcommand*{\gostmedialanguage}{}    % Реализация пакетом biblatex через движок biber
}

\AtEveryBibitem{\stepcounter{citenum}}%подсчет ссылок
\input{template_settings/Dissertation/inclusioncontrol}
%TO-DO: 

% формат А5 

% масштабирование отступов и интервалов на основе параметров, зависимых от шрифтов (em, ex) 

%TO-DO warnings in draft option:
% во введении больше 4 страниц
% в заключении меньше 2 страниц
% в заключении больше 5 страниц
% ключевых слов больше 15
% ключевых слов/словосочетаний больше 5
% ключевых слов меньше 5
% ключевых слов/словосочетаний меньше 3
% в реферате больше 600 печатных знаков
% в конце названия главы/параграфа/подпараграфа отсутствуют точки
% при наличии более 1 строки в названии главы/параграфа/подпараграфа: в конце строки отсутствуют предлоги или союзы (проверка на ~) 
% в задании контрольные даты до защиты
% в библиографии дата обращения не раньше 1 дня преддипломной практики и не позднее даты загрузки ВКР

%TO_DO расширение примеров
% добавление из Положения разнообразных примеров по оформлению таблиц
% все изображения сделать более лёгкими (без расплывчатости) -> шаблон будет меньше весить
% в качестве использования цитат привести примеры на известных политехников (не современников)

%TO-DO улучшение сопутствующего ПО
% в TexStudio задать автоподстановку label в \firef{}, \taref{}.

%TO-DO синхронизация с шаблонами кандидатских и докторских диссертаций А.Акиньшина
% перенос лучшего функционала
% автоматизированная подача данных в http://vkr.spbstu.ru

%TO-DO переработка текущего функционала
%
% на титульной странице в таблице с подписями не должно быть отступов ~1мм слева и справа.
% 
% оформление приложений сейчас реализовано через <<взлом>> memoir-classa. Лучше использовать встроенный функционал, а именно определить дополнительный стиль оформления глав.

% устранить команды \NewPage: \newpage\leavevmode\thispagestyle{empty}\newpage после приложения %начать новое приложение с новой страницы % временное решение, т.к. не корректно работает \ContinueChapterPagesEnd. Пояснение:
%https://tex.stackexchange.com/questions/2958/why-is-newpage-ignored-sometimes

% проверка сортировки списка литературы (А-Я, A-Z).

% Оставить обратную связь, благодарности предложения:
% Google форма

% Внести изменение в шаблон для всех:
% pull-request

% Обсуждения по запуску шаблона, см. кандидатские и докторские диссертации

% Обсуждения по совершентствованию шаблона ВКР:
% gitter-канал



%% Список использованных источников
% текущая реализация - быстрое приближение к требованиям

%1) в действующем варианте env=SSTfirst необходимо выполнить точное выравнивание по абзацному отступу. Сейчас оформление :
%1.1) единиц 1-9 немного выходят за рамки отступа
%1.2) десяток 10-99 немного не добирает до абзацного отступа
%1.3) если будут сотни, то проблема усугубится
% Скороее всего, необходимо сделать выравнивание по левому краю 

%2) необходимо реализовать второй вариант вывода библиографии




%% Экспорт - импорт данных
% 
% 1) Формирование файла renames.tex на основе данных из личного кабинета 
% 2) Экспорт мета-данных на vkr.spbstu.ru



%% Создание сопутствующих файлов
%	\item Файлы \verb|task.pdf|(\verb|.tex|) --- задание;
%	\item Файлы \verb|annotation.pdf|(\verb|.tex|) --- аннотация;
%	\item Файлы \verb|slides.pdf|(\verb|.tex|) --- слайды;
%	\item Файлы \verb|poster.pdf|(\verb|.tex|) --- постер;
%	\item Файлы \verb|advisor_review.pdf|(\verb|.tex|) --- отзыв;
%	\item Файлы \verb|external_review.pdf|(\verb|.tex|) --- рецензия;
%%
%%%% Preamble end %%%%  % лучше не редактировать / please, keep unmodified

\setcounter{docType}{2} % лучше не редактировать / please, keep unmodified

%%%% Настройки автора / Author settings
%% 
%%%% Настройки автора 
%% 
%% 	 Пожалуйста, ознакомьтесь с функционалом шаблона из [1,2], а также с пакетами, подключенными в ch_preamble.
%% 
%%   Новым командам лучше присваивать уникальные имена.
%% 
%%%% Author settings
%% 
%%   Please, see all possible packages using the search in files of ch_preamble. 
%%   
%%   Please, for user-defined commands write only unique command titles.
%%


%%%% Подключение библиографии / Upload bibliography
%% 
%% 
\addbibresource{my_folder/my_biblio.bib} % 



%%%% Полезные настройки / Usefull settings
%% 
%% Раскомментируйте, чтобы
%%
%% pdf при открытии выравнивался по окну
%% pdf fit screen window
\hypersetup{
pdfstartview={FitBH}
}
%% перенумеровать все строки pdf
%% enumerate all lines in pdf 
%\usepackage{lineno}
%\linenumbers
%%
%% установить дату после названия ВКР - расскоментируйте код в title.tex
%% set data after the thesis title - uncomment code in title.tex
\let\ordinal\relax %avoid extra warning
\usepackage{datetime}



%% In case of deleting the following info, please, delete the examples in the chapter body.

%% В случае комментирования (удаления) следующего кода могут появиться ошибки при компиляции примеров, т.е. необходимо будет удалить и примеры в теле главы.

\newcommand{\overbar}[1]{\mkern 1.5mu\overline{\mkern-1.5mu#1\mkern-1.5mu}\mkern 1.5mu}

%http://tex.stackexchange.com/questions/16645/blackboard-italic-font
% for itallic sign of context K to be a parametr
\DeclareMathAlphabet{\mathbbmsl}{U}{bbm}{m}{sl}
\newcommand{\cont}[1][K]{\ensuremath{\mathbbmsl{#1}}}

%%ARROWS

%mu = math unit = 1em
%\mkern-18mu
%"minus quad"

%https://tex.stackexchange.com/a/389805/44348
\newcommand{\fcaarrow}[1]{%
	{}^{\scriptscriptstyle\bm{#1}}
}
%%%%%%%%%%%%%%%%%%%%%%% ARROWS from Formal Concept Analysis
% small and bold \uparrow
\newcommand{\uA}{\fcaarrow{{\uparrow\mkern-12mu}}}
% small and bold \downarrow
\newcommand{\dA}{\fcaarrow{\downarrow\mkern-2mu}}
% small and bold \uparrow+\downarrow
\newcommand{\ud}{\fcaarrow{\uparrow\mkern-12mu}\fcaarrow{\downarrow\mkern-2mu}}
% small and bold \downarrow+\uparrow
\newcommand{\du}{\fcaarrow{\downarrow\mkern-2mu}\fcaarrow{\uparrow\mkern-12mu}}


%http://tex.stackexchange.com/questions/74125/how-do-i-put-text-over-symbols
\newcommand\eqdef{\mathrel{\overset{\makebox[0pt]{\mbox{\normalfont\tiny def}}}{=}}} %\sffamily



%%% Правила задания нового окружения

\theoremstyle{myplain} % первая команда для ввода доказательств
\newtheorem{m-new-env-first}{Название\_окружения}[chapter] 
% вместо m-new-env-first необходимо подставить название нового окружения;
% вместо Название\_окружения необходимо подставить название окружения, выводящееся в pdf;
% последний параметр обеспечивает нумерацию в пределах главы не меняется


\theoremstyle{mydefinition} % первая команда для ввода окружений, не связанных с доказательствами
\newtheorem{m-new-env-second}{Название\_окружения}[chapter] 
% вместо m-new-env-second необходимо подставить название нового окружения;
% вместо Название\_окружения необходимо подставить название окружения, выводящееся в pdf;
% последний параметр обеспечивает нумерацию в пределах главы не меняется % добавляем свои команды / update your commands


\begin{document} % начало документа
	


%%% Внесите свои данные - Input your data
%%
%%
\newcommand{\Author}{И.О.\,Фамилия} % И.О. Фамилия автора 
\newcommand{\AuthorFull}{Фамилия Имя Отчество} % Фамилия Имя Отчество автора
\newcommand{\AuthorFullDat}{Фамилия Имя Отчество} % Фамилия Имя Отчество автора в дательном падеже (Кому?)
\newcommand{\AuthorPhone}{+7-9XX-XXX-XX-XX} % номер телефорна автора для оперативной связи  
\newcommand{\Supervisor}{И.О.\,Фамилия} % И. О. Фамилия научного руководителя
\newcommand{\SupervisorJob}{должность} %
\newcommand{\SupervisorDegree}{степень} %
\newcommand{\SupervisorTitle}{звание} % 
%%
%%
%Руководитель, утверждающий задание
\newcommand{\Head}{И.О.\,Фамилия} % И. О. Фамилия руководителя подразделения (руководителя ОП)
\newcommand{\HeadDegree}{Должность руководителя}% Только должность:   
%Руководитель %ОП 
%Заведующий % кафедрой
%Директор % Высшей школы
%Зам. директора
\newcommand{\HeadDep}{M} % заменить на краткую аббревиатуру подразделения или оставить пустым, если утверждает руководитель ОП

%%% Руководитель, принимающий заявление
\newcommand{\HeadAp}{И.О.\,Фамилия} % И. О. Фамилия руководителя подразделения (руководителя ОП)
\newcommand{\HeadApDegree}{Должность руководителя}% Только должность:   
%Руководитель ОП 
%Заведующий кафедрой
%Директор Высшей школы
\newcommand{\HeadApDep}{O} % заменить на краткую аббревиатуру подразделения или оставить пустым, если утверждает руководитель ОП
%%% Консультант по нормоконтролю
\newcommand{\ConsultantNorm}{И.О.\,Фамилия} % И. О. Фамилия консультанта по нормоконтролю. ТОЛЬКО из числа ППС!
\newcommand{\ConsultantNormDegree}{должность, степень} %   
\newcommand{\ConsultantExtra}{И.О.\,Фамилия} % И. О. Фамилия дополнительного консультанта 
\newcommand{\ConsultantExtraDegree}{должность, степень} % 
\newcommand{\Reviewer}{И.О.\,Фамилия} % И. О. Фамилия резензента. Обязателен только для магистров.
\newcommand{\ReviewerDegree}{должность, степень} % 
%%
%%
\renewcommand{\thesisTitle}{Тема выпускной квалификационной работы}
%\newcommand{\thesisDegree}{бакалавра}% магистра или специалиста% 
\newcommand{\thesisDegree}{работа бакалавра}% дипломный проект, дипломная работа, магистерская диссертация %c 2020
\newcommand{\thesisTitleEn}{Title of the thesis} %2020
\newcommand{\thesisDeadline}{дд.мм.202X}
\newcommand{\thesisStartDate}{дд.мм.202X}
\newcommand{\thesisYear}{202X}
%%
%%
\newcommand{\group}{N} % заменить вместо N номер группы
\newcommand{\thesisSpecialtyCode}{ХХ.ХХ.ХХ}% код направления подготовки
\newcommand{\thesisSpecialtyTitle}{Наименование направления/специальности} % наименование направления/специальности
\newcommand{\thesisOPPostfix}{YY} % последние цифры кода образовательной программы (после <<_>>)
\newcommand{\thesisOPTitle}{Наименование образовательной программы}% наименование образовательной программы
%%
%%
\newcommand{\institute}{
Название института
%Институт компьютерных наук и технологий
%Гуманитарный институт
%Инженерно-строительный институт
%Институт биомедицинских систем и технологий
%Институт металлургии, машиностроения и транспорта
%Институт передовых производственных технологий
%Институт прикладной математики и механики
%Институт физики, нанотехнологий и телекоммуникаций
%Институт физической культуры, спорта и туризма
%Институт энергетики и транспортных систем
%Институт промышленного менеджмента, экономики и торговли
}%
%%
%%




%%% Задание ключевых слов и аннотации
%%
%%
%% Ключевых слов от 3 до 5 слов или словосочетаний в именительном падеже именительном падеже множественного числа (или в единственном числе, если нет другой формы) по правилам русского языка!!!
%%
%%
\newcommand{\keywordsRu}{Стилевое оформление сайта, управление контентом, php, MySQL, архитектура системы} % ВВЕДИТЕ ключевые слова по-русски
%%
%%
\newcommand{\keywordsEn}{Style registration, content management, php, MySQL, system architecture} % ВВЕДИТЕ ключевые слова по-английски
%%
%%
%% Реферат не более 600 знаков на русский или английский текст
\newcommand{\abstractRu}{В данной работе изложена сущность подхода к созданию динамического информационного портала на основе использования открытых технологий Apache, MySQL и PHP. Даны общие понятия и классификация IT-систем такого класса. Проведен анализ систем-прототипов. Изучена технология создания указанного класса информационных систем. Разработана конкретная программная реализация динамического информационного портала на примере портала выбранной тематики.} % ВВЕДИТЕ текст аннотации по-русски
%%
%%
\newcommand{\abstractEn}{In the given work the essence of the approach to creation of a dynamic information portal on the basis of use of open technologies Apache, MySQL and PHP is stated. The general concepts and classification of IT-systems of such class are given. The analysis of systems-prototypes is lead. The technology of creation of the specified class of information systems is investigated. Concrete program realization of a dynamic information portal on an example of a portal of the chosen subjects is developed.} % ВВЕДИТЕ текст аннотации по-английски




%%% Не меняем дальнейшую часть - Do not modify the rest part
%%
%%
%%
%%
\newcommand{\HeadTitle}{\HeadDegree~\HeadDep}
\newcommand{\HeadApTitle}{\HeadApDegree~\HeadApDep}
\newcommand{\thesisOPCode}{\thesisSpecialtyCode\_\thesisOPPostfix}% код образовательной программы
\newcommand{\thesisSpecialtyCodeAndTitle}{\thesisSpecialtyCode~\thesisSpecialtyTitle}% Код и наименование направления/специальности
\newcommand{\thesisOPCodeAndTitle}{\thesisOPCode~\thesisOPTitle} % код и наименование образовательной программы
%%
%%
\hypersetup{%часть болка hypesetup в style
		pdftitle={\thesisTitle},    % Заголовок pdf-файла
		pdfauthor={\AuthorFull},    % Автор
		pdfsubject={Выпускная квалификационная работа \thesisDegree. Шифр и наименование направления подготовки: \thesisSpecialtyCodeAndTitle. \abstractRu},      % Тема
		pdfcreator={LaTeX, SPbPU-student-thesis-template},     % Приложение-создатель
%		pdfproducer={},  % Производитель, Производитель PDF % будет выставлена автоматически
		pdfkeywords={\keywordsRu}
}
%%
%%
%% вспомогательные команды
\newcommand{\firef}[1]{рис.\ref{#1}} %figure reference
\newcommand{\taref}[1]{табл.\ref{#1}}	%table reference
%%
%%
%% Архивный вариант задания ключевых слов, аннотации и благодарностей 
% Too hard to export data from the environment to pdf-info
% https://tex.stackexchange.com/questions/184503/collecting-contents-of-environment-and-store-them-for-later-retrieval
%заменить NewEnviron на newenvironment для распознавания команды в TexStudio
%\NewEnviron{keywordsRu}{\noindent\MakeUppercase{\BODY}}
%\NewEnviron{keywordsEn}{\noindent\MakeUppercase{\BODY}}
%\newenvironment{abstractRu}{}{}
%\newenvironment{abstractEn}{}{}
%\newenvironment{acknowledgementsRu}{\par{\normalfont \acknowledgements.}}{}
%\newenvironment{acknowledgementsEn}{\par{\normalfont \acknowledgementsENG.}}{}


%%% Переопределение именований %%% Не меняем - Do not modify
%\newcommand{\Ministry}{Минобрнауки России} 
\newcommand{\Ministry}{Министерство науки и высшего образования Российской~Федерации} %с 2020
\newcommand{\SPbPU}{Санкт-Петербургский политехнический университет Петра~Великого}
%% Пробел между И. О. не допускается.
\renewcommand{\alsoname}{см. также}
\renewcommand{\seename}{см.}
\renewcommand{\headtoname}{вх.}
\renewcommand{\ccname}{исх.}
\renewcommand{\enclname}{вкл.}
\renewcommand{\pagename}{Pages}
\renewcommand{\partname}{Часть}
\renewcommand{\abstractname}{\textbf{Аннотация}}
\newcommand{\abstractnameENG}{\textbf{Annotation}}
\newcommand{\keywords}{\textbf{Ключевые слова}}
\newcommand{\keywordsENG}{\textbf{Keywords}}
\newcommand{\acknowledgements}{\textbf{Благодарности}}
\newcommand{\acknowledgementsENG}{\textbf{Acknowledgements}}
\renewcommand{\contentsname}{Content} % 
%\renewcommand{\contentsname}{Содержание} % (ГОСТ Р 7.0.11-2011, 4)
%\renewcommand{\contentsname}{Оглавление} % (ГОСТ Р 7.0.11-2011, 4)
\renewcommand{\figurename}{Рис.} % Стиль СПбПУ
%\renewcommand{\figurename}{Рисунок} % (ГОСТ Р 7.0.11-2011, 5.3.9)
\renewcommand{\tablename}{Таблица} % (ГОСТ Р 7.0.11-2011, 5.3.10)
%\renewcommand{\indexname}{Предметный указатель}
\renewcommand{\listfigurename}{Список рисунков}
\renewcommand{\listtablename}{Список таблиц}
\renewcommand{\refname}{\fullbibtitle}
\renewcommand{\bibname}{\fullbibtitle}

\newcommand{\chapterEnTitle}{Сhapter title} % <- input the English title here (only once!) 
\newcommand{\chapterRuTitle}{Название главы}          % <- введите 
\newcommand{\sectionEnTitle}{Section title} %<- input subparagraph title in english
\newcommand{\sectionRuTitle}{Название подраздела} % <- введите название подраздела по-русски
\newcommand{\subsectionEnTitle}{Subsection title} % - input subsection title in english
\newcommand{\subsectionRuTitle}{Название параграфа} % <- введите название параграфа по-русски
\newcommand{\subsubsectionEnTitle}{Subsubsection title} % <- input subparagraph title in english
\newcommand{\subsubsectionRuTitle}{Название подпараграфа} % <- введите название подпараграфа по-русски % Заполнить сведения, 
										 % в т.ч. ключевые слова и аннотацию.

%%% Титульник отчета по практике / Practice report title 
%%
%% добавить лист в pdf-навигацию 
%% add to pdf navigation menu
%%
\pdfbookmark[-1]{\pdfTitle}{tit}
%%
\thispagestyle{empty}%
\makeatletter
\newgeometry{top=2cm,bottom=2cm,left=3cm,right=1cm,headsep=0cm,footskip=0cm}
\savegeometry{NoFoot}%
\makeatother

%%% Распечатать версию документа / Print document version
%%
\begin{flushright}
	%	\vspace{0pt plus0.1fill}
	\boxed{\small
		\begin{tabular}{r} 
			\textbf{Пример отчета по практике <<SPbPU-student-thesis-template>>.} %\\ % перенос на новую строку
			\textbf{Версия от \today % \; время:  \currenttime. % время версии
			}
		\end{tabular}
	} %end boxed
	%	\vspace*{-5pt} % раскомментировать, если не хватает места
	\vspace{0pt plus0.1fill} % раскоментировать, если хватает места
\end{flushright}






% TODO Exact match of font size
%{\centering%
%	\footnotesize
%	\MakeUppercase{Федеральное государственное автономное образовательное учреждение\\высшего образования}\\%
%		{\bfseries <<\SPbPU>>\\%
%			\institute\\%
%			\School}
%}

{\centering%
	\small%
	\MakeUppercase{\SPbPUOfficialPrefix}\\
	{\bfseries %2020 - указание на изменения, которые могут быть введены в 2020 году
	<<\MakeUppercase{\SPbPU}>>\\%
	\MakeUppercase{\institute}\\
	\MakeUppercase{\School}
	}
\par}%
	

\vspace{0pt plus1fill} %число перед fill = кратность относительно некоторого расстояния fill, кусками которого заполнены пустые места


\noindent


%\vspace{0pt plus2fill} %


{\centering%
	{\bfseries{} 
	\DocType\\
	на тему: <<\practiceTitle>>}\\

\intervalS\normalfont%

	\uline{\AuthorFullVin , гр. \group}%

\intervalS\normalfont%

\par}%

%\intervalS% %ОБЯЗАТЕЛЬНО ДОБАВИТЬ ОТСТУП, ЕСЛИ ХВАТАЕТ МЕСТА


{\noindent {\bfseries Направление подготовки:} {\expandafter \ulined \thesisSpecialtyCode~\thesisSpecialtyTitle}.}\par


{\noindent {\bfseries Место прохождения практики:} {\expandafter \ulined \Workplace}.} % включая фактический адрес для практики в сторонней организации по договору


{\noindent {\bfseries Сроки практики:} \uline{с \PracticeStartDate~по \PracticeEndDate.}}\par


{\noindent {\bfseries Руководитель практики от \SPbPUOfficialShort:}} {\expandafter \ulined \SupervisorFull, \SupervisorJob, \SupervisorDegree.} %Ф.И.О., должность, степень



{\noindent \bfseries 
	Консультант практики от \SPbPUOfficialShort:
%	Консультанты практики от \SPbPUOfficialShort:
} 
{\noindent \expandafter \ulined \ConsultantExtraFull, \ConsultantExtraDegree}.%,      %% первый консультант Ф.И.О., должность, степень
%{\noindent \expandafter \ulined \ConsultantExtraTwoFull, \ConsultantExtraTwoDegree.}  %% второй консультант Ф.И.О., должность, степень 

{\noindent {\bfseries Оценка:} \uline{\hspace*{0.1\textheight}} % НЕ ЗАПОЛНЯЕМ!

\vspace{0pt plus1fill}%

\noindent
\begin{tabularx}{\linewidth}{lXl}
	Руководитель практики		&	&\\
	от \SPbPUOfficialShort		&	& \Supervisor     \\[\mfloatsep] % если не хватает места, закомментировать

	Консультант практики		&	&\\
	от \SPbPUOfficialShort		&	& \ConsultantExtra\\[\mfloatsep]
%							&	& \ConsultantExtraTwo\\[\mfloatsep]
	
	Обучающийся				&	&\Author\\[\mfloatsep]
	Дата: \uline{\PracticeEndDate}		&	&
\end{tabularx} %


%
\vspace{0pt plus4fill}% 

\restoregeometry
\newpage				 % Титульный лист
										 % Убираем footnotes, консультанта, если нет

%%% Не мянять - Do not modify !
%%
%%
%% Оглавление (ГОСТ Р 7.0.11-2011, 5.2)
%\ifdefmacro{\microtypesetup}{\microtypesetup{protrusion=false}}{} % не рекомендуется применять пакет микротипографики к автоматически генерируемому оглавлению
%\tableofcontents*
%\addtocontents{ptc}{\protect\tocheader}
%\endTOCtrue
%\ifdefmacro{\microtypesetup}{\microtypesetup{protrusion=true}}{}
%https://tex.stackexchange.com/questions/170912/contents-page-in-two-different-languages


\setlength{\parskip}{0.35\onelineskip} % интервал между элементов - полуторный
\begin{Spacing}{\Single} %интервал внутри элемента - одинарный
\tableofcontents
 \end{Spacing}
\setlength{\parskip}{0pt} % интервал между элементов - полуторный
\OnehalfSpacing*    % Полуторный интервал % * to force it in the floats
\newpage  	         % Оглавление


\chapter*{Введение} % * не проставляет номер
\addcontentsline{toc}{chapter}{Введение} % вносим в содержание



Данный пример выпускной квалификационной работы (далее --- ВКР) создан для того, чтобы продемонстрировать возможности шаблонов SPbPU-student-templates, выполненных с помощью издательской системы \LaTeX{} \cite{spbpu-student-thesis-template}. В примере отображены некоторые обязательные элементы ВКР \cite{spbpu-student-thesis-specification}. Для того, чтобы подробнее ознакомиться с требованиями к наполнению этих элементов, а также с общими требованиями к структуре и оформлению ВКР, пожалуйста, ознакомьтесь с  \cite{spbpu-student-thesis-template-author-guide,spbpu-student-thesis-specification}. 

Пример может быть использован для подготовки отчета по практике с учетом корректировки требований к титульному листу, структуре и содержанию конкретной практики в соответствии с планом её проведения (рабочей программы дисциплины).


Технология написания ВКР на \LaTeX{} подробно изложена в \cite{spbpu-student-thesis-template-author-guide}. В рекомендациях приведены ссылки на учебно-справочные материалы \LaTeX{} (под \LaTeX{} в документе может подразумеваться также \TeX, \LaTeXe).


Авторам, использующим \LaTeX{}, необходимо последовательно заменять текст данного шаблона в файлах <<\verb|thesis.tex|>> на текст своей ВКР, избегая при этом ошибок (errors) при компиляции. Синтаксические конструкции \LaTeX, которые задействованы в формировании того или иного текста выделены \texttt{машинописным шрифтом}. Иные шрифты в тексте ВКР (за исключением математических) использовать запрещено. 

Светлым курсивом выделены \textit{важные} элементы текста (ключевые слова определений, интонационные выделения словосочетаний), полужирным шрифтом --- \textbf{служебные} элементы текста (<<определение>>, <<теорема>>, <<лемма>> и т.п), а также при необходимости ключевые слова в алгоритмах. В соотношении к основному тексту курсив и полужирный шрифт не может превышать 1 \% текста на странице.
 
Полужирный курсив разрешено использовать только в \textbf{\textit{названиях подпараграфов (пунктов)}} и запрещёно использовать в основном тексте. 
\uline{Подчеркивание} допускается использовать только в задании в местах, где данные вписываются студентом, а также в математических формулах при необходимости.

Введение \textit{не должно превышать 4 страницы}. Во введении необходимо обосновать выбор темы, охарактеризовать современное состояние изучаемой проблемы (оба вопроса помещаем в актуальность), ее актуальность, практическую и теоретическую значимость, степень разработанности данной проблемы (последние вопросы см. ближе к концу текста).


\textbf{Aктуальность исследования} заключается в N фактах и явлениях,  а также в их состоянии, связанных с ними нерешенных проблемах, слабо освещенных и требующих уточнения или дальнейшей разработки вопросов. Исследуемая проблема решена не полностью/ не решалась ранее / не имеет аналогов с отрытым исходным кодом и т.п. 


\textbf{Объект исследования} --- это то, на что направлен процесс познания (индивид, коллектив, общность людей, сфера деятельности и т.п.). Связь объекта и предмета легко запоминаются по формуле: <<исследуем такой-то объект на предмет чего-то>>. Это процесс или явление, порождающее проблемную ситуацию, и избран-ное для изучения в целом. Всегда в объекте содержится предмет, а не наоборот. 

\textbf{Предмет исследования} --- один из аспектов, часть рассматриваемого объекта (свой-ства, состояния, процессы, направления и особенности деятельности структур по связям с общественностью, их сотрудников в конкретных сферах общественных отношений и т.д.). Предмет исследования частично совпадает с названием работы и содержится в цели сразу после сказуемого (<<выявить \ldots что?>>, <<определить \ldots что?>>, <<сформировать \ldots что?>>). Именно предмет исследования определяет тему выпускной квалификационной работы.
Объект и предмет исследования соотносятся между собой как целое и частное, общее и частности. 


\textit{Цель исследования} формулируется, исходя из проблемы, которую следует разрешить студенту в процессе выполнения выпускной квалификационной работы и представляет собой в самом сжатом виде тот результат (результаты), который должен быть получен в итоге исследования. Формулировку цели рекомендуется начинать со слов: <<сформировать/создать>>, <<разработать>>, <<провести>>, <<подготовить>>.

\textbf{Цель исследования} --- краткий ожидаемый результат, то есть решение практических задач и новые знания о рассматриваемом предмете исследования. 
В соответствии с целью исследования, логически определяются следующие \textbf{задачи работы} (должно быть \textit{не менее четырех задач, но не более шести задач}):

\begin{enumerate}
	\item Первая задача.
	\item Вторая задача.
	\item Третья задача.
	\item Четвертая задача.
\end{enumerate} 


Задачи отражают \textit{поэтапное достижение цели, при этом уточняют границы проводимого исследования}.
Рекомендуется формулировать задачи с глаголов в форме перечисления: <<изучить \ldots>>, <<выявить \ldots>>, <<проанализировать \ldots>>, <<разработать \ldots>>, <<описать \ldots>> и т.п. Заголовки выпускной квалификационной работы должны отражать суть поставленной задачи.


Общая направленность исследования задается до его начала сформулированными \textbf{гипотезами}, которыми могут быть:
\begin{itemize}
	\item научное предположение, выдвигаемое для объяснения каких-либо факторов, явлений и процессов, которые надо подтвердить или опровергнуть (т. е. требующее верификации);
	\item вероятностное знание, научно обоснованная догадка по объяснению действительности;
	\item прогноз ожидаемого решения проблемы, ответ на вопрос, поставленный в задаче;
	\item условно-категорическое умозаключение по схеме <<если \ldots, то \ldots>>, основными элементами которого являются условие (причина) и результат (следствие).
\end{itemize}  

Гипотеза --- это предполагаемое решение проблемы. В ходе исследования гипотезу проверяют и либо подтверждают, либо опровергают. Формулировка гипотезы \textit{обязательна только для магистров}.


\textbf{Теоретическая и методологическая база исследования}. В теоретической базе необходимо перечислить источники, которые использовались для написания работы. Приведём примеры ключевых фраз: 
\begin{itemize}
	\item <<Теоретической основой выпускной квалификационной работы послужили исследования  \ldots (перечисляются конкретные документы)>>.
	\item <<Практическая часть работы выполнялась на основании документов  \ldots>>.
	\item  <<При написании выпускной квалификационной работы использовалась работы отечественных и зарубежных специалистов \ldots>>.
	\item  <<Для выполнения анализа в практической части были использованы материалы  \ldots>>.
	\item <<При подготовке ВКР были использованы материалы таких учебных дисциплин, как "Технология конструкционных материалов'', "Экономика'' "Начертательная геометрия'' \ldots>>.
	\item <<При выполнении ВКР использовались материалы N организации \ldots (ссылка на официальный сайт)>>.
\end{itemize}


%% Обязательна 2025
\textbf{Методологическая база исследования} должна содержать указание на методы и подходы, на которых основывается данная ВКР. 

Среди методов исследования студенту необходимо обратить внимание на общенаучные методы, включающие эмпирические (наблюдение, эксперимент, сравнение, описание, измерение), теоретические (формализация, аксиоматический, гипотетико-дедуктивный, восхождение от абстрактного к конкретному) и общелогические (анализ, абстрагирование, обобщение, идеализация, индукция, аналогия, моделирование и др.) методы.
Также следует назвать конкретно-научные (частные) методы научного познания, представляющие собой специфические методы конкретных наук: экономики, социологии, психологии, истории, логики и проч.


%% НЕ ЯВЛЯЕТСЯ ОБЯЗАТЕЛЬНОЙ ДЛЯ БАКАЛАВРОВ, ДЛЯ МАГИСТОВ СОГЛАСОВЫВАЕМ С НАУЧНЫМ РУКОВОДИТЕЛЕМ
\textbf{Информационной базой} для разработки ВКР служат материалы, собранные студентом в процессе обучения в ВУЗе, в ходе прохождения учебной и производственной практик, а также во время прохождения преддипломной практики.
Дополнительная информационная база может включать информацию официальных статистических публикаций (например, Госкомстата России), материалы, получаемые из Интернета, информацию международных организаций и ассоциаций. 

%% ЯВЛЯЕТСЯ ОБЯЗАТЕЛЬНОЙ 2025
\textbf{Степень научной разработанности проблемы} --- это состояние теоретической разработанности проблемы, анализ работ отечественных и зарубежных авторов, исследующих эту проблему. Здесь важно подчеркнуть исторические, экономические, политические или профессиональные явления, повлиявшие на выбор темы. Также в данной части введения проводится критический обзор современного состояния и освещения исследуемой темы в научной, профессиональной литературе и СМИ, обобщаются и оцениваются точки зрения различных авторов по теме исследования. 


%% НЕ ЯВЛЯЕТСЯ ОБЯЗАТЕЛЬНОЙ ДЛЯ БАКАЛАВРОВ, ДЛЯ МАГИСТОВ СОГЛАСОВЫВАЕМ С НАУЧНЫМ РУКОВОДИТЕЛЕМ
\textbf{Научная новизна} выявляется в результате анализа литературных источников, уточнения концептуальных положений, обобщения опыта решения подобных проблем. Это принципиально новое знание, полученное в науке в ходе проведенного исследования (теоретические положения, впервые сформулированные и обоснованные, собственные методические рекомендации, которые можно использовать в практике).
Научная новизна выпускной квалификационной работы может состоять: 
\begin{itemize}
	\item в изучении фактов и явлений с помощью специальных научных методов и междисциплинарных подходов;
    \item в изучении уже известного в науке явления на новом экспериментальном материале;
	\item в переходе от качественного описания известных в науке фактов к их точно определяемой количественной характеристике;
	\item в изучении известных в науке явлений и процессов более совершенными методами;
	\item в сопоставлении, сравнительном анализе протекания процессов и явлений;
	\item в изменении условий протекания изучаемых процессов;
	\item в уточнении категориального аппарата дисциплины, определение типологии, признаков, специфики изучаемого явления.
\end{itemize}


%% Обязательна 2025
\textbf{Практическая значимость} работы подробно отражается в:
\begin{itemize}
	\item практических рекомендациях или разработанном автором выпускной квалификационной работы проекте (как основная часть выпускной квалификационной работы);
	\item выявлении важности решения избранной проблемы для будущей деятельности магистра по выбранному направлению подготовки.
\end{itemize} 

Практическая значимость выпускной квалификационной работы может заключаться в возможности:

\begin{itemize}
	\item решения той или иной практической задачи в сфере профессиональной деятельности;
	\item проведения дальнейших научных исследований по теме ВКР;
	\item разработки конкретного проекта, направленного на интенсификацию работы исследуемой организации, предприятия.
\end{itemize}

\textbf{Апробация результатов} исследования включает:
\begin{itemize}
	\item участие в конференции, семинарах и т. д.;
	\item публикации по теме выпускной квалификационной работы;
	\item применение результатов исследования в практической области;
	\item разработку и внедрение конкретного проекта;
	\item выступления на научных конференциях, симпозиумах, форумах и т.п. (\textit{обязательно});
	\item публикации студента, включенные в список использованных источников. 
\end{itemize}

%% Обязательна 2025
\textbf{Теоретическая значимость} работы отражается, прежде всего, в:
\begin{itemize}
	\item исследовании и проверке выдвинутых гипотез;
	\item развитии технологий/методов/методик разработки программного обеспечения \texttt{указать\_какого} класса и т.п.
\end{itemize} 


В силу ограниченности объема необходимо очень тщательно подойти к написанию введения, которое должно стать <<визитной карточкой>>, кратко, но емко характеризующей работу. Во введение не включают схемы, таблицы, описания, рекомендации и т.п. 

Целью первой главы, как правило, является всесторонний анализ предмета и объекта исследования, второй --- разработка предложений (алгоритмов, технологий и т.п.) по улучшению какого-либо процесса, протекающих с участием предмета и объекта исследования, третьей --- практическая реализация (имплементация) --- предложений (алгоритмов, технологий и т.п.) в виде программного (или иного) продукта, четвертой --- апробация разработанных в работе предложений и выводы целесообразности их дальнейшей разработки (использованию). 
Содержание глав в данном шаблоне приведено только для демонстрации возможностей \LaTeX.


%% Вспомогательные команды - Additional commands
%\newpage % принудительное начало с новой страницы, использовать только в конце раздела
%\clearpage % осуществляется пакетом <<placeins>> в пределах секций
%\newpage\leavevmode\thispagestyle{empty}\newpage % 100 % начало новой строки	    	 % Введение

%% Начало основной части
\chapter{Название первой главы: всестороннее изучение объекта и предмета исследования, анализ результатов, полученных другими авторами} \label{ch1}

% не рекомендуется использовать отдельную section <<введение>> после лета 2020 года
%\section{Введение. Сложносоставное название первого параграфа первой главы для~демонстрации переноса слов в содержании} \label{ch1:intro}

Хорошим стилем является наличие введения к главе, которое \textit{начинается непосредственно после названия главы, без оформления в виде отдельного параграфа}. Во введении может быть описана цель написания главы, а также приведена краткая структура главы. Например, в параграфе \ref{ch1:sec1} приведены примеры оформления одиночных формул, рисунков и таблицы. Параграф \ref{ch1:sec2} посвящён многострочным формулам и сложносоставным рисункам.

Текст данной главы призван привести \textit{краткие} примеры оформления текстово-графических объектов. Более подробные примеры можно посмотреть в следующей главе, а также в рекомендациях студентам \cite{spbpu-student-thesis-template-author-guide}. 


\section{Название параграфа} \label{ch1:sec1}


\subsection{Название первого подпараграфа первого параграфа первой главы для~демонстрации переноса слов в содержании} % ~ нужен, чтобы избавиться от висячего предлога (союза) в конце строки

Содержание первого подпараграфа первого параграфа первой главы.



Одиночные формулы оформляют в окружении \texttt{equation}, например, как указано в следующей одиночной нумерованной формуле:
%
%
\begin{equation}% лучше не оставлять пропущенную строку (\par) перед окружениями для избежания лишних отсупов в pdf
\label{eq:Pi-ch1} % eq - equations, далее название, ch поставлено для избежания дублирования
\pi \approx 3,141.
\end{equation}
%
%
\begin{figure}[ht!] 
	\center
	\includegraphics [scale=0.27] {my_folder/images//spbpu_hydrotower}
	\caption{Вид на гидробашню СПбПУ \cite{spbpu-gallery}} 
	\label{fig:spbpu_hydrotower}  
\end{figure}
%
%
%\begin{table} [htbp]% Пример оформления таблицы
%	\centering\small
%	\caption{Представление данных для сквозного примера по ВКР \cite{Peskov2004}}%
%	\label{tab:ToyCompare}		
%		\begin{tabular}{|l|l|l|l|l|l|}
%			\hline
%			$G$&$m_1$&$m_2$&$m_3$&$m_4$&$K$\\
%			\hline
%			$g_1$&0&1&1&0&1\\ \hline
%			$g_2$&1&2&0&1&1\\ \hline
%			$g_3$&0&1&0&1&1\\ \hline
%			$g_4$&1&2&1&0&2\\ \hline
%			$g_5$&1&1&0&1&2\\ \hline
%			$g_6$&1&1&1&2&2\\ \hline		
%		\end{tabular}	
%	\normalsize% возвращаем шрифт к нормальному
%\end{table}


% \firef{} от figure reference
% \taref{} от table reference
% \eqref{} от equation reference

На \firef{fig:spbpu_hydrotower} изображена гидробашня СПбПУ, а в \taref{tab:ToyCompare} приведены данные, на примере которых коротко и наглядно будет изложена суть ВКР.


\section{Название параграфа} \label{ch1:sec2} 



Формулы могут быть размещены в несколько строк. Чтобы выставить номер формулы напротив средней строки, используйте окружение \verb|multlined| из пакета \verb|mathtools| следующим образом \cite{Ganter1999}:
%
\begin{equation} 
\label{eq:fConcept-order-ch1}
\begin{multlined}
(A_1,B_1)\leq (A_2,B_2)\; \Leftrightarrow \\  \Leftrightarrow\; A_1\subseteq A_2\; \Leftrightarrow \\ \Leftrightarrow\; B_2\subseteq B_1. 
\end{multlined}
\end{equation}


Используя команду \verb|\labelcref| из пакета \verb|cleveref|, допустимо следующим образом оформлять ссылку на несколько формул:
(\labelcref{eq:Pi-ch1,eq:fConcept-order-ch1}).
%
%
На \firef{fig:spbpu_whitehall-three-photos} приведены три картинки под~общим номером и~названием, но с раздельной нумерацией подрисунков посредством пакета \verb|subcaption|.
%
\begin{figure}[!htbp]
	\adjustbox{minipage=1.3em,valign=t}{\subcaption{}\label{fig:spbpu_whitehall-a}}%
	\begin{subfigure}[t]{\dimexpr.3\linewidth-1.3em\relax}
		\centering
		\includegraphics[width=.95\linewidth,valign=t]{my_folder/images//spbpu_whitehall}
	\end{subfigure}
	\hfill %выровнять
	\adjustbox{minipage=1.3em,valign=t}{\subcaption{}\label{fig:spbpu_whitehall-b}}%
	\begin{subfigure}[t]{\dimexpr.3\linewidth-1.3em\relax}
		\centering
		\includegraphics[width=.95\linewidth,valign=t]{my_folder/images//spbpu_whitehall_ligh}
	\end{subfigure}
	\hfill %выровнять
		\adjustbox{minipage=1.3em,valign=t}{\subcaption{}\label{fig:spbpu_whitehall-c}}%
	\begin{subfigure}[t]{\dimexpr.3\linewidth-1.3em\relax}
		\centering
		\includegraphics[width=.95\linewidth,valign=t]{my_folder/images//spbpu_whitehall_sculpture}
	\end{subfigure}%
\captionsetup{justification=centering} %центрировать
	\caption{Фотографии Белого зала СПбПУ \cite{spbpu-gallery}, в том числе: {\itshape a} --- со стороны зрителей; {\itshape b} --- со стороны сцены; {\itshape c} --- барельеф}\label{fig:spbpu_whitehall-three-photos}  
\end{figure}

Далее можно ссылаться на три отдельных рисунка: \firef{fig:spbpu_whitehall-a}, \firef{fig:spbpu_whitehall-b} и \firef{fig:spbpu_whitehall-c}. % пример подключения 3х иллюстрации в одном рисунке

Пример ссылок \cite{Article,Book,Booklet,Conference,Inbook,Incollection,Manual,Mastersthesis,Misc,Phdthesis,Proceedings,Techreport,Unpublished,badiou:briefings}, а также ссылок с указанием страниц, на котором отображены номера страниц  \cite[с.~96]{Naidenova2017} или в виде мультицитаты на несколько источников \cites[с.~96]{Naidenova2017}[с.~46]{Ganter1999}. Часть библиографических записей носит иллюстративный характер и не имеет отношения к реальной литературе. 



%\FloatBarrier % заставить рисунки и другие подвижные (float) элементы остановиться

\section{Выводы} \label{ch1:conclusion}

Текст выводов по главе \thechapter.

Кроме названия параграфа <<выводы>> можно использовать (единообразно по всем главам) следующие подходы к именованию последних разделов с результатами по главам:
\begin{itemize}
	\item <<выводы по главе N>>, где N --- номер соответствующей главы;
	\item <<резюме>>;
	\item <<резюме по главе N>>, где N --- номер соответствующей главы.
\end{itemize}

Параграф с изложением выводов по главе \textit{является обязательным}.

%% Вспомогательные команды - Additional commands
%
%\newpage % принудительное начало с новой страницы, использовать только в конце раздела
%\clearpage % осуществляется пакетом <<placeins>> в пределах секций
%\newpage\leavevmode\thispagestyle{empty}\newpage % 100 % начало новой страницы	         	 % Глава 1
\ContinueChapterBegin % размещать главы <<подряд>> 
\chapter{Название второй главы: разработка метода, алгоритма, модели исследования} \label{ch2}
	
% не рекомендуется использовать отдельную section <<введение>> после лета 2020 года
%\section{Введение} \label{ch2:intro}

Глава посвящена более подробным примерам оформления текстово-графических объектов.

В параграфе \ref{ch2:title-abbr} приведены примеры оформления многострочной формулы и одиночного рисунка. Параграф \ref{ch2:sec-abbr} раскрывает правила оформления перечислений и псевдокода. В параграфе \ref{ch2:sec-very-short-title} приведены примеры оформления сложносоставных рисунков, длинных таблиц, а также теоремоподобных окружений.


\section{Название параграфа} \label{ch2:title-abbr} %название по-русски



%%%%
%%		
%%  \input{...} commands are used only to sychronize some parts of the text with the author guide. Authors are free to type the text directly in .tex-files   
%%  \input{...} комманды используются только, чтобы синхронизировать части текта с рекомендациями авторам. Авторы  вольны вносить текст непосредственно в файл главы  
%%  
 %% ВНИМАНИЕ: для того, чтобы избежать лишнего отступа между текстом  и формулами, пожалуйста, начинайте формулы без пропуска строки в исходном коде как в строках #2 и #3.
	Все формулы, размещенные в отдельных строках, подлежат нумерации, например, как формулы \eqref{eq:UpArrow-G} и \eqref{eq:DownArrow-G} из \cite{Ganter1999}. 
	\begin{align}
	\label{eq:UpArrow-G}
	& A\uA =  \{ m\in{}M\:|\:gIm\:\forall  g \in{} A \}; \\ 
	\label{eq:DownArrow-G}
	& B\dA =  \{ g\in{}G\:|\:gIm\:\forall  m \in{} B \}.
	\end{align}

Обратим внимание, что формулы содержат знаки препинания и что они выровнены по левому краю (с помощью знака \verb|&| окружения \texttt{align}). % пример двух выравнивания двух формул в окружении align


На \firef{fig:spbpu-new-bld-autumn-ch2} приведёна фотография Нового научно-исследовательского корпуса СПбПУ.

	\begin{figure}[ht] 
	\center
	\includegraphics [scale=0.27] {my_folder/images/spbpu_new_bld_autumn}
	\caption{Новый научно-исследовательский корпус СПбПУ \cite{spbpu-gallery}} 
	\label{fig:spbpu-new-bld-autumn-ch2}  
	\end{figure}
	


	
\section{Название параграфа} \label{ch2:sec-abbr} %название по-русски
	
Название параграфа оформляется с помощью команды \verb|\section{...}|, название главы --- \verb|\chapter{...}|. 
	

\subsection{Название подпараграфа} \label{ch2:subsec-title-abbr} %название по-русски


Название подпараграфа оформляется с помощью команды  \texttt{\textbackslash{}subsection\{...\}}.


%\subsubsection{Название подподпараграфа} \label{ch2:subsubsec-title-abbr} %название по-русски
	
Использование подподпараграфов в основной части крайне не рекомендуется. В случае использования, необходимо вынести данный номер в содержание.	
Название подпараграфа оформляется с помощью команды  \texttt{\textbackslash{}subsubsecti\-on\{...\}}.



Вместо подподпараграфов рекомендовано использовать перечисления.

Перечисления могут быть с нумерационной частью и без неё и использоваться с иерархией и без иерархии. Нумерационная часть при этом формируется следующим способом:

\begin{enumerate}[1.]
	\item в перечислениях {\itshape без иерархии} оформляется арабскими цифрами с точкой (или длинным тире).
	\item В перечислениях {\itshape с иерархией} --- в последовательности сначала прописных латинских букв с точкой, затем арабских цифр с точкой и далее --- строчных латинских букв со скобкой. 
\end{enumerate}


%% Если в дальнейшем нужно сделать сслыку на один из элементов нумеруемого перечисления, то нужно использовать конктрукцию типа:

%\begin{enumerate}[label=\arabic{enumi}.,ref=\arabic{enumi}]
%	\item text 1 \label{item:text1}
%	\item text 2
%\end{enumerate}
%\ref{item:text1}.


Далее приведён пример перечислений с иерархией.


\begin{enumerate}
	\item Первый пункт.
	\item Второй пункт.
	\item Третий пункт.
	\item По ГОСТ 2.105--95 \cite{gost-russian-text-documents} первый уровень нумерации идёт буквами русского или латинского алфавитов ({\itshape для определенности выбираем английский алфавит}),
	а второй "--- цифрами. 
	\begin{enumerate}
		\item В данном пункте лежит следующий нумерованный список: 
		\begin{enumerate}
			\item первый пункт;
			\item третий уровень нумерации не нормирован ГОСТ 2.105--95 ({\itshape для определенности выбираем английский алфавит});
			\item обращаем внимание на строчность букв в этом нумерованном и следующем маркированном списке:
			\begin{itemize}
				\item первый пункт маркированного списка.
			\end{itemize}    
		\end{enumerate}
	\end{enumerate}
	\item Пятый пункт верхнего уровня перечисления.
\end{enumerate}

Маркированный список (без нумерационной части) используется, если нет необходимости ссылки на определенное положение в списке:
\begin{itemize}
	\item первый пункт c {\itshape маленькой буквы} по правилам русского языка;
	\item второй пункт c {\itshape маленькой буквы} по правилам русского языка.
\end{itemize} % правила использования перечислений	

	
Оформление псевдокода необходимо осуществлять с помощью пакета \verb|algorithm2e| в окружении \verb|algorithm|. Данное окружение интерпретируется в шаблоне как рисунок. Пример оформления псевдокода алгоритма приведён на \firef{alg:AlgoFDSCALING}. 
	
	
	\begin{algorithm} %[h]
		\SetKwFunction{algoDTestsFDSCALING}{} 
		\SetKwProg{myalg}{Algorithm}{}{} %write in 2nd agrument <<Algorithm>>, <<Procedure>> etc
		\nonl\myalg{\algoDTestsFDSCALING}{
			\KwInput{the many-valued context $\cont[M]\eqdef(G,M,W,J)$, the class membership $\epsilon: G\to K$} 
			\KwOutput{positive and negative binary contexts $\overbar{\cont[K]_+}\eqdef(\overbar{G_+},M,I_+)$, $\overbar{\cont[K]_-}\eqdef(\overbar{G_-},M,I_-)$ such that i-tests found in $\overbar{\cont[K]_+}$ are diagnostic tests in $\cont[M]$, and objects from $\overbar{\cont[K]_-}$ are counter-examples} %последние строки формируют начальное множество диагностических тестов
			\For {$\forall g_i,$ $g_j \in G$\label{step:FD-scaling-first-step}}{
				%(\tcp*[f]{possible inlined comment})
				\If{$i < j$ }{
					$\overbar{G} \leftarrow (g_i,g_j)$\;
				}
			}
			%		$M\leftarrow M\setminus k$\;
			\For {$\forall (g_i,g_j)\in \overbar{G}$}{
				%(\tcp*[f]{possible inlined comment})
				\If{$m(g_i) = m(g_j)$ }{ %на самом деле здесь цикл по всем компонентам вектора-строки
					$(g_i,g_j) I m$\; % or setI() function
				}
				\uIf{$\epsilon(g_i) = \epsilon(g_j)$ }{
					$\overbar{G_+} \leftarrow (g_i,g_j)$\;
				}
				\lElse{$\overbar{G_-} \leftarrow (g_i,g_j)$\label{FD-scaling-step-last}}	
			}		
			$I_+= I\cap (\overbar{G_+}\times M)$, $I_-= I\cap (\overbar{G_-}\times M)$\label{FD-scaling-step-newK}\; 
			\For {$\forall \overbar{g_+}\in \overbar{G_+}$, $\forall \overbar{g_-}\in \overbar{G_-}$ }{
				\If{$\overbar{g_+}\uA \subseteq \overbar{g_-}\uA$ }{
					$\overbar{G_+} \leftarrow \overbar{G_+} \setminus \overbar{g_+}$\;
				}
			}
			%		\Return \;
		}
		\caption{Псевдокод алгоритма \texttt{DiagnosticTestsScalingAndInferring} \cite{Naidenova2017}}\label{alg:AlgoFDSCALING}
		% example of adding an item to Index
		% \index for accepted papers only
		\index[ru]{алгоритм!\texttt{название\_алгоритма}} 
		% key words <<алгоритм>> и <<algorithm>> keep unmodified
		\index[en]{algorithm!\texttt{algorighm\_title}}
		% authors can used the key word <<процедура>> (procedure) и т.п.
		%
		%
	    % another example:
		\index[ru]{алгоритм!\texttt{DiagnosticTestsScaling\-AndInferring}} %нужен ручной перенос \- из-за ошибки в MakeIndex для команды \texttt
		%ключевые слова <<алгоритм>> и <<algorithm>> не менять
		\index[en]{algorithm!\texttt{DiagnosticTestsScaling\-AndInferring}} %нужен ручной перенос \- из-за ошибки в MakeIndex для команды \texttt
	\end{algorithm} 
	
	% another example of adding an arbitrary keyword to Index
	% some useful keywords: theorem, proposition, lemma, equation etc
	% please, use short keywords (2-3 max)
	\index[ru]{длинное-название-возможное-например-на-немецком} % длинные названия первого уровня как правило запрещены
	\index[en]{long-title-possible-for-example-in-German} 
	
Обратим внимание, что можно сослаться на строчку \ref{step:FD-scaling-first-step} псевдокода из \firef{alg:AlgoFDSCALING}.  % пример оформления псевдокода алгоритма 	

	
\section{Название параграфа} \label{ch2:sec-very-short-title} %название по-русски


	
%% ВНИМАНИЕ: для того, чтобы избежать лишнего отступа между текстом  и формулами, пожалуйста, начинайте формулы без пропуска строки в исходном коде как в строках #2 и #3.
Одиночные формулы также, как и отдельные формулы в составе группы, могут быть размещены в несколько строк. Чтобы выставить номер формулы напротив средней строки, используйте окружение \verb|multlined| из пакета \verb|mathtools| следующим образом \cite{Ganter1999}:
\begin{equation} % \tag{S} % tag - вписывает свой текст 
\label{eq:fConcept-order-G}
\begin{multlined}
(A_1,B_1)\leq (A_2,B_2)\; \Leftrightarrow \\  \Leftrightarrow\; A_1\subseteq A_2\; \Leftrightarrow \\ \Leftrightarrow\; B_2\subseteq B_1. 
\end{multlined}
\end{equation}

	
Используя команду \verb|\labelcref{...}| из пакета \verb|cleveref|, допустимо оформить ссылку на несколько формул, например, (\labelcref{eq:UpArrow-G,eq:DownArrow-G,eq:fConcept-order-G}). % пример оформления одиночной формулы в несколько строк

Пример оформления четырёх иллюстраций в одном текстово-графическом объекте приведён на \firef{fig:spbpu_sc-four-photos}. Это возможно благодаря использованию пакета \verb|subcaption|.

\begin{figure}[ht]
	\adjustbox{minipage=1.3em,valign=t}{\subcaption{}\label{fig:spbpu_sc-a}}%
	\begin{subfigure}[t]{\dimexpr.5\linewidth-1.3em\relax}
		\centering
		\includegraphics[width=.95\linewidth,valign=t]{my_folder/images/spbpu_sc_system}
	\end{subfigure}
\hfill %выровнять по ширине
	\adjustbox{minipage=1.3em,valign=t}{\subcaption{}\label{fig:spbpu_sc-b}}%
	\begin{subfigure}[t]{\dimexpr.5\linewidth-1.3em\relax}
		\centering
		\includegraphics[width=.95\linewidth,valign=t]{my_folder/images/spbpu_sc_refr}
	\end{subfigure}
\\[20pt]
	\adjustbox{minipage=1.3em,valign=t}{\subcaption{}\label{fig:spbpu_sc-c}}%
\begin{subfigure}[t]{\dimexpr.5\linewidth-1.3em\relax}
	\centering
	\includegraphics[width=.95\linewidth,valign=t]{my_folder/images/spbpu_sc_hall}
\end{subfigure}%
\hfill %выровнять по ширине
\adjustbox{minipage=1.3em,valign=t}{\subcaption{}\label{fig:spbpu_sc-d}}%
\begin{subfigure}[t]{\dimexpr.5\linewidth-1.3em\relax}
	\centering
	\includegraphics[width=.95\linewidth,valign=t]{my_folder/images/spbpu_sc_box}
\end{subfigure}
\captionsetup{justification=centering} %центрировать
\caption{Фотографии суперкомпьютерного центра СПбПУ \cite{spbpu-gallery}: {\itshape a} --- система хранения данных и узлы NUMA-вычислителя; {\itshape b} --- холодильные машины на крыше научно-исследовательского корпуса; {\itshape c} --- машинный зал; {\itshape d} --- элементы вычислительных устройств} 
\label{fig:spbpu_sc-four-photos}
\end{figure}

Далее можно ссылаться на составные части данного рисунка как на самостоятельные объекты: \firef{fig:spbpu_sc-a}, \firef{fig:spbpu_sc-b}, \firef{fig:spbpu_sc-c}, \firef{fig:spbpu_sc-d} или на три из четырёх изображений одновременно: рис.\labelcref{fig:spbpu_sc-a,fig:spbpu_sc-b,fig:spbpu_sc-c}. % пример подключения 4х иллюстраций в одном рисунке

%На \firef{fig:spbpu_whitehall-three-photos} приведены три картинки под~общим номером и~названием, но с раздельной нумерацией подрисунков посредством пакета \verb|subcaption|.
%
\begin{figure}[!htbp]
	\adjustbox{minipage=1.3em,valign=t}{\subcaption{}\label{fig:spbpu_whitehall-a}}%
	\begin{subfigure}[t]{\dimexpr.3\linewidth-1.3em\relax}
		\centering
		\includegraphics[width=.95\linewidth,valign=t]{my_folder/images//spbpu_whitehall}
	\end{subfigure}
	\hfill %выровнять
	\adjustbox{minipage=1.3em,valign=t}{\subcaption{}\label{fig:spbpu_whitehall-b}}%
	\begin{subfigure}[t]{\dimexpr.3\linewidth-1.3em\relax}
		\centering
		\includegraphics[width=.95\linewidth,valign=t]{my_folder/images//spbpu_whitehall_ligh}
	\end{subfigure}
	\hfill %выровнять
		\adjustbox{minipage=1.3em,valign=t}{\subcaption{}\label{fig:spbpu_whitehall-c}}%
	\begin{subfigure}[t]{\dimexpr.3\linewidth-1.3em\relax}
		\centering
		\includegraphics[width=.95\linewidth,valign=t]{my_folder/images//spbpu_whitehall_sculpture}
	\end{subfigure}%
\captionsetup{justification=centering} %центрировать
	\caption{Фотографии Белого зала СПбПУ \cite{spbpu-gallery}, в том числе: {\itshape a} --- со стороны зрителей; {\itshape b} --- со стороны сцены; {\itshape c} --- барельеф}\label{fig:spbpu_whitehall-three-photos}  
\end{figure}

Далее можно ссылаться на три отдельных рисунка: \firef{fig:spbpu_whitehall-a}, \firef{fig:spbpu_whitehall-b} и \firef{fig:spbpu_whitehall-c}. % пример подключения 3х иллюстрации в одном рисунке
%
%На \firef{fig:spbpu_main_bld-two-photos} приведены две картинки под~общим номером и~названием.


\begin{figure}[!htbp]
	\adjustbox{minipage=1.3em,valign=t}{\subcaption{}\label{fig:spbpu_main_bld_entrance_autumn}}%
	\begin{subfigure}[t]{\dimexpr.5\linewidth-1.3em\relax} %разрешили выделить 0,5 стр в ширину на рисунок
		\includegraphics[height=0.20\textheight,valign=t]{my_folder/images//spbpu_main_bld_entrance_autumn} %высоту рисунка выставили как 0,3 от высоты наборного поля
	\end{subfigure}
%	\hfill %выровнять по ширине
	\adjustbox{minipage=1.3em,valign=t}{\subcaption{}\label{fig:spbpu_main_bld_whitehall}}%
	\begin{subfigure}[t]{\dimexpr.5\linewidth-1.3em\relax}%разрешили выделить 0,5 стр в ширину на рисунок
		\includegraphics[height=0.20\textheight,valign=t]{my_folder/images//spbpu_main_bld_whitehall}%высоту рисунка выставили как 0,3 от высоты наборного поля
	\end{subfigure}
\captionsetup{justification=centering} %центрировать
	\caption{Вид на главное здание СПбПУ \cite{spbpu-gallery}, включая: {\itshape a} --- вход со стороны парка осенью; {\itshape b}~--- окна Белого зала}\label{fig:spbpu_main_bld-two-photos} 
\end{figure}

На \firef{fig:spbpu_main_bld_entrance_autumn} изображен вход со стороны парка СПбПУ осенью, а на \firef{fig:spbpu_main_bld_whitehall}~--- окна Белого зала. % пример подключения 2х иллюстраций в одном рисунке

Приведём пример табличного представления данных с записью продолжения на следующей странице на \taref{tab:long}.

%%% отладка longtable
%% 1) для контроля выхода таблицы за границы полей выставляем showframe в \geometry{}, см настройки
%% 2) используем \\* для запрета переноса определенной строки или средства из:
%% https://tex.stackexchange.com/q/344270/44348
%% 3) в крайнем случае для принудительного переноса таблицы на новую страницу используем \pagebreak после \\
\noindent % for correct centering
\begingroup
\centering
\small %выставляем шрифт в 12bp
\begin{longtable}[c]{|l|l|l|l|l|l|}
	\caption{Пример задания данных из \cite{Peskov2004} (с повтором для переноса таблицы на новую страницу)}%
	\label{tab:long}% label всегда желательно идти после caption
	\\
	\hline
	$G$&$m_1$&$m_2$&$m_3$&$m_4$&$K$\\ \hline
	1&2&3&4&5&6\\ \hline
	\endfirsthead%
	\captionsetup{format=tablenocaption,labelformat=continued} % до caption!
	\caption[]{}\\ % печать слов о продолжении таблицы
	\hline
	1&2&3&4&5&6\\ \hline
	\endhead
	\hline
	\endfoot
	\hline
	\endlastfoot
	$g_1$&0&1&1&0&1\\ \hline
	$g_2$&1&2&0&1&1\\ \hline
	$g_3$&0&1&0&1&1\\ \hline
	$g_4$&1&2&1&0&2\\ \hline
	$g_5$&1&1&0&1&2\\ \hline
	$g_6$&1&1&1&2&2\\ \hline
%
	$g_1$&0&1&1&0&1\\ \hline 
	$g_2$&1&2&0&1&1\\ \hline
	$g_3$&0&1&0&1&1\\ \hline
	$g_4$&1&2&1&0&2\\ \hline \noalign{\penalty-5000} % способствуем переносу на следующую стр
	$g_5$&1&1&0&1&2\\ \hline 
	$g_6$&1&1&1&2&2\\ \hline
%
	$g_1$&0&1&1&0&1\\ \hline 
	$g_2$&1&2&0&1&1\\ \hline
	$g_3$&0&1&0&1&1\\ \hline
	$g_4$&1&2&1&0&2\\ \hline
	$g_5$&1&1&0&1&2\\ \hline
	$g_6$&1&1&1&2&2\\ \hline
%		
	$g_1$&0&1&1&0&1\\ \hline 
	$g_2$&1&2&0&1&1\\ \hline
	$g_3$&0&1&0&1&1\\ \hline
	$g_4$&1&2&1&0&2\\ \hline
	$g_5$&1&1&0&1&2\\ \hline
	$g_6$&1&1&1&2&2\\ \hline
%
	$g_1$&0&1&1&0&1\\ \hline 
	$g_2$&1&2&0&1&1\\ \hline
	$g_3$&0&1&0&1&1\\ \hline
	$g_4$&1&2&1&0&2\\ \hline
	$g_5$&1&1&0&1&2\\ \hline
	$g_6$&1&1&1&2&2\\ \hline
%
	$g_1$&0&1&1&0&1\\ \hline 
	$g_2$&1&2&0&1&1\\ \hline
	$g_3$&0&1&0&1&1\\ \hline
	$g_4$&1&2&1&0&2\\ \hline
	$g_5$&1&1&0&1&2\\ \hline
	$g_6$&1&1&1&2&2\\ \hline
%
	$g_1$&0&1&1&0&1\\ \hline 
	$g_2$&1&2&0&1&1\\ \hline
	$g_3$&0&1&0&1&1\\ \hline
	$g_4$&1&2&1&0&2\\ \hline
	$g_5$&1&1&0&1&2\\ \hline
	$g_6$&1&1&1&2&2\\ \hline
\end{longtable}
\normalsize% возвращаем шрифт к нормальному
\endgroup % пример подключения таблицы на несколько страциц


\begin{table} [htbp]% Пример оформления таблицы
	\centering\small
	\caption{Пример представления данных для сквозного примера по ВКР \cite{Peskov2004}}%
	\label{tab:ToyCompare}		
		\begin{tabular}{|l|l|l|l|l|l|}
			\hline
			$G$&$m_1$&$m_2$&$m_3$&$m_4$&$K$\\
			\hline
			$g_1$&0&1&1&0&1\\ \hline
			$g_2$&1&2&0&1&1\\ \hline
			$g_3$&0&1&0&1&1\\ \hline
			$g_4$&1&2&1&0&2\\ \hline
			$g_5$&1&1&0&1&2\\ \hline
			$g_6$&1&1&1&2&2\\ \hline		
		\end{tabular}
%	\caption*{\raggedright\hspace*{2.5em} Составлено (или/и рассчитано) по \cite{Peskov2004}} %Если проведена авторская обработка или расчеты по какому-либо источнику	
	\normalsize% возвращаем шрифт к нормальному
\end{table}



%% please, before using, read the author guide carefully

\noindent % for correct centering
\begin{minipage}{\textwidth}
	\vspace{\mfloatsep} % интервал 
	\centering\small
	\captionof{table}{Пример задания данных в табличном виде из \cite{Peskov2004} (с помощью окружения minipage)}%
	\label{tab:ToyCompare-Peskov-minipage}
	\begin{tabular}{|l|l|l|l|l|l|}
	\hline
	$G$&$m_1$&$m_2$&$m_3$&$m_4$&$K$\\
	\hline
	$g_1$&0&1&1&0&1\\ \hline
	$g_2$&1&2&0&1&1\\ \hline
	$g_3$&0&1&0&1&1\\ \hline
	$g_4$&1&2&1&0&2\\ \hline
	$g_5$&1&1&0&1&2\\ \hline
	$g_6$&1&1&1&2&2\\ \hline
	\hline		
	\end{tabular}
\vspace{\mfloatsep} % интервал 
\normalsize %восстанавливаем шрифт 	
\end{minipage} % пример подключения minipage

\noindent % for correct centering
\begin{minipage}{\textwidth}
	\centering
	\vspace{\mfloatsep} % интервал  	
	\includegraphics[keepaspectratio=true,scale=0.27] {my_folder/images/spbpu_new_bld_autumn}
	\captionof{figure}{Новый научно-исследовательский корпус СПбПУ \cite{spbpu-gallery} (с помощью окружения minipage)}\label{fig:spbpu-new-bld-autumn-minipage}  
	\vspace{\mfloatsep} % интервал  	
\end{minipage} % пример подключения minipage




Вопросы форматирования текстово-графических объектов (окружений) не регламентированы в известных нам ГОСТах, поэтому предлагаем придерживаться следующих правил:

\begin{itemize}
	\item \textbf{полужирный текст} рекомендуем использовать только для названий стандартных окружений с нумерационной частью, например, для представления \textit{впервые}: \textbf{определение 1.1}, \textbf{теорема 2.2}, \textbf{пример 2.3}, \textbf{лемма 4.5};
	
	\item \textit{курсив} рекомендуем использовать только для выделения переменных в формулах, служебной информации об авторах главы (статьи), важных терминов, представляемых по тексту, а также для всего тела окружений, связанных с получением \textit{новых существенных результатов и их доказательством}: теорема, лемма, следствие, утверждение и другие.
\end{itemize}

 

По аналогии с нумерацией формул, рисунков и таблиц нумеруются и иные текстово-графические объекты, то есть включаем в нумерацию номер главы, например: теорема 3.1. для первой теоремы третьей главы монографии. Команды \LaTeX{} выставляют нумерацию и форматирование автоматически. Полный перечень команд для подготовки текстово-графических и иных объектов находится в подробных методических рекомендациях \cite{spbpu-bci-template-author-guide}. 


Для удобства авторов названия стандартных окружений, рекомендованных к использованию, приведены в \taref{tab:enum-std}, а в \taref{tab:enum-spbpu}  перечислены имена специально разработанных окружений для шаблонов SPbPU.

% и примеры их оформления на псевдокоде (см. \cite{cite-spbpu-bci}).


%https://tex.stackexchange.com/questions/2651/should-i-use-center-or-centering-for-figures-and-tables


	\begin{table} [htbp]% Пример записи таблицы с номером, но без отображаемого наименования
	\centering\small
	\caption{Стандартные окружения}%
	\label{tab:enum-std}
	 \begin{Spacing}{\Single} % Одинарный интервал между строками текста 
	  \renewcommand*{\arraystretch}{1.5} % Полуторный интервал между ячейками таблицы
		\begin{tabular}{|l|p{11cm}|} 
			\hline
			Название окружения&Назначение\\
			\hline
			\verb|center| &	центрирование, аналог команды \verb|\centering|, но с добавлением нежелательного пробела, поэтому лучше избегать применения \verb|center|\\ \hline
			\verb|itemize| &{перечисления, в которых нет необходимости нумеровать  пункты (немаркированные списки)} \\ \hline
			\verb|enumerate| & перечисления с нумерацией (немаркированные списки) \\ \hline
			\verb|refsection| & создание отдельных библиографических списков для глав \\ \hline
			\verb|tabular| & оформление таблиц \\ \hline
			\verb|table|   &{автоматическое перемещение по тексту таблиц, оформленных, например, с помощью \verb|tabular|, для минимизации пустых пространств} \\ \hline
			\verb|longtable| & оформление многостраничных таблиц \\ \hline
			\verb|tikzpicture| & создание иллюстраций с помощью пакета \verb|tikz| \cite{ctan-tikz} \\ \hline
			\verb|figure| &{автоматическое перемещение по тексту рисунков, оформленных например, с помощью \verb|tikz| или подключенных с помощью команды \verb|\includegraphics|, для минимизации пустых пространств} \\ \hline 
			\verb|subfigure| & оформление вложенных рисунков в составе \verb|figure| \\ \hline
			\verb|algorithm| &{оформление псевдокода на основе пакета \verb|algorithm2e| \cite{ctan-algorithm2e}} \\ \hline
			\verb|minipage| & {оформление рисунков и таблиц без функций автоматического перемещения по тексту для  минимизации пустых пространств} \\ \hline
			\verb|equation| & {оформление выключенных (не встроенных в текст с помощью \verb|$...$|) одиночных формул на одной строке} \\ \hline
			\verb|multilined| &{оформление выключенных (не встроенных в текст с помощью \verb|$...$|) одиночных формул в несколько строк} \\ \hline 
			\verb|aligned| &{оформление нескольких формул с выравниванием по символу \verb|&|.} \\ \hline
	\end{tabular}
	\end{Spacing}
%	\normalsize
	\end{table}

На базе пакета \verb|tikz| разработано большое количество расширений \cite{ctan-tikz}, например, \verb|tikzcd|, которые мы рекомендуем использовать для оформления иллюстраций.

	\begin{table} [htbp]% Пример записи таблицы с номером, но без отображаемого наименования
	\centering\small
	\caption{Специальные окружения}%
	\label{tab:enum-spbpu}
		\begin{tabular}{|l|l|}
			\hline
			Название окружения & Текстово-графический объект\\
			\hline
			\verb|abstr|	 & реферат (abstract) \\ \hline
			\verb|m-theorem| & теорема \\ \hline 
			\verb|m-corollary| & следствие \\ \hline
			\verb|m-proposition| & утверждение \\ \hline
			\verb|m-lemma|   & лемма \\ \hline
			\verb|m-axiom| & аксиома \\ \hline
			\verb|m-example| & пример \\ \hline
			\verb|m-definition| &  определение \\ \hline
			\verb|m-condition| & условие \\ \hline
			\verb|m-problem| & проблема \\ \hline
			\verb|m-exercise| & упраженение \\ \hline
			\verb|m-question| & вопрос \\ \hline
			\verb|m-hypothesis| & гипотеза \\ \hline
		\end{tabular}	
	\normalsize
\end{table}

В случае, если авторам потребовалось новое окружение, то создать его можно в файле в файле \texttt{my\_fol\-der/{}my\_set\-tings.tex} согласно правилам, приведённым ниже.

\begin{enumerate}[1.]
	\item Для перехода в режим создания окружений следует указать:
	\begin{itemize}
		\item \verb|\theoremstyle{myplain}| --- окружения с доказательствами или аксиомами
		\item \verb|\theoremstyle{mydefinition}| --- окружения, не связанные с доказательствами или аксиомами.
	\end{itemize}
	\item В команде создания окружения следует ввести краткий псевдоним (\verb|m-new-env|) и отображаемое в pdf имя окружения (\verb|Название_окружения|):
	\begin{itemize}
		\item \texttt{\textbackslash{}newtheorem\{m-new-env-second\}\{Название\_окруже\-ния\}\-[chap\-ter]}.
	\end{itemize}
\end{enumerate}


%\begin{m-new-env-first}
%	Тест первого пользовательского окружения
%\end{m-new-env-first}
%
%\begin{m-new-env-second}
%	Тест второго пользовательского окружения
%\end{m-new-env-second} % список некоторых окружений


\begin{m-theorem}[о чем-то конкретном] %при необходимости в [] можно записать название теоремы или убрать его
	\label{th:ex} 
	% \index только для принятых работ
	% шаблон записи теоремы в Предметный указатель
	\index[ru]{теорема!название\_теоремы или о чём} %ключевое слово <<теорема>> не менять
	\index[en]{theorem!1-3 words for detail or description}
	% пример записи алгоритма в Предметный указатель
	\index[ru]{теорема!о неполноте}
	\index[en]{theorem!about incompleteness}
	% пример записи алгоритма в Предметный указатель
	\index[ru]{теорема!о жизни}
	\index[en]{theorem!about life}
	Текст теоремы полностью выделен курсивом. Допустимо математические символы не выделять курсивом, если это искажает их значения. Используется абзацный отсуп, так как ``Абзацы в тексте начинают отступом'' в соответствии с ГОСТ 2.105--95. Название теоремы допустимо убрать. Доказательство окончено.
\end{m-theorem}
Доказательство теоремы \ref{th:ex}, леммы, утверждений, следствий и других подобных окружений (в последнем абзаце) завершаем предложением в котором сказано, что доказательство окончено. Например, доказательство теоремы \ref{th:ex} окончено.

Тело доказательства не выделяется курсивом.
Тело следующих окружений также не выделяется сплошным курсивом: определение, условие, проблема, пример, упражнение, вопрос, гипотеза и другие. %пример оформления теоремы


\begin{m-definition}[термин] %при необходимости в [] можно записать название определения или убрать его
	\label{def:ex}
	% \index только для принятых работ
	% шаблон записи определения в Предметный указатель 
	\index[ru]{название\_определения!1-3 уточняющих слова или~ничего}
	\index[en]{definition\_title!1-3 words for detail or~without "!-part}
	% пример записи определения в Предметный указатель 
	\index[ru]{и-тест!хороший!наилучший}
	\index[en]{i-test!good!best}
	% пример записи определения в Предметный указатель 
	\index[ru]{и-тест!замкнутый}
	\index[en]{i-test!closed}
	В тексте определения только {\itshape важные термины} выделяются курсивом. Если определение носит лишь вспомогательный характер, то допустимо не использовать окружение \texttt{m-definition}, представляя текст определения в обычном абзаце. Ключевые термины при этом обязательно выделяются курсивом.
\end{m-definition} %пример оформления определения


Вместо теоремо-подобных окружений для вставки небольших текстово-графических объектов иногда используются команды. Типичным примером такого подхода является команда \verb|\footnote{text}|\footnote{Внимание! Команда вставляется непосредственно после слова, куда вставляется сноска (без пробела). Лишние пробелы также не указываются внутри команды перед и после фигурных скобок.}, где в аргументе \verb|text| указывают текст \textit{подстрочной ссылки (сноски)}.В них \textit{нельзя добавлять веб-ссылки или цитировать литературу}. Для этих целей используется список литературы. Нумерация сносок сквозная по ВКР без точки на конце выставляется в шаблоне автоматически, однако в каждом приложении к ВКР нумерация, зависящая от номера приложения, выставляется префикс <<П>>, например <<П1.1>> --- первая сноска первого приложения. 




%\FloatBarrier % заставить рисунки и другие подвижные (float) элементы остановиться


\section{Выводы} \label{ch2:conclusion}

Текст заключения ко второй главе. Пример ссылок \cite{Article,Book,Booklet,Conference,Inbook,Incollection,Manual,Mastersthesis,Misc,Phdthesis,Proceedings,Techreport,Unpublished,badiou:briefings}, а также ссылок с указанием страниц, на котором отображены те или иные текстово-графические объекты  \cite[с.~96]{Naidenova2017} или в виде мультицитаты на несколько источников \cites[с.~96]{Naidenova2017}[с.~46]{Ganter1999}. Часть библиографических записей носит иллюстративный характер и не имеет отношения к реальной литературе. 

Короткое имя каждого библиографического источника содержится в специальном файле \verb|my_biblio.bib|, расположенном в папке \verb|my_folder|. Там же находятся исходные данные, которые с помощью программы \texttt{Biber} и стилевого файла \texttt{Biblatex-GOST} \cite{ctan-biblatex-gost} приведены в списке использованных источников согласно ГОСТ 7.0.5-2008.
Многообразные реальные примеры исходных библиографических данных можно посмотреть по ссылке \cite{ctan-biblatex-gost-examples}.

Как правило, ВКР должна состоять из четырех глав. Оставшиеся главы можно создать по образцу первых двух и подключить с помощью команды \verb|\input| к исходному коду ВКР. Далее в приложении \ref{appendix-MikTeX-TexStudio} приведены краткие инструкции запуска исходного кода ВКР \cite{latex-miktex,latex-texstudio}.

В приложении \ref{appendix-extra-examples} приведено подключение некоторых текстово-графических объектов. Они оформляются по приведенным ранее правилам. В качестве номера структурного элемента вместо номера главы используется <<П>> с номером главы. Текстово-графические объекты из приложений не учитываются в реферате.



%% Вспомогательные команды - Additional commands
%
%\newpage % принудительное начало с новой страницы, использовать только в конце раздела
%\clearpage % осуществляется пакетом <<placeins>> в пределах секций
%\newpage\leavevmode\thispagestyle{empty}\newpage % 100 % начало новой страницы	         	 % Глава 2
\chapter{Название третьей главы: разработка программного обеспечения} \label{ch3}

% не рекомендуется использовать отдельную section <<введение>> после лета 2020 года
%\section{Введение} \label{ch3:intro}

Хорошим стилем является наличие введения к главе. Во введении может быть описана цель написания главы, а также приведена краткая структура главы. 
	
\section{Название параграфа} \label{ch3:sec1}

\section{Название параграфа} \label{ch3:sec2}

%\FloatBarrier % заставить рисунки и другие подвижные (float) элементы остановиться


\section{Выводы} \label{ch3:conclusion}

Текст выводов по главе \thechapter.


%% Вспомогательные команды - Additional commands
%
%\newpage % принудительное начало с новой страницы, использовать только в конце раздела
%\clearpage % осуществляется пакетом <<placeins>> в пределах секций
%\newpage\leavevmode\thispagestyle{empty}\newpage % 100 % начало новой страницы           	 % Глава 3
\chapter{Название четвёртой главы. Апробация результатов исследования, а~именно: метода, алгоритма, модели исследования} \label{ch4}

% не рекомендуется использовать отдельную section <<введение>> после лета 2020 года
%\section{Введение} \label{ch4:intro}

Хорошим стилем является наличие введения к главе. Во введении может быть описана цель написания главы, а также приведена краткая структура главы. 
	
\section{Название параграфа} \label{ch4:sec1}

\section{Название параграфа} \label{ch4:sec2}

Пример ссылки на литературу \cite{avtonomova:fya,Peskov2004-ru,Kotelnikov2004-ru,Kotelnikov2004}.

%\FloatBarrier % заставить рисунки и другие подвижные (float) элементы остановиться

\section{Выводы} \label{ch4:conclusion}

Текст выводов по главе \thechapter.

%% Вспомогательные команды - Additional commands
%
%\newpage % принудительное начало с новой страницы, использовать только в конце раздела
%\clearpage % осуществляется пакетом <<placeins>> в пределах секций
%\newpage\leavevmode\thispagestyle{empty}\newpage % 100 % начало новой страницы           	 % Глава 3
\ContinueChapterEnd % завершить размещение глав <<подряд>>
%% Завершение основной части

\chapter*{Заключение} \label{ch-conclusion}
\addcontentsline{toc}{chapter}{Заключение}	% в оглавление 

Заключение (2 -- 5 страниц) обязательно содержит выводы по теме работы, \textit{конкретные
предложения и рекомендации} по исследуемым вопросам. Количество общих выводов
должно вытекать из количества задач, сформулированных во введении выпускной
квалификационной работы.

Предложения и рекомендации должны быть органически увязаны с выводами
и направлены на улучшение функционирования исследуемого объекта. При разработке
предложений и рекомендаций обращается внимание на их обоснованность,
реальность и практическую приемлемость.

Заключение не должно содержать новой информации, положений, выводов и
т. д., которые до этого не рассматривались в выпускной квалификационной работе.
Рекомендуется писать заключение в виде тезисов.

Последним абзацем в заключении можно выразить благодарность всем людям, которые помогали автору в написании ВКР.        	 % Заключение

%% Наличие следующих перечней не исключает расшифровку сокращения и условного обозначения при первом упоминании в тексте!
\chapter*{Список сокращений и условных обозначений}             % Заголовок
\addcontentsline{toc}{chapter}{Список сокращений и условных обозначений}  % Добавляем его в оглавление
\noindent
\addtocounter{table}{-1}% Нужно откатить на единицу счетчик номеров таблиц, так как следующая таблица сделана для удобства представления информации по ГОСТ
%\begin{longtabu} to \dimexpr \textwidth-5\tabcolsep {r X}
\begin{longtabu} to \textwidth {r X} % Таблицу не прорисовываем!
% Жирное начертание для математических символов может иметь
% дополнительный смысл, поэтому они приводятся как в тексте
% диссертации
\textbf{DOI} & Digital Object Identifier. \\
\textbf{WoS} & Web of Science. \\
\textbf{ВКР}  & Выпускная квалификационная работа. \\
\textbf{ТГ-объект}  & Текстово-графический объект. \\
%$\begin{rcases}
%a_n\\
%b_n
%\end{rcases}$  & 
%\begin{minipage}{\linewidth}
%Коэффициенты разложения Ми в дальнем поле, соответствующие
%электрическим и магнитным мультиполям.
%\end{minipage}
%\\
%${\boldsymbol{\hat{\mathrm e}}}$ & Единичный вектор. \\
%$E_0$ & Амплитуда падающего поля.\\
%$\begin{rcases}
%a_n\\
%b_n
%\end{rcases}$  & 
%Коэффициенты разложения Ми в дальнем поле соответствующие
%электрическим и магнитным мультиполям ещё раз, но без окружения
%minipage нет вертикального выравнивания по центру.
%\\
%$j$ & Тип функции Бесселя.\\
%$k$ & Волновой вектор падающей волны.\\
%
%$\begin{rcases}
%a_n\\
%b_n
%\end{rcases}$  & 
%\begin{minipage}{\linewidth}
%\vspace{0.7em}
%Коэффициенты разложения Ми в дальнем поле соответствующие
%электрическим и магнитным мультиполям, теперь окружение minipage есть
%и добавленно много текста, так что описание группы условных
%обозначений значительно превысило высоту этой группы... Для отбивки
%пришлось добавить дополнительные отступы.
%\vspace{0.5em}
%\end{minipage}
%\\
%$L$ & Общее число слоёв.\\
%$l$ & Номер слоя внутри стратифицированной сферы.\\
%$\lambda$ & Длина волны электромагнитного излучения
%в вакууме.\\
%$n$ & Порядок мультиполя.\\
%$\begin{rcases}
%{\mathbf{N}}_{e1n}^{(j)}&{\mathbf{N}}_{o1n}^{(j)}\\
%{\mathbf{M}_{o1n}^{(j)}}&{\mathbf{M}_{e1n}^{(j)}}
%\end{rcases}$  & Сферические векторные гармоники.\\
%$\mu$  & Магнитная проницаемость в вакууме.\\
%$r,\theta,\phi$ & Полярные координаты.\\
%$\omega$ & Частота падающей волны.\\
%
%  \textbf{BEM} & Boundary element method, метод граничных элементов.\\
%  \textbf{CST MWS} & Computer Simulation Technology Microwave Studio.
\end{longtabu}
		         % Необязательная рубрика! Список сокращений и условных обозначений

\chapter*{Словарь терминов}             % Заголовок
\addcontentsline{toc}{chapter}{Словарь терминов}  % Добавляем его в оглавление

\textbf{TeX} --- язык вёрстки текста и издательская система, разработанные Дональдом Кнутом.

\textbf{LaTeX} --- язык вёрстки текста и издательская система, разработанные Лэсли Лампортом как надстройка над TeX.

    		 % Необязательная рубрика! Словарь терминов
% По порядку после Списка сокращений и условных обозначений, если есть.	


%%% Не мянять - Do not modify
%%
%%
\clearpage                                  % В том числе гарантирует, что список литературы в оглавлении будет с правильным номером страницы
%\hypersetup{ urlcolor=black }               % Ссылки делаем чёрными
%\providecommand*{\BibDash}{}                % В стилях ugost2008 отключаем использование тире как разделителя 
\urlstyle{rm}                               % ссылки URL обычным шрифтом
\ifdefmacro{\microtypesetup}{\microtypesetup{protrusion=false}}{} % не рекомендуется применять пакет микротипографики к автоматически генерируемому списку литературы
%\newcommand{\fullbibtitle}{Список литературы} % (ГОСТ Р 7.0.11-2011, 4)
%\insertbibliofull  
%\noindent
%\begin{group}
\chapter*{Список использованных источников}	
\label{references}
\addcontentsline{toc}{chapter}{Список использованных источников}	% в оглавление 
\printbibliography[env=SSTfirst]                         % Подключаем Bib-базы
%\ifdefmacro{\microtypesetup}{\microtypesetup{protrusion=true}}{}
%\urlstyle{tt}                               % возвращаем установки шрифта ссылок URL
%\hypersetup{ urlcolor={urlcolor} }          % Восстанавливаем цвет ссылок



%\urlstyle{rm}                               % ссылки URL обычным шрифтом
%\ifdefmacro{\microtypesetup}{\microtypesetup{protrusion=false}}{} % не рекомендуется применять пакет микротипографики к автоматически генерируемому списку литературы
%\insertbibliofull                           % Подключаем Bib-базы
%\ifdefmacro{\microtypesetup}{\microtypesetup{protrusion=true}}{}
%\urlstyle{tt}                               % возвращаем установки шрифта ссылок URL
		     % Список литературы

% Здесь можно поместить список иллюстративного материала

\appendix % не редактировать / keep unmodified


\chapter{Краткие инструкции по настройке издательской системы \LaTeX}\label{appendix-MikTeX-TexStudio}							% Заголовок
%\addcontentsline{toc}{chapter}{Second call for chapters to participate in the book Machine learning in analysis of biomedical and socio-economic data}	% Добавляем его в оглавление

В SPbPU-BCI-template {\itshape автоматически выставляются необходимые настройки и в исходном тексте шаблона приведены примеры оформления текстово-графических объектов}, поэтому авторам достаточно заполнить имеющийся шаблон текстом главы (статьи), не вдаваясь в детали оформления, описанные далее. Возможный <<быстрый старт>> оформления главы (статьи) под Windows следующий\footnote{Внимание! Пример оформления подстрочной ссылки (сноски).}:

\begin{enumerate}
	\item Установка полной версии MikTeX  \cite{latex-miktex}.  В процессе установки лучше выставить параметр доустановки пакетов <<на лету>>.
	
	\item Установка TexStudio \cite{latex-texstudio}.
	
%		\item установка шрифтов PSCyr для работы с TimesNew\-Roman\-PSMT  	\href{https://github.com/AndreyAkinshin/Russian-Phd-LaTeX-Dissertation-Template/blob/master/PSCyr/Windows.md}{по данной инструкции}. В итоговом документе будет, скорее всего, использован Newton.
	
%	\item Переименование следующих файлов, где вместо \texttt{AuthorsSur\-names} необходимо подставить фамилии авторов (можно сокращать до первых четырех букв): 
%	
%	\begin{enumerate}
%		\item Основной файл \texttt{Book\_title\_ch\_Authors\-Sur\-names.tex}.
%		\item Библиография \texttt{biblio\textbackslash{}Book\_title\_bib\_Authors\-Sur\-na\-mes\-.bib}.
%		\item Пользовательские настройки (при необходимости), \texttt{common\textbackslash{}Book\_\-tit\-le\_ext\_Authors\-Sur\-names.tex}. 
%	\end{enumerate}
%	
%	\item После открытия основного файла \texttt{Book\_title\_ch\_Authors\-Sur\-names.tex} (с новым названием)   переименовать названия по аналогии в следующих командах \texttt{\textbackslash{}input\{\}}:
%	
%	\begin{enumerate}
%		\item \texttt{biblio/Book\_title\_bib\_Authors\-Sur\-names.bib},
%		\item \texttt{common/Book\_title\_ext\_Authors\-Sur\-names.tex (при необходимости) }.
%	\end{enumerate}
%	
	
	\item Запуск TexStudio и компиляция \verb|my_chapter.tex| с помощью команды <<Build\&View>> (например, с помощью двойной зелёной стрелки в верхней панели). {\itshape Иногда, для достижения нужного результата необходимо несколько раз скомпилировать документ.}
	
	\item В случае, если не отобразилась библиография, можно
	
	\begin{itemize}
		\item воспользоваться командой Tools $\to$ Commands $\to$ Biber, затем запустив Build\&View;
		
		\item настроить автоматическое включение библиографии в настройках Options $\to$ Configure TexStudio $\to$ Build $\to$  Build\&View (оставить по умолчанию, если сборка происходит слишком долго): \texttt{txs:///pdflatex | txs:///biber | txs:///pdflatex | txs:///pdflatex | txs:///\-view-pdf}.
	\end{itemize}
	
\end{enumerate}

В случае возникновения ошибок, попробуйте скомпилировать документ до последних действий или внимательно ознакомьтесь с описанием проблемы в log-файле. Бывает полезным переход (по подсказке TexStudio) в нужную строку в pdf-файле или запрос с текстом ошибке в поисковиках. Наиболее вероятной проблемой при первой компиляции может быть отсутствие какого-либо установленного пакета \LaTeX. 

В случае корректной работы настройки <<установка на лету>> все дополнительные пакеты будут скачиваться и устанавливаться в автоматическом режиме. Если доустановка пакетов осуществляется медленно (несколько пакетов за один запуск компилятора), то можно попробовать установить их в ручном режиме следующим образом:

\begin{enumerate}[1.]
	\item Запустите программу: меню $\to$ все программы $\to$ MikTeX $\to$ Maintenance (Admin) $\to$ MiKTeX Package Manager (Admin).
	\item Пользуясь поиском, убедитесь, что нужный пакет присутствует, но не установлен (если пакет отсутствует воспользуйтесь сначала MiKTeX Update (Admin)).
	\item Выделив строку с пакетом (возможно выбрать несколько или вообще все неустановленные пакеты), выполните установку Tools $\to$ Install или с помощью контекстного меню.
	\item После завершения установки запустите программу MiKTeX Settings (Admin).
	\item Обновите базу данных имен файлов Refresh FNDB.
\end{enumerate}


Для проверки текста статьи на русском языке полезно также воспользоваться настройками Options $\to$ Configure TexStudio $\to$ Language Checking $\to$  Default Language. Если русский язык <<ru\_RU>> не будет доступен в меню выбора, то необходимо вначале выполнить Import Dictionary, скачав из интернета любой русскоязычный словарь. 


%\chapter{\normalfont\normalsize{}Часто задаваемые вопросы (FAQ)}\label{Appendix-FAQ}							% Заголовок
%%\addcontentsline{toc}{chapter}{Second call for chapters to participate in the book Machine learning in analysis of biomedical and socio-economic data}	% Добавляем его в оглавление


Далее приведены формулы \eqref{eq:Pi-app2}, \eqref{eq:Pi-app2-},  \firef{fig:spbpu_hydrotower-app2}, \firef{fig:spbpu_hydrotower-app2-}, \taref{tab:ToyCompare-app2}, \taref{tab:ToyCompare-app2-}.


\begin{equation}% лучше не оставлять пропущенную строку (\par) перед окружениями для избежания лишних отсупов в pdf
\label{eq:Pi-app2-} % eq - equations, далее название, ch поставлено для избежания дублирования
\pi \approx 3,141.
\end{equation}

%
\begin{figure}[ht!] 
	\center
	\includegraphics [scale=0.27] {my_folder/images//spbpu_hydrotower}
	\caption{Вид на гидробашню СПбПУ \cite{spbpu-gallery}} 
	\label{fig:spbpu_hydrotower-app2-}  
\end{figure}

\begin{table} [htbp]% Пример оформления таблицы
	\centering\small
	\caption{Представление данных для сквозного примера по ВКР \cite{Peskov2004}}%
	\label{tab:ToyCompare-app2-}		
	\begin{tabular}{|l|l|l|l|l|l|}
		\hline
		$G$&$m_1$&$m_2$&$m_3$&$m_4$&$K$\\
		\hline
		$g_1$&0&1&1&0&1\\ \hline
		$g_2$&1&2&0&1&1\\ \hline
		$g_3$&0&1&0&1&1\\ \hline
		$g_4$&1&2&1&0&2\\ \hline
		$g_5$&1&1&0&1&2\\ \hline
		$g_6$&1&1&1&2&2\\ \hline		
	\end{tabular}	
	\normalsize% возвращаем шрифт к нормальному
\end{table}




\section{Параграф приложения}\label{app-2-1}							


\subsection{Название подпараграфа} \label{ch2:subsec-title-abbr} %название по-русски


Название подпараграфа оформляется с помощью команды  \texttt{\textbackslash{}subsection\{...\}}.

Использование подподпараграфов в основной части крайне не рекомендуется.
\subsubsection{Название подподпараграфа}\label{ch2:subsubsec-title-abbr} %название по-русски

\begin{equation}% лучше не оставлять пропущенную строку (\par) перед окружениями для избежания лишних отсупов в pdf
\label{eq:Pi-app2} % eq - equations, далее название, ch поставлено для избежания дублирования
\pi \approx 3,141.
\end{equation}
%
%
\begin{figure}[ht!] 
	\center
	\includegraphics [scale=0.27] {my_folder/images//spbpu_hydrotower}
	\caption{Вид на гидробашню СПбПУ \cite{spbpu-gallery}} 
	\label{fig:spbpu_hydrotower-app2}  
\end{figure}
%




\begin{table}[t!]% Пример оформления таблицы
	\centering\small
	\caption{Представление данных для сквозного примера по ВКР \cite{Peskov2004}}%
	\label{tab:ToyCompare-app2}		
	\begin{tabular}{|l|l|l|l|l|l|}
		\hline
		$G$&$m_1$&$m_2$&$m_3$&$m_4$&$K$\\
		\hline
		$g_1$&0&1&1&0&1\\ \hline
		$g_2$&1&2&0&1&1\\ \hline
		$g_3$&0&1&0&1&1\\ \hline
		$g_4$&1&2&1&0&2\\ \hline
		$g_5$&1&1&0&1&2\\ \hline
		$g_6$&1&1&1&2&2\\ \hline		
	\end{tabular}	
	\normalsize% возвращаем шрифт к нормальному
\end{table}


%% В случае, когда таблица (рисунок) размещаются на последней странице, для переноса названия приложения на новую строку используем:
\NewPage % начать новое приложение с новой страницы 			     % Приложение 1

\chapter{Некоторые дополнительные примеры}\label{appendix-extra-examples}							% 

В приложении приведены формулы \eqref{eq:Pi-app}, \eqref{eq:Pi-app-}, \firef{fig:spbpu_hydrotower-app}, \firef{fig:spbpu_hydrotower-app-}, \taref{tab:ToyCompare-app}, \taref{tab:ToyCompare-app-}\footnote{Внимание! Пример оформления подстрочной ссылки (сноски).}.


\begin{equation}% лучше не оставлять пропущенную строку (\par) перед окружениями для избежания лишних отсупов в pdf
\label{eq:Pi-app-} % eq - equations, далее название, ch поставлено для избежания дублирования
\pi \approx 3,141.
\end{equation}
%
%
\begin{figure}[ht!] 
	\center
	\includegraphics [scale=0.27] {my_folder/images//spbpu_hydrotower}
	\caption{Вид на гидробашню СПбПУ \cite{spbpu-gallery}} 
	\label{fig:spbpu_hydrotower-app-}  
\end{figure}

\begin{table} [htbp]% Пример оформления таблицы
	\centering\small
	\caption{Представление данных для сквозного примера по ВКР \cite{Peskov2004}}%
	\label{tab:ToyCompare-app-}		
	\begin{tabular}{|l|l|l|l|l|l|}
		\hline
		$G$&$m_1$&$m_2$&$m_3$&$m_4$&$K$\\
		\hline
		$g_1$&0&1&1&0&1\\ \hline
		$g_2$&1&2&0&1&1\\ \hline
		$g_3$&0&1&0&1&1\\ \hline
		$g_4$&1&2&1&0&2\\ \hline
		$g_5$&1&1&0&1&2\\ \hline
		$g_6$&1&1&1&2&2\\ \hline		
	\end{tabular}	
	\normalsize% возвращаем шрифт к нормальному
\end{table}




\section{Подраздел приложения}\label{app-2-1}							


\begin{equation}% лучше не оставлять пропущенную строку (\par) перед окружениями для избежания лишних отсупов в pdf
\label{eq:Pi-app} % eq - equations, далее название, ch поставлено для избежания дублирования
\pi \approx 3,141.
\end{equation}
%
%
\begin{figure}[ht!] 
	\center
	\includegraphics [scale=0.27] {my_folder/images//spbpu_hydrotower}
	\caption{Вид на гидробашню СПбПУ \cite{spbpu-gallery}} 
	\label{fig:spbpu_hydrotower-app}  
\end{figure}

\begin{table} [htbp]% Пример оформления таблицы
	\centering\small
	\caption{Представление данных для сквозного примера по ВКР \cite{Peskov2004}}%
	\label{tab:ToyCompare-app}		
	\begin{tabular}{|l|l|l|l|l|l|}
		\hline
		$G$&$m_1$&$m_2$&$m_3$&$m_4$&$K$\\
		\hline
		$g_1$&0&1&1&0&1\\ \hline
		$g_2$&1&2&0&1&1\\ \hline
		$g_3$&0&1&0&1&1\\ \hline
		$g_4$&1&2&1&0&2\\ \hline
		$g_5$&1&1&0&1&2\\ \hline
		$g_6$&1&1&1&2&2\\ \hline		
	\end{tabular}	
	\normalsize% возвращаем шрифт к нормальному
\end{table}

			 	 % Приложение 2


\end{document} % конец документа


%%% Удачной защиты ВКР! - Good luck on the thesis defense!
%%
%%% Поддержать проект
%%
%% Запросы на добавление / изменение просим писать на следующей странице:
%% https://github.com/ParkhomenkoV/SPbPU-student-thesis-template/issues
%%
%% Список пожеланий в файле шаблона <<TO-DO-list.tex>>
%%
%% Благодарности просим указывать в виде 
%%
%% 1. Добавление <<Звезды>> проекту https://github.com/ParkhomenkoV/SPbPU-student-thesis-template/stargazers
%%
%% 2. Добавления <<Сердечка>> и репоста проекта в социальных сетях:
%%		https://vk.com/latex_polytech 
%%		https://www.fb.com/groups/latex.polytech
%%

%%% Support project
%%
%% Requests on adding / modifications is better to be publishen on the following web-page:
%% https://github.com/ParkhomenkoV/SPbPU-student-thesis-template/issues
%%
%% Wishlist is in the template's file called <<TO-DO-list.tex>>
%%
%% Acknowledgements are better to be done in the form of 
%%
%% 1. Adding <<Star>> to the project https://github.com/ParkhomenkoV/SPbPU-student-thesis-template/stargazers
%%
%% 2. Adding <<Likes>> and Project repost in the social networks:
%%		https://vk.com/latex_polytech 
%%		https://www.fb.com/groups/latex.polytech
%% 

% Check list при передаче ВКР:
% - Количество страниц в Задании 2. Если нет, то комментирование последней строки в my_task.tex
% - Зачистка всех вспомогательных файлов (Clear auxilary files) и компиляция ВКР не менее 3х раз